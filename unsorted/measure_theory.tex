\documentclass[12pt]{article}

\usepackage{../preamble}

\title{Measure Theory Crash Course}
\author{Runi Malladi}

\begin{document}
\maketitle

\section{motivation} % {{{

Lebesgue integration can be thought of as the ``completion'' of Riemann integration. This is because
Riemann integrals do not behave well at the limit, whereas Lebesgue integrals do. 

\begin{example}[uniform limit of Riemann integrable functions is Riemann integrable]
	TODO
\end{example}

\begin{example}[pointwise limit of Riemann integrable functions may fail to be Riemann integrable]
	TODO
\end{example}

% }}}

\section{Lebesgue measure} 

\subsection{Tao's approach}

This approach essentially takes Littlewood's first principle to be its definition: it definines 
``Lebesgue outer measure'' $m^\ast$ and then calls a set $E$ Lebesgue measurable if, for any 
$\epsilon>0$, we can find an open set $U$ such that $m^ast(U\setminus E) < \epsilon$, and defines 
the Lebesgue measure of $E$ in that case to be equal to $m^\ast(E)$. 

\begin{definition}[Lebesgue outer measure]
	TODO
\end{definition}

So in some sense, $m^\ast(E)$ is the smallest area we can get by covering $E$ with countably many boxes
(actually the infimum, of course).

\begin{definition}[Lebesgue measure]
	TODO
\end{definition}

\section{properties of measurability} % {{{

\subsection{outer measure properties}

\begin{proposition}[axioms of outer measures] Let $m^\ast$ be an outer measure. \hfill
	\begin{enumerate}
		\item $m^\ast(\emptyset)=0$
		\item (monotonicity)
		\item (countable subadditivity) If $(E_n)$ is a countable sequence of sets, then $m^\ast(\bigcup_n E_n) \leq \sum_n m^\ast(E_n)$.
	\end{enumerate}
\end{proposition}

\begin{proposition}[outer measure of almost disjoint boxes]
	Let $E=\bigcup_n B_n$ be the countable union of almost disjoint boxes (their interiors are all disjoint). Then
	\begin{equation*}
		m^\ast(E)=\sum_n |B_n|.
	\end{equation*}
\end{proposition}

\begin{proposition}[Lebesgue (outer) measure of open sets] Let $U\subset \mathbb{R}^d$ be an open set. Then 
	\begin{equation*}
		m^\ast(U)=\sup_{E\subset U}m(E)
	\end{equation*}
	where $E$ is an elementary set (finite union of boxes).
\end{proposition}

In other terminology, this is saying that the Lebesgue outer measure of open sets is equal to the Jordan inner measure of open sets.

\begin{proposition}[Lebesgue outer measure of an arbitrary set]
	Let $A\subset \mathbb{R}^d$. Then 
	\begin{equation*}
		m^\ast(A)=\inf_{U\supset A} m^\ast(U),
	\end{equation*}
	where the $U$ are open.
\end{proposition}

\subsection{Lebesgue measurability}

\begin{proposition}[quick tests for Lebesgue measurability] The following are Lebesgue measurable: \hfill
	\begin{enumerate}
		\item open and closed sets 
		\item sets with Lebesgue outer measure 0
		\item the empty set
		\item the complement of a Lebesgue measurable set
		\item countable unions and intersections of Lebesgue measurable sets
	\end{enumerate}
\end{proposition}

The above immediately implies Lebesgue measurable sets form a $\sigma$-algebra.

\begin{proposition} Let $E\subset\mathbb{R}^d$, and $\epsilon>0$ arbitrary. The following are equivalent: \hfill
	\begin{enumerate}
		\item $E$ is Lebesgue measurable.
		\item (outer approximation by open) One can contain $E$ in an open set $U$ with $m^\ast(U\setminus E)\leq \epsilon$.
		\item (almost open) One can find an open set $U$ such that $m^\ast(U\ominus E)\leq \epsilon$.
		\item (inner approximation by closed) One can find a closed $F\subset E$ with $m^\ast(E\setminus F)\leq \epsilon$.
		\item (almost closed) One can find a closed set $F$ such that $m^\ast(F\ominus E)\leq \epsilon$.
		\item (almost measurable) One can find a Lebesgue measureable set $E_\epsilon$ such that $m^\ast(E_\epsilon\ominus E)\leq \epsilon$.
	\end{enumerate}
\end{proposition}

In the above, $\ominus$ denotes the symmetric difference of sets. It is like an XOR.

\begin{proposition}[axioms of measures] \hfill
	\begin{enumerate}
		\item $m(\emptyset)=0$
		\item (countable additivity) If $(E_n)$ is a countable sequence of disjoint, Lebesgue measurable set, then $m(\bigcup_n E_n)=\sum_n m(E_n)$. 
	\end{enumerate}
\end{proposition}

\begin{example}[a set which is not Lebesgue measurable]
	TODO
\end{example}

% }}}

\section{Lebesgue integral} % {{{1

\subsection{simple functions} % {{{2

\begin{definition}[simple function]
	A simple function $f:\mathbb{R}^d\to\mathbb{C}$ is a finite linear combination
	\begin{equation*}
		f=\sum_{i=1}^k c_i 1_{E_i}
	\end{equation*}
	where $c_i\in\mathbb{C}$, $E_i\subset\mathbb{R}^d$ are Lebesgue measurable, and $1$ is the indicator function. If the $c_i\in[0,+\infty]$ then $f$ is called unsigned.
\end{definition}

\begin{remark}[]
	The space of simple functions, $\text{Simp}(\mathbb{R}^d)$, is a commutative $\ast$-algebra under pointwise multiplication and complex conjugation.
\end{remark}

\begin{definition}[integral of unsigned simple functions]
	If $f=\sum_{i=1}^k c_i1_{E_i}$ is an unsigned simple function, then its integral is defined as
	\begin{equation*}
		\int f(x)\ dx\coloneqq \sum_{i=1}^k c_im(E_i).
	\end{equation*}
	Clearly this integral takes values in $[0,\infty]$.
\end{definition}

\begin{remark}[]
	One would need to check this is well defined under different representations of the function.
\end{remark}

\begin{proposition}[properties of simple unsigned integral]
	Let $f,g$ be simple unsigned functions. \hfill
	\begin{enumerate}
		\item (unsigned linearity)
		\item (finiteness) $\int f(x)\ dx<\infty$ if and only if $f$ is finite a.e., and its support has finite measure.
		\item (vanishing) $\int f(x)\ dx=0$ if and only if $f=0$ a.e.
		\item (equivalence) If $f=g$ a.e., then $\int f(x)\ dx=\int g(x)\ dx$. 
		\item (monotonicity) If $f\leq g$ a.e., then $\int f(x)\ dx\leq \int g(x)\ dx$. 
		\item (compatibility with Lebesgue measure) For any Lebesgue measurable $E$, $\int 1_E(x)\ dx=m(E)$. 
	\end{enumerate}
\end{proposition}

% simple functions }}}2

\subsection{measurable functions} % {{{2

\subsubsection{unsigned} % {{{3

\begin{definition}[unsigned measurability]
	Let $f$ be an unsigned function. The following are equivalent: \hfill
	\begin{enumerate}
		\item $f$ is unsigned Lebesgue measurable.
		\item $f$ is the pointwise limit of unsigned simple functions. 
		\item $f$ is the pointwise a.e. limit of unsigned simple functions.
		\item $f$ is the supremum of an increasing sequence $0\leq f_1\leq f_2,\dots$ of unsigned simple functions, each of which are bounded with finite measure support.
		\item For every $\lambda\in[0,\infty]$, the set $\{x\ : \ f(x)>\lambda\}$ is Lebesgue measurable.
		\item For every $\lambda\in[0,\infty]$, the set $\{x\ : \ f(x)\geq\lambda\}$ is Lebesgue measurable.
		\item For every $\lambda\in[0,\infty]$, the set $\{x\ : \ f(x)<\lambda\}$ is Lebesgue measurable.
		\item For every $\lambda\in[0,\infty]$, the set $\{x\ : \ f(x)\leq\lambda\}$ is Lebesgue measurable.
		\item For every interval $I\subset[0,\infty)$, the set $f^{-1}(I)$ is Lebesgue measurable.
		\item For every (relatively) open set $U\subset[0,\infty)$, the set $f^{-1}(U)$ is Lebesgue measurable.
		\item For every (relatively) closed set $K\subset[0,\infty)$, the set $f^{-1}(K)$ is Lebesgue measurable.
	\end{enumerate}
\end{definition}

Recall that, for $A\subset X$, a set $U_A\subset A$ is relatively open (in $A$) if there exists an open set $U\subset X$ such that $U\cap A=U_A$. 

\begin{proposition}[common classes of unsigned measurable functions] \hfill
	\begin{enumerate}
		\item every unsigned continuous function
		\item every unsigned simple function 
		\item the supremum, infimum, limit superior, or limit inferior of unsigned measurable functions
		\item an unsigned function equal a.e. to an unsigned measurable function 
		\item pointwise a.e. limit of unsigned measurable functions, provided the limit is unsigned 
		\item postcomposition of an unsigned measurable function with an unsigned continuous function
		\item sum and product of unsigned measurable functions
	\end{enumerate}
\end{proposition}

% unsigned }}}3

\subsubsection{complex} % {{{3

\begin{definition}[complex measurability]
	Let $f:\mathbb{R}^d\to\mathbb{C}$ be a.e. defined. The following are equivalent: \hfill
	\begin{enumerate}
		\item $f$ is measurable.
		\item $f$ is the pointwise, a.e. limit of simple functions
		\item The (magnitudes of the) positive and negative parts of the real and complex parts are unsigned measurable functions.
		\item $f^{-1}(U)$ is Lebesgue measurable for every open set $U\subset\mathbb{C}$.
		\item $f^{-1}(K)$ is Lebesgue measurable for every closed set $K\subset\mathbb{C}$.
	\end{enumerate}
\end{definition}

\begin{proposition}[common classes of complex measurable functions] \hfill
	\begin{enumerate}
		\item continuous functions 
		\item simple functions 
		\item a function equal a.e. to a measurable function
		\item the pointwise, a.e. limit of measurable functions 
		\item the postcomposition of a measurable function by a continuous function 
		\item the sum and product of measurable functions
	\end{enumerate}
\end{proposition}

% complex }}}3

% measurable functions }}}2

\subsection{unsigned integrals} % {{{2 

\begin{definition}[lower, upper unsigned Lebesgue integrals]
	Let $f$ be unsigned (not necessarily measurable). The lower integral is 
	\begin{equation*}
		\underline{\int_{\mathbb{R}^d}}f(x)\ dx\coloneqq\sup_{0\leq g\leq f}\int_{\mathbb{R}^d}g(x)\ dx
	\end{equation*}
	and the upper integral is 
	\begin{equation*}
		\overline{\int_{\mathbb{R}^d}}f(x) \ dx\coloneqq\inf_{h\geq f}\int_{\mathbb{R}^d}h(x)\ dx
	\end{equation*}
	where the $g$ and $h$ are simple.
\end{definition}

\begin{proposition}[properties of lower/upper integral]
	Let $f,g$ be unsigned. \hfill
	\begin{enumerate}
		\item (compatibility with simple integral) The lower, upper, and simple integrals of a simple function all agree.
		\item (monotonicity)
		\item (homogeneuity)
		\item (equivalence) If $f,g$ agree a.e., then their upper/lower integrals are the same.
		\item (super/sub-additivity) $\underline{\int}f(x)+g(x)\ dx\geq\underline{\int}f(x)\ dx+\underline{\int}g(x)\ dx$ and the direction of the inequality flips for upper integrals.
		\item (divisibility) For any measurable set $E$, one has $\underline{\int}f(x)\ dx=\underline{\int}f(x)1_{E}(x)\ dx+\underline{\int}f(x)1_{\mathbb{R}^d\setminus E}(x)\ dx$.
		\item (horizontal truncation) $\underline{\int}\min(f(x),n)\ dx\to\underline{\int}f(x)\ dx$ as $n\to\infty$.
		\item (vertical truncation) $\underline{\int}f(x)1_{|x|\leq n}\ dx\to\underline{\int}f(x)\ dx$ as $n\to\infty$.
		\item (reflection) If $f+g$ is a simple function that is bounded with finite measure support (= absolutely integrable), then $\int f(x)+g(x)\ dx=\underline{\int}f(x)\ dx + \overline{\int}g(x)\ dx$.
	\end{enumerate}
\end{proposition}

\begin{definition}[unsigned Lebesgue integral]
	If $f$ is unsigned and measurable, define the unsigned Lebesgue integral $\int f(x)\ dx$ to be equal to the lower unsigned Lebesgue integral.
\end{definition}

% unsigned integrals }}}2

\subsection{Lebesgue integral} % {{{2 

\begin{definition}[Lebesgue integral]
	Let $f$ be a.e. defined, measurable, and absolutely integrable. If $f$ is real-valued, define the Lebesgue integral to be 
	\begin{equation*}
		\int f(x)\ dx\coloneqq \int \max(f(x),0)\ dx - \int \max(-f(x),0)\ dx.
	\end{equation*}
	If $f$ is complex valued, then define the Lebesgue integral to be 
	\begin{equation*}
		\int f(x)\ dx\coloneqq \int\text{Re}\ f(x)\ dx+i\int\text{Im}\ f(x)\ dx.
	\end{equation*}
\end{definition}

\begin{proposition}[integration is $\ast$-linear]
	TODO
\end{proposition}

\begin{proposition}[triangle inequality]
	TODO
\end{proposition}

% absolute integrability }}}

% Lebesgue integral }}}1

\section{Littlewood's three principles} % {{{1 

\begin{enumerate}
	\item Every measurable set is nearly a finite sum of intervals.
	\item Every absolutely integrable function is nearly continuous.
	\item Every pointwise convergent sequenece of functions is nearly uniformly convergent.
\end{enumerate}

\subsection{first principle}

\subsection{second principle} % {{{2 

The following, particularly (3), is a stronger version of Littlewood's second principle.

\begin{theorem}[approximation of $L^1$ functions] 
	The following classes of functions are dense in $L^1(\mathbb{R}^d)$: \hfill
	\begin{enumerate}
		\item absolutely integrable 
		\item step
		\item continuous, compactly supported
	\end{enumerate}
\end{theorem}

\begin{theorem}[Lusin's theorem]
	Let $f$ be absolutely integrable. For any $\epsilon>0$, there exists a Lebesgue measurable set $E$ subh that $m(E)<\epsilon$ and the restriction of $f$ outside of $E$ is continuous.
\end{theorem}

% second principle }}}2

\subsection{third principle} % {{{2 

Fix domain $\mathbb{R}^d$ and codomain $\mathbb{C}$. 

\begin{definition}[locally uniform convergence]
	We say $f_n\to f$ locally uniformly if, on every bounded $E\subset\mathbb{R}^D$, $f_n\to f$ uniformly.
\end{definition}

\begin{theorem}[Egorov's theorem]
	Let $f_n\to f$ pointwise a.e, and suppose all these functions are measurable. For any $\epsilon>0$, there exists a Lebesgue measurable set $A$ such that $m(A)<\epsilon$ and $f_n\to f$ locally uninformly outside of $A$. 
\end{theorem}

% third principle }}}2

\subsection{related principles} % {{{2 

\begin{enumerate}
	\item (absolutely integrable functions almost have bounded support) Let $f$ be absolutely integrable. For any $\epsilon>0$, there exists a ball $B(0,R)$ outside of which $\|f\|_{L^1}<\epsilon$.
	\item (measurable functions are almost locally bounded) Let $f$ be measurable. For any $\epsilon>0$, there exists a measurable set $E$ such that $m(E)<\epsilon$ and $f$ is locally bounded outside of it, i.e. for every $R>0$ there exists $M<\infty$ such that $|f(x)|\leq M$ for all $x\in B(0,R)\setminus E$.
\end{enumerate}

% related principles }}}2

% Littlewood's three principles }}}1

\section{abstract measures} % {{{1 

\subsection{$\sigma$-algebras} % {{{2 

\begin{definition}[$\sigma$-algebras] 
	Let $X$ be a set. A $\sigma$-algebra on $X$ is a collection $\mathcal{B}$ of $X$ which obeys the following properties:
	\begin{enumerate}
		\item $\emptyset\in\mathcal{B}$
		\item (closed under complements)
		\item (closed under countable union)
	\end{enumerate}
\end{definition}

It follows that $\sigma$-algebras are closed under countable intersection as well.

\begin{definition}[measurable space]
	A pair $(X,\mathcal{B}$) is called a measurable space.
\end{definition}

\begin{definition}[generation of $\sigma$-algebras] 
	Let $\mathcal{F}$ be any family of sets in $X$. Define the $\sigma$-algebra generated by $\mathcal{F}$, $\langle \mathcal{F}\rangle$, to be the coarsest $\sigma$-algebra containing $\mathcal{F}$.
\end{definition}

\begin{definition}[Borel $\sigma$-algebra]
	For a topological space $X$, the Borel $\sigma$-algebra $\mathcal{B}[X]$ of $X$ is the $\sigma$-algebra generated by the open subsets of $X$.
\end{definition}

\begin{remark}[Lebesgue and Borel $\sigma$-algebras are different]
	TODO
\end{remark}

% $\sigma$-algebras }}}2

\subsection{measure} % {{{2 

\begin{definition}[measure]
	Let $(X,\mathcal{B})$ be a measurable space. An (unsigned) measure is a map $\mu:\mathcal{B}\to[0,+\infty]$ such that 
	\begin{enumerate}
		\item $\mu(\emptyset)=0$
		\item (countable additivity) If $(E_i)$ are a disjoint and measurable, then $\mu(\bigcup_i E_i) = \sum_i \mu(E_i)$.
	\end{enumerate}
\end{definition}

\begin{definition}[measure space]
	A triplet $(X,\mathcal{B},\mu)$ is called a measure space.
\end{definition}

\begin{proposition}[]
	Let $(X,\mathcal{B},\mu)$ be a measure space. Then \hfill
	\begin{enumerate}
		\item (countable subadditivity)
		\item (upwards monotone convergence) If $E_1\subset E_2\subset\cdots$ are measurable, then 
			\begin{equation*}
				\mu(\bigcup_n E_n)=\lim_{n\to\infty}\mu(E_n)=\sup_n\mu(E_n).
			\end{equation*}
		\item (downwards monotone convergence) If $E_1\supset E_2\supset\cdots$ are measurable, and $\mu(E_n)<\infty$ for at least one $n$, then 
			\begin{equation*}
				\mu(\bigcap_n E_n)=\lim_{n\to\infty}\mu(E_n)=\inf_n\mu(E_n).
			\end{equation*}
	\end{enumerate}
\end{proposition}

\begin{definition}[completeness]
	A measure space is complete if every sub-null set (complement of a null set) is a null set (a set of measure zero).
\end{definition}

\begin{remark}[]
	On $\mathbb{R}^d$, the Lebesgue measure space is complete but the Borel measure space is not.
\end{remark}


% measure }}}2

\subsection{integration} % {{{2 

\begin{definition}[measurable function]
	Let $(X,\mathcal{B})$ be a measurable space. A function $f:X\to[0,+\infty]$ or $f:X\to\mathbb{C}$ is measurable if $f^{-1}(U)$ for every open subset $U$ of $[0,+\infty]$ or $\mathbb{C}$.
\end{definition}

\begin{definition}[integral of simple functions]
	An (unsigned) simple function $f:X\to[0,+\infty]$ on a measurable space is a measurable function taking on finitely many values $a_1,\dots,a_k$. Its integral is defined as 
	\begin{equation*}
		\int_X f\ d\mu\coloneqq\sum_{j=1}^k a_j\mu(f^{-1}(\{a_j\})).
	\end{equation*}
\end{definition}

\begin{definition}[unsigned integral]
	Let $f$ be unsigned and measurable. Define its integral to be 
	\begin{equation*}
		\int_X f\ d\mu\coloneqq \sup_{0\leq g\leq f} \int_X g\ d\mu,
	\end{equation*}
	where the $g$ are simple.
\end{definition}

\begin{proposition}[properties of unsigned integral]
	Let $f,g$ be unsigned and measurable. \hfill
	\begin{enumerate}
		\item (a.e. equivalence)
		\item (monotonicity)
		\item (homogeneity)
		\item (supperadditivity)
		\item (compatibility with simple integral)
		\item (Markov's inequality)
		\item (finiteness)
		\item (vanishing)
		\item (vertical truncation) We have 
			\begin{equation*}
				\lim_{n\to\infty}\int_X\min(f,n)\ d\mu=\int_X f\ d\mu.
			\end{equation*}
		\item (horizontal truncation) If $(E_i)$ is an non-decreasing sequence of measurable sets, 
			\begin{equation*}
				\lim_{n\to\infty}\int_X f\ 1_{E_n}\ d\mu=\int_X f 1_{\bigcup_n E_n}\ d\mu.
			\end{equation*}
		\item (restriction)
	\end{enumerate}
\end{proposition}

\begin{proof}
	\hfill
	\newline
	\textbf{Vertical truncation}

	This is a consequence of upwards monotone convergence for measurable sets (TODO ref). 
	\begin{align*}
		\lim_{n\to\infty}\int_X \min(f,n)\ d\mu =& \lim_{n\to\infty}\int_X f 1_{\{x\mid f(x)\leq n\}}\ d\mu \\
		=& \lim_{n\to\infty}\sup_{0\leq g\leq f}\int g 1_{\{x\mid f(x)\leq n\}}\ d\mu \\
		\intertext{(where the $g$ are simple)}
		=& \lim_{n\to\infty}\sum_{0\leq g\leq f}\sum_{i=1}^k c_i\mu(g^{-1}(c_i)\cap \{x\mid f(x)\leq n\}) \\
		\intertext{(where the $c_i$ are the finitely many values of $g$)}
		=& \sup_{0\leq g\leq f}\sum_{i=1}^k c_i \mu(g^{-1}(c_i)) \\
		\intertext{(since the sets $g^{-1}(c_i)\cap \{x\mid f(x)\leq n\}$ are monotonically increasing as $n\to\infty$)}
		=& \int_X f\ d\mu.
	\end{align*}

	\textbf{Horizontal truncation}

	This is another consequence of the upwards monotone convergence for measurable sets, and the proof is analagous to the one above.
\end{proof}

\begin{theorem}[linearity of unsigned integral]
	TODO
\end{theorem}

\begin{proposition}[linearity in $\mu$]
	TODO
\end{proposition}

\begin{proposition}[change of variables formula]
	TODO
\end{proposition}

\begin{definition}[general integral]
	TODO
\end{definition}

% integration }}}

\subsection{signed measures} % {{{2 

\begin{definition}[signed measure]
	A signed measure is a map $\mu:\mathcal{X}\to[-\infty,+\infty]$ such that 
	\begin{enumerate}
		\item $\mu(\emptyset)=0$.
		\item $\mu$ can take either $+\infty$ or $-\infty$ but not both.
		\item If $(E_i)$ are disjoint, then $\sum_n\mu(E_n)\to\mu(\bigcup_n E_n)$. If the latter is finite, the sum is absolutely convergent. 
	\end{enumerate}
\end{definition}

\begin{theorem}[Hahn decomposition]
	Let $\mu$ be a signed measure. Then we can find (explicitly) a partition $X=X_+\cup X_-$ such that $\mu_+\coloneqq\mu|_{X_+}\geq 0$ and $\mu_- \coloneqq -\mu|_{X_-} \geq 0$. Moreover, these sets are unique modulo null\footnote{A set $E$ is called null for a signed measure $\mu$ if $\mu|_E=0$. Note this implies $\mu(E)=0$, but not conversely since $E$ may contain subsets of non-zero measure even if it has signed measure zero.} sets.
\end{theorem}

\begin{proof}
	Without loss of generality, we may assume $\mu$ avoids $+\infty$, by replacing $\mu$ with $-\mu$ if necessary.

	We will construct $X_+$ to be the totally positive\footnote{A set $E$ is \emph{totally positive} if $\mu|_E\geq 0$, i.e. $\mu(E')\geq 0$ for any $E'\subset E$ measurable. We can analagously define \emph{totally negative} sets.} set of maximal measure. Define $m_+$ to be the supremum of $\mu(E)$ where $E$ ranges over all totally positive sets. Then we can find a sequence of totally positive sets $(E_n)$ such that $\mu(E_n)\to m_+$. Let $X_+=\bigcup_n E_n$. Then $\mu(X_+)=\mu(\bigcup_n E_n)=\lim_{n\to\infty}\mu(E_n)=m_+$. Since $\mu$ avoids $+\infty$ (by our without loss of generality assumption), $m_+$ is finite.

	Let $X_-\coloneqq X\setminus X_+$. It suffices to show that $X_-$ is totally negative. Suppose otherwise. Our idea will be to construct a subset $E\subset X_-$ that belongs to $X_+$. Since we have supposed that $X_-$ is not totally negative, there exists a subset $E_1\subset X_-$ of strictly positive measure. If $E_1$ were totally positive, then $X_+\cup E_1$ would be a positive set having measure strictly greater than $m_+$ (since $E_1\not\subset X_+$), which is a contradiction. Thus $E_1$ must contain a subset of strictly negative measure, whose complement must have strictly larger measure than $E_1$. 

	So pick $E_2$ such that $\mu(E_2)\geq \mu(E_1)+1/n_1$, where $n_1$ is the smallest integer for which such an $E_2$ exists. Note that this is well-defined; at the end of the previous paragraph we have asserted that at least one possible $n_1$ exists, and from there there are only finitely many more choices to pick from. If $E_2$ is totally positive, we again have a contradiction and so we can find a subset $E_2$ with $\mu(E_3)\geq\mu(E_2)+1/n_2$ where $n_2$ is the smallest integer for which such an $E_3$ exists. In this way, we obtain a sequence $E_1\supset E_2\supset\cdots$ in $X_-$ of increasing positive measure, specifically $\mu(E_{j+1})\geq\mu(E_j)+1/n_j$.

	Define $E\coloneqq\bigcap_j E_j$. It is measurable, being the countable union of measurable sets, and $\mu(E)=\lim_{j\to\infty}\mu(E_j)>0$ by the TODO REF (does this apply to signed measures?). By assumption $\mu(E)<\infty$, hence it must be the case that $n_j\to\infty$. We claim $E$ does not contain any subsets of strictly larger measure, hence a contradiction by previous remarks. Suppose it contained such a set $D$. But then $D\subset E_j$ for some $j$, hence $\mu(D)\geq\mu(E_j)+1/n_j$ for some $n_j$, hence $E\subset D$ and so $E=D$, which is a contradiction.

	Now we will show uniqueness up to null sets. Let $X=Y_+ \cup Y_-$ be another decomposition. By construction, $X_+$ is the maximal totally positive subset of $X$, hence $Y_+\subset X_+$. Let $C=X_+ \setminus Y_+$. Then $C\subset Y_-$. If $\mu(C)\neq 0$, then $\mu(C)>0$ which contradicts this fact. 
\end{proof}

\begin{corollary}[Jordan decomposition]
	Every signed measure $\mu$ can be uniquely decomposed as $\mu=\mu_+-\mu_-$, where $\mu_+$ and $\mu_-$ are mutually singular\footnote{Two signed measures $\mu$ and $\nu$ are \emph{mutually singular}, $\mu\perp\nu$, if they can be supported on disjoint sets. A signed measure $\mu$ is \emph{supported} on $E$ if the complement of $E$ is null.} unsigned measures, called the positive and negative parts/variations of $\mu$.
\end{corollary}

\begin{definition}[finite signed measure]
	A signed measure $\mu$ is finite if any of the following are equivalent hold:
	\begin{enumerate}
		\item $\mu(E)$ is finite for every $E$.
		\item $|\mu|$ is a finite unsigned measure.
		\item $\mu_+$ and $\mu_-$ are finite unsigned measures.
	\end{enumerate}
\end{definition}

\begin{definition}[$\sigma$-finite signed measure]
	A signed measure $\mu$ is $\sigma$-finite if $|\mu|$ is a $\sigma$-finite unsigned measure.
\end{definition}

\begin{theorem}[Lebesgue-Radon-Nikodym]
	Let $m$ be an unsigned $\sigma$-finite measure, and $\mu$ a signed $\sigma$-finite measure. There exists a unique decomposition 
	\begin{equation*}
		\mu=m_f+\mu_s
	\end{equation*}
	where $f\in L^1(X,dm)$ and $\mu_s\perp m$. Moreover, if $\mu$ is unsigned then $f$ and $\mu_s$ are also.
\end{theorem}

% signed measures }}}2

% abstract measures }}}1

\section{convergence theorems} % {{{1 

\subsection{summary} % {{{2 

One of our primary motivations in developing measure theory has been to obtain an integral that obtains well ``at the limit'', in various senses. Perhaps most directly, we ask if 
\begin{equation*}
	\lim_{n\to\infty}\int_X f_n\ d\mu \overset{?}{=} \int_X\lim_{n\to\infty}f_n\ d\mu.
\end{equation*}

There are two major ways to ensure compatibility of the integral with limits: monotonicity and domination.

% summary }}}2

\subsection{counterexamples} % {{{2 

% counterexamples }}}2

\subsection{monotone convergence theorem} % {{{2 

\begin{theorem}[monotone covergence theorem for measurable sets]
	\hfill
	\begin{enumerate}
		\item (upward) Let $(E_i)$ be a countable, non-decreasing sequence of measurable sets. Then $m(\bigcup_n E_n)=\lim_{n\to\infty} m(E_n)$.
		\item (downward) Let $(E_i)$ b e acountable, non-increasing sequence of measurable sets. If at least one of the $m(E_n)$ is finite, then $m(\bigcap_n E_n)=\lim_{n\to\infty}m(E_n)$. 
	\end{enumerate}
\end{theorem}

\begin{remark}[]
	The basic reason (1) is true is that montonicity allows us to ``disjointify'' the $E_n$, so that countable additivity becomes available to us.
\end{remark}

\begin{proof}
	First we will address (1). Under the assumptions, we can rewrite 
	\begin{equation*}
		\bigcup_{n=1}^N E_n = \bigcup_{n=1}^N \left(E_n\setminus\bigcup_{m=1}^{n-1}E_n\right).
	\end{equation*}
	Therefore
	\begin{equation*}
		\bigcup_n E_n = \bigcup_n\left(E_n\setminus\bigcup_{m=1}^{n-1}E_n\right),
	\end{equation*}
	where we observe the countable union on the right is now of disjoint sets. We use the convention that whenever the upper index in a finite union is less than the lower index, the result is the empty set. Note that it may be possible for many sets in the right-hand-side union to be empty, but this does not affect disjointedness. By applying countable additivity to the right, we see that $m(\bigcup_n E_n)=\sum_n m(E_n)$. 

	Now we will address (2). TODO.
\end{proof}

% monotone convergence theorem {{{3
\begin{theorem}[monotone convergence theorem]
	Let $0\leq f_1\leq f_2\leq\cdots$ be a monotone, non-decreasing sequence of unsigned measurable functions on $X$. Then 
	\begin{equation*}
		\lim_{n\to\infty}\int_X f_n\ d\mu = \int_{X}\lim_{n\to\infty}f_n\ d\mu.
	\end{equation*}
\end{theorem}

\begin{proof}
	Half of the equality is apparant:
	\begin{equation*}
		\lim_{n\to\infty}\int_X f_n\ d\mu\leq\int_X f\ d\mu.
	\end{equation*}

	In the other direction, we know that 
	\begin{equation*}
		\int_X f\ d\mu=\sup_{0\leq g\leq f}\int_X g\ d\mu
	\end{equation*}
	where the $g$ are simple, and so it suffices to show that 
	\begin{equation*}
		\int_X g\ d\mu\leq\lim_{n\to\infty}\int_X f_n\ d\mu.
	\end{equation*}

	The simple function $g$ is bounded everywhere by $f$, but not bounded everywhere (it may even be bounded nowhere) by $f_n$ for some $n$. The idea is that the measure of the places where $g$ is bounded by $f_n$ approaches the measure of the entire space as we take limits, since the $f_n\to f$ and $g\leq f$ everywhere. Monotonicity makes this possible by the previous theorem (TODO ref).

	There is a subtle technical detail we need to be weary of, however. If $g$ is exactly equal to $f$ somewhere, then it may never become bounded by the $f_n$, since the values of $f$ may never be attained by any $f_n$. This is a stiuation where we would like to give ourselves an ``epsilon of room'', to consider not the actual values $c_i$ of $g$, but a scaled down version $(1-\epsilon)c_i$ for $0<\epsilon<1$. This puts separation between $g$ and $f$, but only if $c \neq +\infty$.  

	By vertical truncation (TODO ref), this is not an issue. We have that 
	\begin{equation*}
		\lim_{m\to\infty}\int_X \min(g,n)\ d\mu = \int_X g\ d\mu,
	\end{equation*}
	so to show that
	\begin{equation*}
		\int_X g\ d\mu\leq\lim_{n\to\infty}\int_X f_n\ d\mu
	\end{equation*}
	it suffices that show 
	\begin{equation*}
		\int_X \min(g,n)\ d\mu\leq\lim_{n\to\infty}\int_X f_n\ d\mu
	\end{equation*}
	for all $n$. Of course, $\min(g,n)$ only takes on finite values.

	We proceed as follows. Let $\min(g,n)$ take the finite (unsigned) values $(c_i)_1^k$ on the (disjoint, measurable) sets $(A_i)_1^k$. Pick $\epsilon\in(0,1)$. Then, 
	\begin{equation*}
		\sup_n f_n(x)=f(x)>(1-\epsilon)c_i
	\end{equation*}
	for all $x\in A_i$. Now define the sets 
	\begin{equation*}
		A_{i,n}\coloneqq\{x\in A_i: f_n(x)>(1-\epsilon)c_i\}.
	\end{equation*}
	The $A_{i,n}$ are measurable by (TODO ref). We claim $\bigcup_n A_{i,n}=A_i$. Indeed, for any $x\in A_i$, $\alpha\coloneqq f(x)-(1-\epsilon)c_i>0$, so there exists (in particular) $n>0$ such that $0<f(x)-f_n(x)<\alpha$. Then $f_n(x)>(1-\epsilon)c_i$, so $x\in A_{i,n}$. 

	So we can apply upwards monotonicity of measure (TODO ref) to get that 
	\begin{equation*}
		\lim_{n\to\infty}\mu(A_{i,n})=\mu(A_i).
	\end{equation*}
	By construction of the $A_{i,n}$,
	\begin{equation*}
		f_n\geq\sum_{i=1}^k(1-\epsilon)c_i1_{A_{i,n}}
	\end{equation*}
	for any $n$, we we can integrate to get
	\begin{equation*}
		\int_X f_n\ d\mu\geq (1-\epsilon)\sum_{i=1}^k c_i\mu(A_{i,n}).
	\end{equation*}
	Taking limits,
	\begin{equation*}
		\lim_{n\to\infty}\int_X f_n\ d\mu\geq (1-\epsilon)\sum_{i=1}^kc_i\mu(A_i)=(1-\epsilon)\int_X\min(g,n)\ d\mu,
	\end{equation*}
	and by sending $\epsilon\to 0$ we get what we were looking for.
\end{proof}
% monotone convergence theorem }}}3

\begin{corollary}[Tonelli's theorem for sums and integrals]
	Let $(f_n)$ be a sequence of unsigned measurable functions. Then 
	\begin{equation*}
		\int_X\sum_{n=1}^\infty f_n\ d\mu=\sum_{n=1}^\infty\int_X f_n\ d\mu.
	\end{equation*}
\end{corollary}

\begin{proof}
	The idea is that the partial sums are a monotone sequence, and we can apply the monotone convergence theorem.
\end{proof}

% Fatou's lemma {{{3
\begin{corollary}[Fatou's lemma]
	Let $(f_n)$ be a sequence of unsigned measurable functions. Then 
	\begin{equation*}
		\int_X\liminf_{n\to\infty}f_n\ d\mu\leq\liminf_{n\to\infty}\int_X f_n\ d\mu.
	\end{equation*}
\end{corollary}

\begin{proof}
	The idea is that $F_N\coloneqq\inf_{n\geq N}f_n$ is a montone (non-decreasing) sequence of unsigned measurable functions (TODO ref). So by the montone convergence theorem (TODO ref)
	\begin{equation*}
		\int_X\lim_{N\to\infty}F_n\ d\mu = \lim_{N\to\infty}\int_X F_N\ d\mu.
	\end{equation*}
	Now by construction (and monotonicity, TODO ref) $\int_X F_N\ d\mu\leq\int_X f_n\ d\mu$ for all $n\geq N$. So 
	\begin{equation*}
		\int_X F_N\ d\mu\leq\inf_{n\geq N}\int_X f_n\ d\mu.
	\end{equation*}
	Taking limits, 
	\begin{equation*}
		\int_X \liminf_{n\to\infty}f_n\ d\mu=\int_X \lim_{N\to\infty}F_n\ d\mu=\lim_{N\to\infty}\int_X F_N\ d\mu\leq\lim_{n\to\infty}\inf_{n\geq N}\int_X f_n\ d\mu
	\end{equation*}
\end{proof}
% Fatou's lemma }}}3

% monotone convergence theorem }}}2

\subsection{dominated convergence theorem} % {{{2 

\begin{theorem}[dominated convergence theorem]
\label{thm_dct}
	Let $(f_n)$ be a sequence of measurable functions $X\to\mathbb{C}$ that converge pointwise a.e. to a measurable limit $f$. Suppose there exists an unsigned, absolutely integrable function $G$ that bounds each $|f_n|$ pointwise a.e. Then 
	\begin{equation*}
		\lim_{n\to\infty}\int_X f_n\ d\mu = \int_X f\ d\mu.
	\end{equation*}
\end{theorem}

\begin{proof}
	We may assume without loss of generality that the convergence and boundedness in the hypothesis is everywhere rather than a.e., and that the $f_n$ and $f$ are real. Then $-G\leq f_n\leq G$ and likewise for the limit $f$. Thus the functions $f_n+G$ are unsigned, and we can apply Fatou's lemma (TODO ref) to obtain 
	\begin{equation*}
		\int_X f+G\ d\mu=\int_X\lim_{n\to\infty}f_n+G\ d\mu=\int_X\liminf_{n\to\infty}f_n+G\ d\mu\leq\liminf_{n\to\infty}\int_X f_n+G\ d\mu.
	\end{equation*}
	The key thing to note is that $\int_X G\ d\mu$ is finite by assumption, so we can subtract it to get
	\begin{equation*}
		\int_X f\ d\mu \leq \liminf_{n\to\infty}\int_X f_n\ d\mu. 
	\end{equation*}
	Conversely, the functions $G-f_n$ are also unsigned, and we proceed analagously. First, 
	\begin{equation*}
		\int_X G-f\ d\mu\leq\liminf_{n\to\infty}\int_X G-f_n\ d\mu,
	\end{equation*}
	and subtracting $\int_X G\ d\mu$ gives us 
	\begin{equation*}
		\int_X -f\ d\mu\leq\liminf_{n\to\infty} \int_X -f_n\ d\mu.
	\end{equation*}
	Negating both sides,
	\begin{equation*}
		\limsup_{n\to\infty}\int_X f_n \ d\mu\leq\int_X f\ d\mu.
	\end{equation*}
	In conclusion,
	\begin{equation*}
		\limsup_{n\to\infty}\int_X f_n\ d\mu\leq\int_X f\ d\mu\leq\liminf_{n\to\infty}\int_X f_n\ d\mu,
	\end{equation*}
	and since $\liminf \leq \limsup$, we must have equalities in the equation above, and since $\liminf = \limsup$ the limit must also exist and equal that same number.
\end{proof}

% dominated convergence theorem }}}2

% convergence theorems }}}1

\section{constructing measures} % {{{1 

\subsection{basic operations}

\begin{proposition}
\label{prop_basic_operations_on_measures}
	Let $(X, \mathcal{A})$ be a measure space, and let $\mu$ be an unsigned measure on it. Let $A,B\in\mathcal{A}$.
	\begin{enumerate}
		\item The function $\nu(A) = c\mu(A)$ is an unsigned measure for $c\geq 0$.
		\item The function $\nu(A) = \mu(A \cap B)$ is an unsigned measure, and 
			\begin{equation*}
				\int_X f d\nu = \int_B f d\mu.
			\end{equation*}
			for every nonnegative function $f$.
	\end{enumerate}
\end{proposition}
\begin{proof}
	\hfill
	\begin{enumerate} 
		\item Needed?
		\item Consider 
			\begin{align*}
				\int_X f d\nu 
				=& \sup_{0\leq g \leq f \text{ simple}} \sum_{i=1}^k a_k \mu (A_k \cap B) \\
				=& \sup_{0\leq g \leq f} \int g\cdot 1_B\ d\mu \\
				=& \int_B f\ d\mu.
			\end{align*}
	\end{enumerate}
\end{proof}

\begin{proposition}[]
	Given two unsigned measures $\mu,\nu$, then 
	\begin{enumerate}
		\item $(\mu+\nu)(E)\coloneqq\mu(E)+\nu(E)$
		\item $(c\mu)(E)\coloneqq c(\mu(E))$
	\end{enumerate}
	are unsigned measured, and 
	\begin{enumerate}
		\item $(\mu-\nu)(E)\coloneqq\mu(E)-\nu(E)$
	\end{enumerate}
	is a signed measure provided either $\mu$ and $\nu$ is finite.
\end{proposition}

\begin{remark}[]
	By the Hahn decomposition theorem (Thm TODO ref), every signed measure is a difference as above.
\end{remark}

\begin{proposition}[absolute value]
	The absolute value, or \emph{total variation}, of a signed measure $\mu$ is the unsigned measure $|\mu|\coloneqq \mu_+ + \mu_-$.
\end{proposition}

\begin{proposition}[]
	$|\mu|$ is the minimal unsigned measure such that $- |\mu| \leq \mu \leq |\mu|$.
\end{proposition}

\begin{proposition}[]
	$|\mu|(E)$ is equal to the maximum value of $\sum_n |\mu(E_n)|$, where $(E_n)$ ranges over the partitions of $E$.
\end{proposition}

\subsection{Radon-Nikodym}

\begin{definition}
	Let $m$ be an unsigned measure, and $f$ an unsigned function. Then we can construct a new unsigned measure 
	$m_f$ as follows:
	\begin{equation*}
		m_f(E)\coloneqq\int_X 1_E f\ dm.
	\end{equation*}
	If $f$ is signed, and either $\max(f,0)$ or $\max(-f,0)$ is absolutely integrable, then $m_f$ is a signed measure.
\end{definition}

\begin{remark}[]
	When $f=1_A$, this this is called the \emph{restriction} of $m$ to $A$.
\end{remark}

\begin{proposition}[]
	$m_f$ is an unsigned measure.
\end{proposition}

\begin{proposition}[]
	For any $g:X\to[0,+\infty]$, we have $\int_X g\ dm_f = \int_X gf\ dm$. (We write this relationship as $dm_f=f\ dm$).
\end{proposition}

\begin{proposition}[]
	If $m$ is $\sigma$-finite, $m_f=m_g$ if and only if $f(x)=g(x)$ for $m$-almost every $x$.
\end{proposition}

\begin{definition}[Radon-Nikodym derivative]
	A measure $\mu$ is differentiable with respect to $m$ if $\mu=m_f$ (i.e. $\mu=m_f$) for some $f$. We call this $f$ the \emph{Radon-Nikodym derivative} of $\mu$ with respect to $m$, writing 
	\begin{equation*}
		f=\frac{d\mu}{dm}.
	\end{equation*}
\end{definition}

\begin{remark}[]
	If $m$ is $\sigma$-finite, the Radon-Nikodym derivative is defined up to $m$-a.e. equivalence.
\end{remark}

% constructing measures }}}

\section{product measures} % {{{1

\todo{comment on $\sigma$-finite assumption}

\subsection{product $\sigma$-algebras} % {{{2

Given two measure spaces $(E,\mathcal{A})$ and $(F,\mathcal{B})$, we define the product $\sigma$-algebra to be 
\begin{equation*}
	\mathcal{A}\otimes\mathcal{B}\coloneqq\sigma(A\times B\mid A\in\mathcal{A}, B\in\mathcal{B}),
\end{equation*}
i.e. the $\sigma$-algebra generated by so called ``measurable rectangles''. It is universal in the following sense:

\begin{proposition}[]
	$\mathcal{A}\otimes\mathcal{B}$ is the smallest $\sigma$-algebra such that the projection maps 
	\begin{gather*}
		\pi_1:E\times F\to E, \\
		\pi_2: E\times F\to F
	\end{gather*}
	are measurable.
\end{proposition}
\begin{proof}[Proof idea]
	For $\pi_1$ to be measurable, the sets $(e,F)$ must be measurable, likewise the sets $(E,f)$ if $\pi_2$ is to be measurable. 
\end{proof}

Now, for $C\in\mathcal{A}\otimes\mathcal{B}$, Let
\begin{gather*}
	C_{x\in E}\coloneqq \{y\in F\mid (x,y)\in C\}, \\
	C^{y\in F}\coloneqq \{x\in E\mid (x,y)\in C\}.
\end{gather*}

\begin{proposition}[]
	$C_x\in \mathcal{B}$ and $C^y\in \mathcal{A}$.
\end{proposition}
\begin{proof}[Proof idea]
	This is a consequence of the universal property of $\mathcal{A}\otimes\mathcal{B}$. By construction the proposition holds for when $C$ is a measurable rectangle. One then sees this property is preserved under complements, countable unions, and countable intersections. Thus the set $\mathcal{C}$ of all $C$ such that the proposition holds is a $\sigma$-algebra. But $\mathcal{C}\subset\mathcal{A}\otimes\mathcal{B}$, and $\mathcal{A}\otimes\mathcal{B}$ is minimal, hence $\mathcal{C}=\mathcal{A}\otimes\mathcal{B}$.
\end{proof}

We can build on this definition with the following: given a function
\begin{equation*}
	f:(E\times F,\mathcal{A}\otimes\mathcal{B})\to (G,\mathcal{G}),
\end{equation*}
define the functions
\begin{gather*}
	f_x(y)\coloneqq f(x,y), \\
	f^y(x)\coloneqq f(x,y).
\end{gather*}

\begin{proposition}[]
	$f_x$ is $\mathcal{B}$-measurable, and $f^y$ is $\mathcal{A}$-measurable.
\end{proposition}
\begin{proof}[Proof idea]
	This builds on the previous proposition in the following way: for any measurable subset $D\in\mathcal{G}$, notice that $f_x^{-1}(D)=(f^{-1}(D))_x$.
\end{proof}

% product $\sigma$-algebras }}}2

\subsection{product measures} %{{{2

We now state an assumption we will hold for the rest of this section: the measures we will consider are $\sigma$-finite. So let $(E,\mathcal{A},\mu)$ and $(F,\mathcal{B},\nu)$ be $\sigma$-finite.

\begin{theorem}[]
	There exists a unique measure $(E\times F,\mathcal{A}\otimes\mathcal{B}, \mu\otimes\nu)$ such that 
	\begin{equation*}
		\mu\otimes\nu(A\times B)=\mu(A)\nu(B).
	\end{equation*}
	This measure is $\sigma$-finite, and can be defined in the following equivalent ways:
	\begin{equation*}
		\mu\otimes\nu(C)= \int_E \nu(C_x)\mu(dx) = \int_F \mu(C^y)\nu(dy).
	\end{equation*}
	Implicit in the above statement is that the functions $x\mapsto \nu(C_x)$ and $y\mapsto\mu(C^y)$ are $\mathcal{A}$- and $\mathcal{B}$-measurable, respectively.
\end{theorem}
\begin{proof}[Proof idea]
	\textit{Uniqueness:} Since $\mathcal{A}\otimes\mathcal{B}$ is generated by sets of the form $A\times B$, any two measures which satisfy the property in the theorem will agree on any set in $\mathcal{A}\otimes\mathcal{B}$.

	\textit{Existence:} Let us discuss the first equality, as the second is analagous. We first need to show that $x\mapsto \nu(C_x)$ is $\mathcal{A}$-measurable. To do this, we begin with the case where $\nu$ is finite, and then generalize to the $\sigma$-finite case. So suppose $\nu$ is finite. Again, we utilize the universal property of $\mathcal{A}\otimes\mathcal{B}$: let $\mathcal{C}$ be the class of all sets $C\in\mathcal{A}\otimes\mathcal{B}$ is measurable. One then shows $\mathcal{C}$ contains all measurable rectangles and is a monotone class (TODO REF). Then, by the monotone class theorem, it is a $\sigma$-algebra and by minimality it must be equal to $\mathcal{A}\otimes\mathcal{B}$.

	To generalize to the $\sigma$-finite case, determine (the existence of) an increasing sequence of finite measure sets $B_n$ whose union is $F$. Restricting $\nu$ to any $B_n$ gives a finite measure, to which the previous paragraph applies. Then take $\nu=\lim_n \nu_n$ in an appropriate sense.

	Now one carries out an explicit calculation to show that the first equality in the theorem is a measure and satisfies the characteristic property.
\end{proof}

\begin{remark}[]
	One generalizes this to the product of an arbitrary finite number of $\sigma$-finite measures:
	\begin{equation*}
		\mu_1\otimes\cdots\otimes \mu_n = \mu_1 \otimes (\mu_2 \otimes \cdots (\mu_{n-1}\otimes \mu_n)).
	\end{equation*}
	The order of the parenthesis is unimportant since the measure is characterized by its value on measurable rectangles,
	and ordinary multiplication is associative.
\end{remark}

% product measures }}}2

\subsection{Fubini theorems} % {{{2 

We can get similar results under varying assumptions. If $f$ is nonnegative, then all we require is that $f$ be measurable. But if $f$ is signed, then we require $f$ be $L^1$.

In what follows, let $(E,\mathcal{A},\mu)$ and $(F,\mathcal{B},\nu)$ be $\sigma$-finite.

\begin{theorem}[Fubini-Tonelli]
\label{thm_fubini-tonelli}
	Suppose $f:E\times F\to [0,\infty]$ be measurable (with respect to the product measure). Then, 
	\begin{enumerate}
		\item Letting $x\in E$ and $y\in F$, the functions 
			\begin{align*}
				x &\mapsto \int_F f(x,y)\ \nu(dy), \\
				y &\mapsto \int_E f(x,y)\ \mu(dx)
			\end{align*}
			are $\mathcal{A}$- and $\mathcal{B}$-measurable, respectively.
		\item 
			\begin{equation*}
				\int_{E\times F} f\ d\mu\otimes\nu 
				= \int_E \left( \int_F f(x,y)\ \nu(dy) \right) \mu(dx)
				= \int_F \left( \int_E f(x,y) \mu(dx) \right) \nu(dy).
			\end{equation*}
	\end{enumerate}
\end{theorem}
\begin{proof}
	\todo{prove}
\end{proof}

\begin{theorem}[Fubini-Lebesgue]
	Suppose $f\in L^1(E\times F, \mathcal{A}\otimes\mathcal{B}, \mu\otimes\nu)$. Then 
	\begin{enumerate}
		\item 
			\begin{itemize}
				\item $f_x(y)\in L^1(F,\mathcal{B},\nu)$ for almost all $x$ ($\mu(dx)$ a.e.)
				\item $f_y(x)\in L^1(E,\mathcal{A},\mu)$ for almost all $y$ ($\nu(dy)$ a.e.)
			\end{itemize}
		\item 
			\begin{itemize}
				\item $\left( x \mapsto \int_F f(x,y)\ \nu(dy) \right)\in L^1(E,\mathcal{A},\mu)$
				\item $\left( y \mapsto \int_F f(x,y)\ \mu(dx) \right)\in L^1(F,\mathcal{B},\nu)$
			\end{itemize}
		\item (just as in the Fubini-Tonelli theorem)
			\begin{equation*}
				\int_{E\times F} f\ d\mu\otimes\nu 
				= \int_E \left( \int_F f(x,y)\ \nu(dy) \right) \mu(dx)
				= \int_F \left( \int_E f(x,y) \mu(dx) \right) \nu(dy).
			\end{equation*}
	\end{enumerate}
\end{theorem}
\begin{proof}
	\todo{prove}
\end{proof}

% Fubini theorems }}}2

% }}}1

\section{unsorted} % {{{1 

Let $(E, \mathcal{A}, \mu)$ be a measure space, and $U$ be a metric space.

\begin{theorem}
\label{thm_integral_depending_on_parameter}
	Let $u_0\in U$. If 
	\begin{equation*}
		f:U\times E \to \mathbb{R}
	\end{equation*}
	is such that
	\begin{enumerate}
		\item $f_u:E\to \mathbb{R}$ is measurable
		\item $f_x:U\to \mathbb{R}$ is continuous at $u_0\in E$ for almost all $x$
		\item for every $u\in U$ there exists $g\in L^1_+(E)$ such that $|f_u(x)|\leq g(x)$ for almost all $x$,
	\end{enumerate}
	then  
	\begin{equation*}
		F(u) \coloneqq \int_E f(u,x) \mu(dx)
	\end{equation*}
	is well defined for all $u\in U$ and continuous at $u_0$.
\end{theorem}
\begin{proof}
	Well definedness follows by assumption (3). For continuity at $u_0$, let $(u_n)\subset U$ be a sequence converging to $u_0$. By assumption (2), $f_x(u_n) \to f_x(u)$ for almost all $x$. Assumption (3) again allows us to invoke the dominated convergence theorem (Theorem \ref{thm_dct}) to get 
	\begin{equation*}
		\lim_{n\to\infty} \int_X f_{u_n} d\mu = \int_x f_{u_0} d\mu,
	\end{equation*}
	proving continuity of $F$ at $u_0$.
\end{proof}

\begin{remark}
	We may replace $\mathbb{R}$ with $\mathbb{C}$ in the above theorem.
\end{remark}

Under stronger conditions, we can make it so that the function $F$ is not only continuous at $u_0$, but also differentiable.

\begin{theorem}
	Let $I$ be an open interval of $\mathbb{R}$, and $u_0\in I$. Suppose 
	\begin{equation*}
		f:I\times E \to \mathbb{R}
	\end{equation*}
	is such that 
	\begin{enumerate}
		\item $f_u: E\to \mathbb{R}$ is in $L^1(E)$ for all $u\in I$
		\item $f_x: I\to \mathbb{R}$ is differentiable at $u_0$, i.e. $D_{f_x}(u_0)$ exists, for almost all $x$.
		\item there exists $g\in L^1(E)$ such that, for all $u\in I$, 
			\begin{equation*}
				|f_u(x)-f_{u_0}(x)| \leq g(x) |u-u_0|	
			\end{equation*}
			for almost all $x$.
	\end{enumerate}
	Then 
	\begin{equation*}
		F(u) = \int f(u,x)\mu(dx)
	\end{equation*}
	is well-defined and differentiable at $u_0$:
	\begin{equation*}
		F'(u_0) = \int D_{f_x}(u_0) \mu(dx).
	\end{equation*}
\end{theorem}
\begin{proof}
	\todo{prove, legall 2.13}
\end{proof}

\begin{example}[convolution]
	Let $\phi \in L^1(\mathbb{R})$ \todo{does $\phi$ need to be Borel?} and $h:\mathbb{R}\to\mathbb{R}$ be bounded continuous. Let us verify that the function 
	\begin{align*}
		f(u,x): \mathbb{R}\times\mathbb{R} &\to \mathbb{R} \\
		(u,x) &\to h(u-x)\phi(x)
	\end{align*}
	satisfies the conditions of Theorem \ref{thm_integral_depending_on_parameter}:
	\begin{enumerate}
		\item $f_u$ is measurable since it is the pointwise product of a continuous and measurable function.
		\item $f_x$ is continuous (everywhere) since the $\phi$ term becomes constant, so it is essentially a constant times $h$.
		\item $|f_u(x)|$ bounds itself and is in $L^1_+(E)$ since $\phi$ is integrable and $h$ is bounded, hence their product is integrable.
	\end{enumerate}
	Thus we can apply the theorem to define the \emph{convolution}
	\begin{equation*}
		h \ast \phi(u) \coloneqq \int h(u-x)\phi(x) \lambda(dx).
	\end{equation*}
	This is a bounded function, and theorem also implies it is continuous on all $\mathbb{R}$. 

	Now suppose $h$ is continuously differentiable, and both $h$ and $h'$ are bounded. Then $h\ast\phi$ is differentiable on $\mathbb{R}$, and 
	\begin{equation*}
		(h\ast\phi)' = h'\ast\phi.
	\end{equation*}
	\todo{finish proof}
\end{example}

% unsorted }}}1

\end{document}

