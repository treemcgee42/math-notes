\documentclass[12pt]{article}

\usepackage{../preamble}
\usepackage{../tikz_tm42}

\title{ML}
\author{Runi Malladi}

\begin{document}
\maketitle

\section{unsorted} % {{{1 

\subsection{transforming feature vectors} % {{{2 

\textbf{Situation}

We represent words as feature $n$-vectors. There exist two subclasses of words, say $\mathcal{C}$ and $\mathcal{D}$. We believe there is a relationship between a word and another word, e.g. 
\begin{itemize}
	\item a country and its capital 
	\item an English word and its Telugu translation
\end{itemize}
We specifically assume this relationship is a linear transformation, consistent between individual feature vectors, i.e. the same linear transform maps 
\begin{itemize}
	\item every country to its capital 
	\item every English word to its Telugu translation.
\end{itemize}

\textbf{Approximations}

In reality, the relationship is not exactly the same for all words; partly because of noise in measurements, partly because the mapping between $\mathcal{C}$ and $\mathcal{D}$ is not one-to-one (e.g. there isn't a direct translation for every English word).

Consider $m$ feature ($n$-)vectors belonging to $\mathcal{C}$. Arrange then as the rows of an $m\times n$ matrix $A$. Arrange their corresponding vectors in $D$ into an $(m\times n)$ matrix $B$. 

Let $R$ be an $(n\times n)$-matrix, and let's consider what the product $AR$ represents. Well the $i$th row of $AR$ is $A_{i,\bullet}R$, hence $R$ is a lienar map sending rows of $A$ to rows of $AR$. Hence we seek an $R$ which is the best approximation 
\begin{equation*}
	AR = B.
\end{equation*}

\textbf{measuring closeness}

Since we can't hope for $R$ to exactly map rows of $A$ to the rows of $B$, we seek a best approximation. But how do we measure how good an approximation is? One way is to use the Frobenius norm.

\begin{definition}
	Let $A$ be an $m\times n$ matrix. The \emph{Frobenius norm} of $A$ is the square root of the sum of the squares of its entries: 
	\begin{equation*}
		\|A\|_F = \sqrt{\sum_{i=1}^m \sum_{j=1}^n (A_{i,j})^2}.
	\end{equation*}
	Equivalently,
	\begin{equation*}
		\|A\|_F = \sqrt{\text{tr}(A^TA)}.
	\end{equation*}
\end{definition}

\begin{proposition}
	The two definitions are equivalent.
\end{proposition}
\begin{proof} 
	\begin{align*}
		\text{tr}(A^TA)
		=& \sum_{j=1}^n A^T_{j,\bullet}A_{\bullet, j} = \sum_{j=1}^n \left( \sum_{i=1}^m A^T_{j,i}A_{i,j}\right) \\
		=& \sum_{j=1}^n \sum_{i=1}^m A_{i,j}^2.
	\end{align*}
\end{proof}	

Using the Frobenius norm, our goal now is to minimize 
\begin{equation*}
	\text{Loss}(R) = \|AR-B\|_F.
\end{equation*}
We use gradient descent.

\begin{proposition}
	\begin{equation*}
		\nabla_R\text{Loss}(R) = 2A^T(AR-B).
	\end{equation*}
\end{proposition}
\begin{proof} 
	Be sure to check our conventions. Well 
	\begin{align*}
		\text{Loss}(R) 
		=& \|AR-B\|^2_F = \text{tr}((AR-B)^T(AR-B)) \\
		=& \text{tr}((R^TA^T - B^T)(AR-B)) \\
		=& \text{tr}(R^TA^TAR - R^TA^TB - B^TAR - B^TB).
	\end{align*}
	Then 
	\begin{align*}
		\frac{\partial}{\partial R}\text{Loss}(R) 
		=& \frac{\partial}{\partial R}\text{tr}(R^T(A^TA)R) - \frac{\partial}{\partial R}(R^T(A^TB)) - \frac{\partial}{\partial R}((B^TA)R) - \frac{\partial}{\partial R}(B^TB) \\ 
		=& R^T(A^TA + A^TA) - B^TA - B^TA = 2(R^TA^TA - B^TA) \\
		=& 2(R^TA^T - B^T)A.
	\end{align*}
	Then 
	\begin{equation*}
		\nabla_R\text{Loss}(R) = (\frac{\partial}{\partial R}\text{Loss}(R))^T = 2A^T(AR-B).
	\end{equation*}
\end{proof}	

% transforming feature vectors }}}2

% unsorted }}}1

\section{appendix} % {{{1 

\subsection{logistic regression} % {{{2 

The purpose of logistic regression is to take data associating various values of the independent variables to binary outcomes and produce a model which takes values of the independent variables and returns a probability of a binary outcome occuring.

\textbf{Background}

Consider $p$ to be the probability of an event occurring. We can further assume only two outcomes: either the event occurs, or it doesn't. We define the \emph{odds ratio} to be 
\begin{gather*}
	\text{odds ratio}: [0, 1) \to [0,\infty) \\
	p \mapsto \frac{p}{1-p}.
\end{gather*}
We define the \emph{log-odds ratio}, or \emph{logit}, to be 
\begin{gather*}
	\text{logit}: (0, 1) \to (-\infty, \infty) \\
	p \mapsto \log\left(\frac{p}{1-p}\right).
\end{gather*}
The graphs of these functions are depicted below (Figure \ref{fig_odds_logodds_graphs}):
\begin{figure}[h]
\centering
\begin{tikzpicture}
	\begin{axis}[
		legend style={at={(0.98,0.02)},anchor=south east},
		axis lines=middle,
		axis line style={-},
		xtick={0, 0.5, 1},
		ymajorticks=false,
		xmin=-0.1, xmax=1.1,
		ymin=-2.5, ymax=2.5
	]
		\addplot[domain=0:0.99, color=funkyfuture8_lightblue] {x / (1-x)};
		\addlegendentry{odds}
		
		\addplot[domain=0.01:0.99, color=funkyfuture8_brightorange]{ln(x/(1-x))};
		\addlegendentry{log-odds}

		\addplot[mark=none, black, dotted] coordinates {(1,-2.5) (1,2.5)};
	\end{axis}
\end{tikzpicture}
\caption{Graphs of odds and log-odds functions.}
\label{fig_odds_logodds_graphs}
\end{figure}

\textbf{Assumptions}

The fundamental assumption of logistic regression is a linear relationship between the independent variables and the log-odds. 

For instance, consider a situation with two independent variables $X_1,X_2$ which determine a binary outcome (either $0$ or $1$). We assume 
\begin{itemize}
	\item it is reasonable to model the probability of an input $(x_1,x_2)$ resulting in the binary outcome $1$. That is, each outcome $y_i$ is Bernoulli distributed. 
	\item this relationship is linear: letting $p$ denote the probability of $(x_1,x_2)$ producing $1$, we have 
		\begin{equation*}
			\log\left(\frac{p}{1-p}\right) = \beta_0 + \beta_1x_1 + \beta_2x_2.
		\end{equation*}
		for some $\beta_0,\beta_1,\beta_2\in\mathbb{R}$. Note that the $\beta_i$ do not depend on the $x_i$.
\end{itemize}

\textbf{Objective}

Assuming a linear relationship between log-odds and the independent variables 
\begin{equation*}
	\text{logit}(p(x)) = \beta \cdot \begin{pmatrix} 1 \\ x \end{pmatrix} = \beta_0 + \beta_1x_1 + \cdots \beta_nx_n,
\end{equation*}
the objective of logistic regression is to determine (or approximate) the coefficients $\beta$ in the above linear combination. As a matter of convention, by $\beta\cdot x$ or $\beta^Tx$ we will mean the above dot product, where we have added an $x_0=1$ term to the original $x$.

As we will demonstrate, once the coefficients $\beta$ have been determined, we can determine the probability $p(x)$ of input $x$ succeeding using the following formula:
\begin{equation*}
	p(x) = \frac{1}{1+e^{-\beta^Tx}}.
\end{equation*}

% logistic regression }}}2

\subsection{naive Bayes classifier} % {{{2 

The situation is when we have some number of features (random variables) $x=(x_1,\dots,x_n)$ and finitely many possible outcomes $C_k$. The \emph{naive Bayes probabilistic model} is a computation of the conditional probabilities $p(C_k | x)$, i.e. the probability of the outcome $C_k$ given the features $x$. The \emph{naive Bayes classifier} determines which outcome $C_k$ is most likely given the feature $x$. It essentially picks the largest conditional probability $p(C_k | x)$. What makes both of these models "naive" is their assumption on the independence of the features in $x$, which greatly simplifies the above computations.

\textbf{background}

The key insight is the simplification provided by assuming the features in $x$ are independent. Specifically, we are assuming that the probability of the $i$th feature equaling $x_i$ is indenpendent of the $j$th feature equaling $x_j$ (provided $i\neq j$). Then $p(x_i|x_j)=p(x_i)$. 

Let's see this simplification in action (this computation will be relevant later). By Bayes' rule, we can write a joint probability in terms of the conditional probability:
\begin{equation*}
	p(x,y)=p(x|y)p(y).
\end{equation*}
Using this, we can write 
\begin{align*}
	p(C_k,x_1,\dots,x_n) 
	=& p(x_1,\dots,x_n,C_k) \\
	=& p(x_1 | x_2,\dots,x_n,C_k) p(x_2,\dots,x_n,C_k) \\
	=& \cdots \\
	=& p(x_1 | x_2,\dots,x_n,C_k)p(x_2|x_3,\dots,x_n,C_k)\cdots p(x_{n-1}|x_n,C_k)p(x_n|C_k)p(C_k).
\end{align*}
By our independence assumption, 
\begin{equation*}
	p(x_i | x_{i+1},\dots,x_n,C_k) = p(x_i|C_k).
\end{equation*}
So 
\begin{equation*}
	p(C_k,x_1,\dots,x_n) = p(C_k)\prod_{i=1}^n p(x_i|C_k).
\end{equation*}

\textbf{assumptions}

We assume that the values of each feature are independent of each other. For example, given a collection of features $X=(X_1,\dots,X_n)$, the features $X_i$ and $X_j$ are independent (provided $i\neq j$). 

This is rarely true, and the dependence among features is often significant. Suppose we are trying to classify fruits. Suppose we are tracking two features: color and taste. Is it reasonable to assume these two are independent? You have to decide.

\textbf{the probabilistic model}

As mentioned earlier, the naive Bayes probabilistic model is concerned with computing the conditional probability $p(C_k | x)$. 

\begin{proposition}
	Under the assumptions of the naive Bayes model, 
	\begin{equation*}
		p(C_k | x) = \frac{p(C_k)}{p(x)}\prod_{i=1}^n p(x_i | C_k).
	\end{equation*}
\end{proposition}
\begin{proof}
	We compute using the independence assumption:
	\begin{align*}
		p(C_k | x)
		=& \frac{p(C_k,x)}{p(x)} = \frac{p(C_k,x_1,\dots,x_n)}{p(x)} \\
		=& \frac{p(C_k)}{p(x)}\prod_{i=1}^n p(x_i | C_k).
	\end{align*}
\end{proof}

\textbf{the classifier}

The naive Bayes classifier is concerned with, given a fixed $x$, which conditional probability $p(C_k|x)$ is the largest. Intuitively, it determines which outcome $C_k$ is most likely given the features $x$. 

\begin{proposition}
	For a fixed $x$, 
	\begin{equation*}
		\underset{k\in\{1,\dots,K\}}{\text{argmax}}\ p(C_k|x) = \underset{k\in\{1,\dots,K\}}{\text{argmax}}\ p(C_k)\prod_{i=1}^n p(x_i|C_k).
	\end{equation*}
\end{proposition}
\begin{proof}
	The point is that we can remove the $p(x)$ from the probabilistic model. We can do this because we fix $x$, so $p(x)$ is a constant factor shared by all terms we are taking the $\text{argmax}$ over.
\end{proof}

\begin{definition}
	The naive Bayes classifier is the function 
	\begin{align*}
		X_1\times \cdots X_n \to& \{C_k\} \\
		(x_1,\dots,x_n) \mapsto& \underset{k\in\{1,\dots,K\}}{\text{argmax}}\ p(C_k)\prod_{i=1}^n p(x_i|C_k).
	\end{align*}
\end{definition}

Let's consider a special case. There are two possible outcomes $C_1, C_2$ and $n$ features $X_1,\dots,X_n$. Then we just need to compare the two conditional probabilities $p(C_1|x)$ and $p(C_2|x)$. Assuming $p(C_2|x)\neq 0$, we can consider the ratio 
\begin{equation*}
	R=\frac{p(C_1|x)}{p(C_2|x)}=\frac{p(C_1)\prod_{i=1}^n p(x_i|C_k)}{p(C_2)\prod_{i=1}^n p(x_i|C_2)}.
\end{equation*}
If $R>1$ the $C_1$ is more likely, if $R<1$ then $C_2$ is more likely, and if $R=1$ the both are equally likely. 

% naive Bayes classifier }}}2

\subsection{additive smoothing} % {{{2 

Additive smoothing helps assign a nonzero probability to conditional properties which are calculated to be zero from the data.

\textbf{background} 

Consider a $d$-dimensional multinomial distribution, of which we take $N$ trials. This means that there are $d$ possible outcome classes, and we have classified $N$ data points into these classes (one at a time, so the same data point may be placed in different classes from one trial to another). Suppose we take another data point $x_i$. We want to determine the probability.
\begin{equation*}
	p(x_i | C_k).
\end{equation*}
By Bayes' theorem, we know 
\begin{equation*}
	p(x_i | C_k) = \frac{p(x_i,C_k)}{p(C_k)}.
\end{equation*}
Let $\text{freq}(x_i,C_k)$ denote the number of times $x_i$ was classified as $C_k$ (in the $N$ trials we took of the multinomial distribution). Note this may be 0. Also let $N_k$ be the number of times a data point was sorted into class $C_k$. Then, assuming $N_k\neq 0$, 
\begin{gather*}
	p(x_i,C_k) = \frac{\text{freq}(x_i,C_k)}{N}, \\
	p(C_k) = \frac{N_k}{N}, \\
	p(x_i | C_k) = \frac{\text{freq}(x_i,C_k)}{N_k}.
\end{gather*}
Now there are two potential problems here. The first is that $p(x_i|C_k)=0$ if $\text{freq}(x_i,C_k)$. But it might not be accurate to say this; just because $x_i$ was never classified as $C_k$ in any over our $N$ trials doesn't necessarily mean that it should never by classified as $C_k$. We took a finite number of trials which may not accurate represent the true probabilities. The second issue is that we assumed $N_k\neq 0$, i.e. that during our $N$ trials at least one data point was classified as $C_k$. This may also not be reasonable to assume.

\textbf{assumptions}

We assume that each data point $x_i$ has a nonzero probability of being $C_k$. The extent of this assumption is codified in the "pseudocount" parameter in the additive smoothing formula.

\textbf{formula}

Additive smoothing redefines all conditional probabilities $p(x_i|C_k)$:

\begin{definition}
	Additive smoothing with pseudocount $\alpha>0$ is 
	\begin{equation*}
		\hat{p}(x_i | C_k) = \frac{\text{freq}(x_i,C_k)+\alpha}{N_k + d\alpha}.
	\end{equation*}
\end{definition}

The reason this is reasonable is the following:

\begin{proposition}
	Suppose $\alpha$ is an integer. Consider $n$ data points $x_1,\dots,x_n$. Suppose we have a multinomial distribution with $N+dn\alpha$ trials, which is identical to the old distribution for the first $N$ trails and afterwards each data point $x_i$ was classified $\alpha$ times into each $C_k$. Then 
	\begin{equation*}
		p_{\text{new}}(x_i|C_k) = \frac{\text{freq}_{\text{old}}(x_i,C_k) + \alpha}{(N_k)_{\text{old}} + n\alpha}.
	\end{equation*}
\end{proposition}
\begin{proof}
	Well 
	\begin{gather*}
		\text{freq}_{\text{new}}(x_i, C_k) = \text{freq}_{\text{old}}(x_i, C_k) + \alpha, \\
		(N_k)_{\text{new}} = (N_k)_{\text{old}}+n\alpha.
	\end{gather*}
	Just plug that into 
	\begin{equation*}
		p_{\text{new}}(x_i|C_k) = \frac{\text{freq}_{\text{new}}(x_i,C_k)}{(N_k)_{\text{neq}}}.
	\end{equation*}
\end{proof}

% additive smoothing }}}2

% appendix }}}1

\end{document}
