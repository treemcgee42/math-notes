\documentclass[12pt]{article}

\usepackage{../preamble}
\usepackage{../tikz_tm42}

\title{ML}
\author{Runi Malladi}

\begin{document}
\maketitle

\section{appendix} % {{{1 

\subsection{logistic regression} % {{{2 

The purpose of logistic regression is to take data associating various values of the independent variables to binary outcomes and produce a model which takes values of the independent variables and returns a probability of a binary outcome occuring.

\textbf{Background}

Consider $p$ to be the probability of an event occurring. We can further assume only two outcomes: either the event occurs, or it doesn't. We define the \emph{odds ratio} to be 
\begin{gather*}
	\text{odds ratio}: [0, 1) \to [0,\infty) \\
	p \mapsto \frac{p}{1-p}.
\end{gather*}
We define the \emph{log-odds ratio}, or \emph{logit}, to be 
\begin{gather*}
	\text{logit}: (0, 1) \to (-\infty, \infty) \\
	p \mapsto \log\left(\frac{p}{1-p}\right).
\end{gather*}
The graphs of these functions are depicted below (Figure \ref{fig_odds_logodds_graphs}):
\begin{figure}[h]
\centering
\begin{tikzpicture}
	\begin{axis}[
		legend style={at={(0.98,0.02)},anchor=south east},
		axis lines=middle,
		axis line style={-},
		xtick={0, 0.5, 1},
		ymajorticks=false,
		xmin=-0.1, xmax=1.1,
		ymin=-2.5, ymax=2.5
	]
		\addplot[domain=0:0.99, color=funkyfuture8_lightblue] {x / (1-x)};
		\addlegendentry{odds}
		
		\addplot[domain=0.01:0.99, color=funkyfuture8_brightorange]{ln(x/(1-x))};
		\addlegendentry{log-odds}

		\addplot[mark=none, black, dotted] coordinates {(1,-2.5) (1,2.5)};
	\end{axis}
\end{tikzpicture}
\caption{Graphs of odds and log-odds functions.}
\label{fig_odds_logodds_graphs}
\end{figure}

\textbf{Assumptions}

The fundamental assumption of logistic regression is a linear relationship between the independent variables and the log-odds. 

For instance, consider a situation with two independent variables $X_1,X_2$ which determine a binary outcome (either $0$ or $1$). We assume 
\begin{itemize}
	\item it is reasonable to model the probability of an input $(x_1,x_2)$ resulting in the binary outcome $1$. That is, each outcome $y_i$ is Bernoulli distributed. 
	\item this relationship is linear: letting $p$ denote the probability of $(x_1,x_2)$ producing $1$, we have 
		\begin{equation*}
			\log\left(\frac{p}{1-p}\right) = \beta_0 + \beta_1x_1 + \beta_2x_2.
		\end{equation*}
		for some $\beta_0,\beta_1,\beta_2\in\mathbb{R}$. Note that the $\beta_i$ do not depend on the $x_i$.
\end{itemize}

\textbf{Objective}

Assuming a linear relationship between log-odds and the independent variables 
\begin{equation*}
	\text{logit}(p(x)) = \beta \cdot \begin{pmatrix} 1 \\ x \end{pmatrix} = \beta_0 + \beta_1x_1 + \cdots \beta_nx_n,
\end{equation*}
the objective of logistic regression is to determine (or approximate) the coefficients $\beta$ in the above linear combination. As a matter of convention, by $\beta\cdot x$ or $\beta^Tx$ we will mean the above dot product, where we have added an $x_0=1$ term to the original $x$.

As we will demonstrate, once the coefficients $\beta$ have been determined, we can determine the probability $p(x)$ of input $x$ succeeding using the following formula:
\begin{equation*}
	p(x) = \frac{1}{1+e^{-\beta^Tx}}.
\end{equation*}

% logistic regression }}}2

% appendix }}}1

\end{document}
