\documentclass[12pt]{article}

\usepackage{../preamble}

\setcounter{secnumdepth}{5}
\newtheorem{para}[theorem]{}

\title{Lecture 1}
\author{Runi Malladi}

\begin{document}
\maketitle

\section{Galois connections} % {{{1 

\begin{definition}
	Let $X,Y$ be partially-ordered sets, and suppose we have maps 
	\begin{equation*}
		X \stackrel[g]{f}{\rightleftarrows} Y
	\end{equation*}
	such that
	\begin{itemize}
		\item (order reversing)
			\begin{align*}
				x_1 \leq x_2 \quad \Rightarrow& \quad f(x_1) \geq f(x_2) \\
				y_1 \leq y_2 \quad \Rightarrow& \quad g(y_1) \geq g(y_2)
			\end{align*}
		\item (inflationary)
			\begin{align*}
				g(f(x)) \geq x \quad &\forall x\in X \\
				f(g(y)) \geq y \quad &\forall y\in Y
			\end{align*}
	\end{itemize}
	In this case, we say that $f$ and $g$ form an \emph{antitone Galois connection}.
\end{definition}

\begin{proposition}
\label{prop_antitone_Galois}
	Let $f,g$ be an antitone Galois connection. Then they restrict to bijections on their images:
	\begin{equation*}
		\text{im}(g) \stackrel[\tilde{g}]{\tilde{f}}{\rightleftarrows} \text{im}(f).
	\end{equation*}
\end{proposition}
\begin{proof}
	We show that $f$ restricts to a bijection, and the case for $g$ is analagous. By the inflationary property, $x\leq g(f(x))$ for any $x\in X$. Then by the order reversing property, $f(x) \geq f(g(f(x)))$. But applying the inflationary property to $y=f(x)$ yields $f(x) \leq f(g(f(x)))$, and so $f(x)=f(g(f(x)))$.
\end{proof}

\begin{remark}
	Since $g\circ f$ is the identity on $\text{im}(g)$, we have $g\circ f\circ g = g$ on all of $Y$. Then $g\circ f) \circ (g\circ f) = g\circ f$, so $g\circ f$ is idempotent on all of $X$. We often associate idempotent operators with closure-like qualities. So we may think about
	\begin{equation*}
		g(f(X)) \subset X
	\end{equation*}
	as a sort of ``closed object'' in $X$. 
\end{remark}

\begin{example}[classical Galois theory]
	Let $K/k$ be a finite Galois extension with Galois group $\text{Gal}(K/k)$. Then there is an antitone Galois connection between the intermediate fields $k\subset F\subset K$ and the subgroups $H<\text{Gal}(K/k)$, given by sending $F\mapsto \text{Gal}(K/F)$ and $H\mapsto K^H$ (the subfield of $K$ fixed by the automorphisms in $H$).

	By the above, this statement is straightforward, provided we can show that every subgroup of $\text{Gal}(K/k)$ is isomorphic to $\text{Gal}(K/F)$ for some intermediate field $F$, and that every intermediate field of $K/k$ consists exactly of the elements of $K$ fixed by the automorphisms in a subgroup of $\text{Gal}(K/k)$.
\end{example}

\begin{corollary}
	Suppose $f,g$ are 
	\begin{itemize}
		\item (order preserving)
		\item (inflationary on $X$) 
			\begin{equation*}
				g(f(x)) \geq x \quad \forall x\in X
			\end{equation*}
		\item (deflationary on $Y$)
			\begin{equation*}
				f(g(y)) \leq y \quad \forall y\in Y
			\end{equation*}
	\end{itemize}
	Then $f,g$ restrict to bijections on their images.
\end{corollary}
\begin{proof}
	Apply Proposition \ref{prop_antitone_Galois} to $X$ and $Y$, with the order on $Y$ reversed.
\end{proof}

\begin{example}[extension/contraction of ideals]
	Let $\phi:A\to B$ be a ring map (under our assumptions). Consider the pair 
	\begin{gather*}
		\{\text{ideals in } A\} \rightleftarrows \{\text{ideals in }B \} \\
		\mathfrak{a} \mapsto \mathfrak{a}^e \\
		\mathfrak{b}^e \leftarrow \mathfrak{b}
	\end{gather*}
	where
	\begin{itemize}
		\item $\mathfrak{a}^e$ is the extension of $\mathfrak{a}$ in $B$, i.e. $B\phi(\mathfrak{a})$ which means the $B$-ideal generated by $\phi(\mathfrak{a})$. 
		\item $\mathfrak{b}^e$ is the ... \todo{this}
	\end{itemize}
\end{example}

\begin{example}
	As an extension of the previous example, let $\mathfrak{a}$ be an ideal and consider the quotient map $A\to A/\mathfrak{a}$... \todo{finish}
\end{example}

% Galois connections }}}1

\section{motivating example: the Gelfand-Kolmogorov theorem} % {{{1 

\begin{definition}
	Let $(X, d)$ be a compact metric space. Let 
	\begin{align*}
		\mathcal{O}_X \coloneqq& \ C(X) \coloneqq C(X,\mathbb{R}) \\
		\mathcal{X} \coloneqq& \ \text{Spec}_m(\mathcal{O}_X) \coloneqq \{\text{maximal ideals }\mathfrak{m}\subset \mathcal{O}_X \}.
	\end{align*}
\end{definition}

\begin{definition}
	For all $x\in X$, consider the evaluation map 
	\begin{gather*}
		\mathcal{O}_X \overset{\epsilon_x}{\longrightarrow} \mathbb{R} \\
		f \mapsto f(x)
	\end{gather*}
	We write $\mathfrak{m}_x \coloneqq \text{ker}(\epsilon_x) \subset \mathcal{O}_X$.
\end{definition}

\begin{proposition}
	$\mathfrak{m}_x\subset\mathcal{O}_X$ is a maximal ideal. 
\end{proposition}
\begin{proof}
	The quotient $\mathcal{O}_X / \mathfrak{m}_x \cong \mathbb{R}$ is a field.
\end{proof}

\begin{definition}
	For $\mathfrak{m}\in \text{Spec}_m(\mathcal{O}_X)$, define 
	\begin{equation*}
		v(\mathfrak{m}) \coloneqq \{ x\in X : f(x)=0 \text{ for all } f\in\mathcal{m} \}.
	\end{equation*}
	We call $v(\mathfrak{m})$ the \emph{vanishing locus} of $\mathfrak{m}$.
\end{definition}

\begin{remark}
	$\mathfrak{m}_x\subset\mathcal{O}_X$ and $v(\mathfrak{m})\subset X$ are in some sense dual notions: $\mathfrak{m}_x$ is the set of functions which vanish at $x$, while $v(\mathfrak{m})$ is the set of points $x$ on which each $f\in\mathfrak{m}$ vanishes.
\end{remark}

\begin{proposition}[weak Nullstellensatz]
\label{prop_weak_nullstellensatz}
	For $\mathfrak{m} \in \mathfrak{X}$, the vanishing locus $v(\mathfrak{m})$ is nonempty. 
\end{proposition}
\begin{proof}
	Suppose $v(\mathfrak{m})$ is empty. Then for all $x\in X$ there exists $f_x\in\mathfrak{m}$ such that $f_x(x)\neq 0$. Then also $f_x^2(x)\neq 0$ (and $f_x^2\in\mathfrak{m}$). For the sake of notation we replace $f_x$ with $f_x^2$, whence we can now assume $f_x(x) > 0$ strictly. 

	We claim $f_x\geq 0$ on all of $X$. Let's show this. Since $f_x(x)>0$ and $f_x$ is continuous, there exists a neighborhood $U_x$ of $x$ such that $f_x|_{U_x}>0$. Doing this process for all $x\in X$ yields an open cover $\{U_x\}_{x\in X}$ such that $f_x|_{U_x}>0$. Since $X$ is compact, we may find a finite subcover $U_1,\dots U_N$. Define 
	\begin{equation*}
		f = f_{x_1} + \cdots + f_{x_N}.
	\end{equation*}
	By construction $f\in\mathfrak{m}$.	Also $f(x)>0$ for all $x\in X$, since there exists $j$ such that $x\in U_j$ and so $f_{x_j}(x)>0$ and $f_{x_i}(x)\geq 0$ for all $i\neq j$. But then $1/f$ is continuous, so $1\in\mathfrak{m}$ which contradicts the maximality of $\mathfrak{m}$.
\end{proof}

\begin{remark}
	Compare with Theorem 2.12 in cstar \todo{ref}.
\end{remark}

\begin{proposition}
\label{prop_x_specmOX_bijection}
	The map 
	\begin{gather*}
		X \longrightarrow \text{Spec}_m(\mathcal{O}_X) \\
		x \mapsto \mathfrak{m}_x
	\end{gather*}
	is bijective.
\end{proposition}
\begin{proof}
	First we show injectivity. Suppose $p\neq q\in X$. Consider the functions 
	\begin{equation*}
		f_{q,\delta} = 
		\begin{cases}
			\delta - d(x,q) & d(x,q) > 0 \\
			0 & \text{otherwise}
		\end{cases}.
	\end{equation*}
	For any $\delta>0$ and $q\in X$ it then follows that $f_{q,\delta}\in \mathcal{O}_X$. Now let $\delta = \frac{1}{3}d(p,q)$. Then $f_{q,\delta}\in\mathfrak{m}_p$ but not in $\mathfrak{m}_q$. This shows injectivity.

	For surjectivity, pick some $\mathfrak{m}\in\text{Spec}_m(\mathcal{O}_X$). By Proposition \ref{prop_weak_nullstellensatz}, there exists $x\in v(\mathfrak{m})$, so in particular $\mathfrak{m}\subset \mathfrak{m}_x\coloneqq \text{ker}(\epsilon_x)$. But since $\mathfrak{m}$ is maximal, it must be that $\mathfrak{m}=\mathfrak{m}_x$. 
\end{proof}

We have thus achieved a setwise correspondence. One way of thinking about this is that we can recover the points of $X$ from $\text{Spec}_m(\mathcal{O}_X)$ (and vice versa). A natural next question is whether we can recover the topology on $X$ as well.

\begin{definition}
	For $S\subset X$, define 
	\begin{equation*}
		I(S) = \{f\in \mathcal{O}_X : f|_S=0 \} \subset \mathcal{O}_X.
	\end{equation*}
	For $E\subset \mathcal{O}_X$, define 
	\begin{equation*}
		V(E) = \{x\in X : f(x)=0 \text{ for all } f\in E \} \subset X.
	\end{equation*}
\end{definition}

\begin{proposition}
	We make the following observations:
	\begin{enumerate}
		\item For $x\in X$, 
			\begin{equation*}
				I(\{x\}) = \mathfrak{m}_x.
			\end{equation*}
		\item For $S\subset X$, 
			\begin{equation*}
				I(S) = \bigcap_{x\in S} I(\{x\}) = \bigcap_{x\in S}\mathfrak{m}_x \subset \mathcal{O}_X
			\end{equation*}
			is an ideal (\todo{ref}). It is proper if $S$ is nonempty, by Proposition \ref{prop_weak_nullstellensatz}.
		\item $I$ is order reversing: 
			\begin{equation*}
				S_1\subset S_2\subset X \quad \Rightarrow \quad I(S_1) \supset I(S_2).
			\end{equation*}
		\item The pair $I,V$ is inflationary on $X$:
			\begin{equation*}
				S\subset X \quad \Rightarrow \quad S\subset V(I(S)).
			\end{equation*}
		\item $E$ is order reversing: 
			\begin{equation*}
				E_1\subset E_2\subset \mathcal{O}_X \quad \rightarrow \quad V(E_1) \supset V(E_2).
			\end{equation*}
		\item The pair $I, V$ is inflationary on $\mathcal{O}_X$:
			\begin{equation*}
				E\subset \mathcal{O}_X \quad \Rightarrow \quad E\subset I(V(E)).
			\end{equation*}
	\end{enumerate}
\end{proposition}

\begin{corollary}
	The maps 
	\begin{equation*}
		\left\{\parbox{5em}{\centering{subsets of $X$}}\right\} \stackrel[V]{I}{\rightleftarrows} \left\{\parbox{5em}{\centering{ideals in $\mathcal{O}_X$}}\right\} 
	\end{equation*}
	form an antitone Galois connection.
\end{corollary}

\begin{proposition}
	If $E\subset\mathcal{O}_X$, then $V(E)=V(\mathcal{O}_XE)$, where $\mathcal{O}_XE\subset\mathcal{O}_X$ is the ideal generated by $E$.
\end{proposition}
\begin{proof}
	Since $E\subset \mathcal{O}_XE$ and $V$ is order reversing, we have that $V(E) \supset V(\mathcal{O}_XE)$. Conversely, let $a\in V(E)$. Then there $f(a)=0$ for all $f\in E$. But any $h\in \mathcal{O}_XE$ has the form $h=\sum_i g_if_i$, where $g_i\in\mathcal{O}_X$ and $f_i\in E$. Then it follows that $h(a)=0$, so $a\in V(\mathcal{O}_XE)$, and hence $V(E) \subset V(\mathcal{O}_XE)$.
\end{proof}

Recall that we though of images of maps belonging to an antitone Galois connection as ``closed'' in some sense. The next proposition shows that this notion coincides with the topological one in this case:

\begin{proposition}
	For $\mathfrak{a}\subset\mathcal{O}_X$ an ideal, $V(\mathfrak{a})\subset X$ is closed in the topological sense.
\end{proposition}
\begin{proof}
	The arbitrary union of closed sets is closed, and 
	\begin{equation*}
		V(\mathfrak{a}) = \bigcap_{f\in\mathfrak{a}}V(f) = \bigcap_{f\in\mathfrak{a}}f^{-1}(\{0\})
	\end{equation*}
	and $\{0\}\subset X$ is closed and $f$ is continuous, hence $f^{-1}(\{0\})$ is closed.
\end{proof}

So ``closed'' in the sense of Galois connections implies closed in the topological sense. Recall that the composition of the two functions in a Galois connection were thought of as acting like closures. This following makes this concrete in the topological sense:

\begin{proposition}
	If $S\subset X$, then $V(I(S))=\overline{S}$. 
\end{proposition}
\begin{proof}
	By the previous proposition, we know $V(I(S))$ is closed, and by the inflationary property $S\subset V(I(S))$, and so $\overline{S}\subset V(I(S))$. Conversely, it suffices to show that if $x\not\in\overline{S}$ then $x\not\in V(I(S))$. So $x\not\in\overline{S}$ implies there exists an $\epsilon$-ball centered at $x$ which avoids $S$. Take $\delta=\epsilon/2$. Then the bump function $f_{x,\delta}$ is 0 on $S$, and hence $f_{x,\delta}\in I(S)$. But also $f_{x,\delta}(x)=\delta>0$, and so $x\not\in V(I(S))$.
\end{proof}

\begin{corollary}
	Let $\mathfrak{a}\subset\mathcal{O}_X$. Then points in $V(\mathfrak{a})\subset X$ corresponds to maximal ideals $\mathfrak{a}\subset\mathfrak{b}\subset\mathcal{O}_X$ containing $\mathfrak{a}$.
\end{corollary}
\begin{proof}
	Let $x\in V(\mathfrak{a})$. Then $\{x\}\subset V(\mathfrak{a})$ and by the order-reversing property $I(\{x\})\supset I(V(\mathfrak{a})) \supset \mathfrak{a}$. But $I(\{x\})=\mathfrak{m}_x$. 
\end{proof}

\begin{definition}
	For an ideal $\mathfrak{a}\subset\mathcal{O}_X$, define 
	\begin{equation*}
		\mathcal{V}(\mathfrak{a}) = \{\mathfrak{m}\in\text{Spec}_m(\mathcal{O}_X : \mathfrak{m}\supset\mathfrak{a})\} \subset\text{Spec}_m(\mathcal{O}_X).
	\end{equation*}
\end{definition}

The previous corollary suggests $\mathcal{V}(\mathfrak{a}) \subset \mathcal{O}_X$ corresponds to the points of $V(\mathfrak{a})\subset X$.

\begin{corollary}
	$\mathcal{V}(-)$ is order reversing.
\end{corollary}

Recalling the bijection in Proposition \ref{prop_x_specmOX_bijection}, denote by $m(V(f))$ the image of $V(f)\subset X$ under the map $x\mapsto \mathfrak{m}_x$.

\begin{proposition}
	$\mathfrak{m}(V(f)) = \mathcal{V}((f))$.
\end{proposition}
\begin{proof}
	$x\in V(f)$ if and only if $f(x)=0$ if and only if $f\in\mathfrak{m}_x$ if and only if $(f)\subset\mathfrak{m}_x$ if and only if $\mathfrak{m}_x\in \mathcal{V}((f))$. 
\end{proof}

\begin{corollary}
	If $\mathfrak{a}\subset\mathcal{O}_X$ is an ideal, then $\mathfrak{m}(V(\mathfrak{a}))=\mathcal{V}(\mathfrak{a})$.
\end{corollary}
\begin{proof}
	We calculate
	\begin{equation*}
		\mathfrak{m}(V(\mathfrak{a})) = m\left(\bigcap_{f\in\mathfrak{a}} V(f)\right) = \bigcap_{f\in\mathfrak{a}}m(V(f)) = \bigcap_{f\in\mathfrak{a}}\mathcal{V}(f)=V(\mathfrak{a},
	\end{equation*}
	where the commuting with $\bigcap$ is justified since $\mathfrak{m}(-)$ is bijective (Proposition \ref{prop_x_specmOX_bijection}) and we also used the fact that $V(f)=V((f))$ \todo{ref}.
\end{proof}

So since $V(\mathfrak{a})\subset X$ is closed, and $\mathcal{V}(\mathfrak{a})\subset\mathcal{O}_X$ corresponds to it, we may think of $\mathcal{V}(a)$ as closed too:

\begin{definition}
	The topology on $\text{Spec}_m(\mathcal{O}_X)$ whose closed sets are of the form $\mathcal{V}(\mathfrak{a})$ for some ideal $\mathfrak{a}\subset\mathcal{O}_X$ is called the \emph{Zariski topology} on $\text{Spec}_m(\mathcal{O}_X)$. 
\end{definition}

\begin{corollary}
	$\mathfrak{m}(-):X\to\text{Spec}_m(\mathcal{O}_X)$ is a homeomorphism.
\end{corollary}
\begin{proof}
	It is bijective by Proposition \ref{prop_x_specmOX_bijection}. Suppose $S\subset X$ is closed. Then it is equal to $V(I(S))$. Since $I(S)$ is an ideal, it follows that $\mathfrak{m}(V(I(S)))=\mathcal{V}(I(S))$ which is closed by definition. Hence the map is closed.
\end{proof}

\begin{definition}
	For $f\in \mathcal{O}_X$, define the \emph{distinguished} open set in $X$ by 
	\begin{equation*}
		D(f) \coloneqq X - V(f).
	\end{equation*}
	Dually, define an open set in $\text{Spec}_m(\mathcal{O}_X)$ by 
	\begin{equation*}
		\mathcal{D}(f) \coloneqq X - \mathcal{V}(f).
	\end{equation*}
\end{definition}

\begin{corollary}
	$m(D(f)) = \mathcal{D}(f)$.
\end{corollary}

\begin{proposition}
	The distinguished open sets form a basis for the topology on $X$.
\end{proposition}
\begin{proof}
	The open balls are a basis for the topology on $X$. But $B(x,\epsilon) = D(f_{x,\epsilon})$. \todo{check holo}
\end{proof}

So we have built up that for a compact metric space $X$, we may associate a commutative $\mathbb{R}$-algebra $\mathcal{O}_X$, and can again obtain a (space homeomorphic to our original) compact metric space. An outstanding question is whether a continuous function $\phi:X\to Y$ induces a continuous map $\text{Spec}_m(\mathcal{O}_X)\to\text{Spec}_m(\mathcal{O}_Y)$.

\begin{proposition}
	A continuous function $\phi:X\to Y$ induces a ring homomorphism $\phi^\#: \mathcal{O}_Y \to \mathcal{O}_X$ in the other direction sending $f\mapsto f\circ\phi$. 
\end{proposition}

\begin{corollary}
	$\phi:X\to Y$ induces a function from the ideals of $\mathcal{O}_X$ to the ideals of $\mathcal{O}_Y$, sending $\mathfrak{a}$ to its contraction $(\phi^\#)^{-1}(\mathfrak{a})$. 
\end{corollary}

This is almost what we want: we want this to restrict to maximal ideals.

\begin{proposition}
	$\tilde{\phi}$ carries maximal ideals to maximal ideals, hence restricts to a map 
	\begin{equation*}
		\tilde{\phi}: \text{Spec}_m(\mathcal{O}_X) \to \text{Spec}_m(\mathcal{O}_Y).
	\end{equation*}
	Moreover, $\tilde{\phi}(\mathfrak{m}_x) = \mathfrak{m}_{\phi(x)}$. \todo{diagram}
\end{proposition}

The last statement of the propositionis telling us that the algebraic map $\tilde{\phi}$ on maximal ideals encodes all the information about our initial continuous function $\phi$. This is reliant on the crucial fact, which is false in general, that the contraction of a maximal ideal along is maximal.

\begin{proof}
	Now we will show $(\phi^\#)^{-1} = \mathfrak{m}_{\phi(x)}$. Let $f\in \mathfrak{m}_{\phi(x)}$. Then $0=f(\phi(x)) = (\phi^\#(f))(x)$, which shows $\phi^\#(f)\in\mathfrak{m}_x$ and hence $f\in(\phi^\#)^{-1}(\mathfrak{m}_x)$ since the contraction of a maximal ideal is maximal by the first part of the proof. So $\mathfrak{m}_{\phi(x)}\subset(\phi^\#)^{-1}(\mathfrak{m}_x)$. Equality follows since $\mathfrak{m}_{\phi(x)}$ is maximal. \todo{first part}
\end{proof}

\begin{corollary}
	We have a pair contravariant functors
	\begin{equation*}
		\left\{\parbox{7em}{\centering{compact metric spaces}}\right\} \stackrel[\text{Spec}_m(-)]{\mathcal{O}_(-)}{\rightleftarrows} \left\{\parbox{7em}{\centering{commutative $\mathbb{R}$-algebras}}\right\} 
	\end{equation*}
	and $\text{Spec}_m(-) \circ \mathcal{O}_{(-)}$ is naturally equivalent to the identity on the category of compact metric spaces.
\end{corollary}

\begin{remark}
	It is not true that this is an equivalence of categories. \todo{why?}
\end{remark}

% motivating example: the Gelfand-Kolmogorov theorem }}}1

\section{affine algebraic varieties} % {{{1 

Just as one may begin to consider manifolds as embedded in Euclidean space, we will begin to consider varieties within affine space.

In the following, fix the field $k$ to be an algebraically closed field.

\begin{definition}
	\emph{Affine $n$-space} (over $k$) is, as a set, $\mathbb{A}^n = k^n$. We will write its elements as $\underline{a}=(a_1,\dots, a_n)$ where $a_i\in k$.
\end{definition}

Observe that every polynomial $f\in k[X_1,\dots, X_n]$ determines a function $\hat{f}:\mathbb{A}^n\to k$ on affine space by evaluation:
\begin{equation*}
	\hat{f}(a_1,\dots, a_n) = f(a_1,\dots, a_n).
\end{equation*}
It is not immediately obvious that $\hat{f}=0$ implies $f=0$. For instance, consider the finite field $\mathbb{F}_p$ of $p$ elements. In this field we have the identity $x^p=x$ for all elements $x$. In particular, the polynomial $X^p-X$ vanishes on all of $\mathbb{F}_p$, even though this is not 0 in the polynomial ring $\mathbb{F}_p[X]$. However, this situation is impossible under our assumptions:

\begin{proposition}
	The map 
	\begin{gather*}
		k[X_1,\dots, X_n] \to \Hom(\mathbb{A}^n, k) \\
		f\mapsto \hat{f}
	\end{gather*}
	is injective. \footnote{this proposition is true for any infinite field}
\end{proposition}
\begin{proof}
	We induct on $n$.

	For $n=1$, consider a nonzero $f\in k[X]$. This has only finitely many roots, and since $k$ is infinite it must be that $\hat{f}\neq 0$. 

	For $n>1$. consider a nonzero $f\in k[X_1,\dots, X_n]$. We may regard this as a polynomial in $k[X_1,\dots, X_{n-1}][X_n]$. Now if $\text{deg}_{X_n}(f)=0$, then $f\in k[X_1,\dots,X_{n-1}]$ which is already handled by the induction hypothesis. So suppose $d=\text{deg}_{X_n}(f)>0$. Then 
	\begin{equation*}
		f = \sum_{j=0}^d f_j X_n^j
	\end{equation*}
	for some $f_j\in k[X_1,\dots, X_n]$ with $f_d\neq 0$. By induction, $\hat{f_d}\neq 0$ so there exists $(a_1,\dots, a_{n-1})\in\mathbb{A}^{n-1}$ such that $f_d(a_1,\dots,a_{n-1})\neq 0$. Let 
	\begin{equation*}
		g = f(a_1,\dots,a_{n-1},X_n) = \sum_{j=1}^d f_j(a_1,\dots,a_{n-1})X_n^j \in k[X_n].
	\end{equation*}
	Then $g\neq 0$. By the $n=1$ case there exists $b\in k$ such that 
	\begin{equation*}
		0 \neq g(b) = f(a_1,\dots,a_{n-1},b)=\hat{f}(a_1,\dots,a_{n-1},b)
	\end{equation*}
	which shows $\hat{f}\neq 0$. 
\end{proof}

\begin{definition}
	The \emph{coordinate ring} of $\mathbb{A}^n$ is 
	\begin{equation*}
		\Gamma(\mathbb{A}^n) = k[X_1,\dots,X_n].
	\end{equation*}
	The \emph{function field} of $\mathbb{A}^n$ is the fraction field of its coordinate ring, i.e. 
	\begin{equation*}
		K(\mathbb{A}^n) = \text{Frac}(\Gamma(\mathbb{A}^n)) = k(X_1,\dots, X_n).
	\end{equation*}
\end{definition}

\begin{definition}
	For a subset $E\subset\Gamma(\mathbb{A}^n)$, we define its \emph{vanishing locus} as 
	\begin{equation*}
		\mathbf{V}(E) \coloneqq \{ \underline{a}\in\mathbb{A}^n : f(\underline{a})=0 \text{ for all } f\in E \}.
	\end{equation*}
	Conversely, any subset of $\mathbb{A}^n$ that is a vanishing locus as above is called an \emph{affine algebraic set}.
\end{definition}

\begin{definition}
	Given a subset $S\subset\mathbb{A}^n$, define its \emph{vanishing ideal} to be 
	\begin{equation*}
		\mathbf{I}(S) \coloneqq \{f \in \Gamma(\mathbb{A}^n) : f|_S=0 \}.
	\end{equation*}
\end{definition}

\begin{proposition}
	$\mathbf{I}(S)$ is an ideal.
\end{proposition}
\begin{proof}
	Suppose $f,g\in\mathbf{I}(S)$. Then for any $s\in S$, we have $(f+g)(s)=f(s)+g(s)=0$ hence $f+g\in \mathbf{I}(S)$. Now let $h\in\Gamma(\mathbb{A}^n)$. Then $fg(s)=f(s)g(s)=0$ so $fg\in\mathbf{I}(S)$.
\end{proof}

\begin{proposition}
	The pair 
	\begin{equation*}
		\left\{\parbox{5em}{\centering{subsets of $\mathbb{A}^n$}}\right\} \stackrel[\mathbf{V}(-)]{\mathbf{I}(-)}{\rightleftarrows} \left\{\parbox{5em}{\centering{ideals in $\Gamma(\mathbb{A}^n)$}}\right\} 
	\end{equation*}
	forms an antitone Galois connection.
\end{proposition}
\begin{proof}
	\todo{prove}
\end{proof}

\begin{corollary}
\label{cor_vanishing_locus_set_same_gend_ideal}
	$\mathbf{V}(E) = \mathbf{V}(\langle E\rangle)$, where $\langle E\rangle$ is the ideal generated by $E$.
\end{corollary}

Theorem \todo{ref} gives us an inverse bijection on the images of these functions, but we still need to determine what those images are. Based on Remark \todo{ref}, we want $\im(\mathbf{V}(-))\subset \mathbb{A}^n$ to be like ``closed sets'' and $\im(\mathbf{I}(-))$ to be like ``closed ideals''. One direction is not so difficult:

\begin{proposition}
	$\im(\mathbf{V}(-))$ is the set of affine algebraic sets.
\end{proposition}
\begin{proof}
	It is clear by definition that everything in $\im(\mathbf{V}(-))$ is an affine algebraic set. It remains to show that every affine algebraic set has this form. Well by definition every affine algebraic set has the form $\mathbf{V}(E)$ for some subset $E\subset\Gamma(\mathbb{A}^n)$. By Corollary \ref{cor_vanishing_locus_set_same_gend_ideal}, this is the same as $\mathbf{V}(\langle E\rangle)$, and we are done.
\end{proof}

If we consider $\mathbf{I}(-)$, the situation is more subtle, and is essentially the content of Hilbert's Nullstellensatz. For example, we can imagine the following situation to believe that the image is not all ideals in $\Gamma(\mathbb{A}^n)$:

\begin{example}
\label{ex_fn_in_a_implies_f_in_iva}
	Consider a nonzero $f\in \mathfrak{a}$. Consider the ideal $\mathfrak{a} = \langle f^N\rangle$ for some $N>1$. By construction it is an ideal in $\Gamma(\mathbb{A}^n)$. 

	Suppose $\underline{a}\in \mathbf{V}(\mathfrak{a})$. Then
	\begin{equation*}
		0=f^N(\underline{a})=f(\underline{a})^N,
	\end{equation*}
	implying $f(\underline{a})$ by our assumptions on $k$ (in particular it is a field, hence has no zero divisors). Since this holds for all $\underline{a}\in\mathbf{V}(\mathfrak{a})$, we see that $f\in\mathbf{I}(\mathbf{V}(\mathfrak{a}))$.

	However, the point of our construction was that $f$ is not necessarily in $\mathfrak{a}$, while $f^N$ is. In other words, $f\in\sqrt{a}$. This suggests we should be considering the radical ideals instead of all ideals.
\end{example}

\begin{corollary}
\label{cor_iv_is_radical}
	If $V\subset\mathbb{A}^n$ is an algebraic subset, then $\mathbf{I}(V)$ is radical, i.e. $\mathbf{I}(V)=\sqrt{\mathbf{I}(V)}$. 
\end{corollary}

Now observe that a function $\frac{f}{g}\in K(\mathbb{A}^n)$ defines a function 
\begin{equation*}
	\mathbb{A}^n - \mathbf{V}(\{g\}) \to k.
\end{equation*}

\begin{definition}
	An element of $K(\mathbb{A}^n)$ defined on all of $\mathbb{A}^n$ is called a \emph{regular function}.
\end{definition}

By the above observation, a function $\frac{f}{g}$ is regular if $g$ is nowhere-vanishing. \todo{conversely?}

\begin{definition}
	Let $V\subset\mathbb{A}^n$ be an algebraic set. A function $f:V\to k$ is called \emph{regular} if it is the restriction of a regular function on $K(\mathbb{A}^n)$. 
\end{definition}

Eventually we will want to define regular functions independently of the ambient affine space. For the moment, we have a surjection 
\begin{equation*}
	\Gamma(\mathbb{A}^n) \overset{\text{res}}{\longrightarrow} \Gamma(V)
\end{equation*}
whose kernel consists of regular functions vanishing on $V$. 

\begin{corollary}
\label{cor_gamma_v_quotient}
	\begin{equation*}
		\Gamma(V) \cong \frac{\Gamma(\mathbb{A}^n)}{\mathbf{I}(V)} \cong \frac{k[X_1,\dots, X_n]}{\mathbf{I}(V)}.
	\end{equation*}
\end{corollary}

\begin{example}
	Consider the algebraic set $V = \mathbf{V}(Y-X^2)\subset\mathbb{A}^2$, where $(Y-X^2)\subset \Gamma(\mathbb{A}^2)=k[X,Y]$ is an ideal. We will show that the geometric correspondence in affine space 
	\begin{equation*}
		\mathbb{A}^1 \stackrel[\phi]{\psi}{\rightleftarrows} V\subset\mathbb{A}^2
	\end{equation*}
	corresponds to the algebraic correspondence of rings 
	\begin{equation*}
		\Gamma(\mathbb{A}^1) \stackrel[\psi^\#]{\phi^\#}{\rightleftarrows} \Gamma(V).
	\end{equation*}

	First consider the situation on rings. By Corollary \ref{cor_gamma_v_quotient}, we know 
	\begin{equation*}
		\Gamma(V) = \frac{k[X,Y]}{(Y-X^2)}.
	\end{equation*}
	As a ring, this is generated by $\{1, X\}$, since $k[X,Y]$ is generated by $\{1,X,Y\}$ and in the quotient ring $Y-X^2=0$, i.e. $Y=X^2$. This provides an explicit isomorphism 
	\begin{gather*}
		\frac{k[X,Y]}{(Y-X^2)} \to k[T] \\
		X \mapsto T.
	\end{gather*}

	On the level of affine space, consider the maps 

	\begin{centering}
		\begin{tikzpicture}
			\draw [thick] (-3, 0) -- (3, 0);
			\draw [thick] (0, -1) -- (0, 5);
			\node[] at (0.5, -0.5) {$\mathbb{A}^2$};
			\draw (0,0) parabola (2,4);
			\draw (0,0) parabola (-2, 4);

			\node[] at (0.5,-3.5) {$\mathbb{A}^1$};
			\draw [thick] (-3, -3) -- (3, -3);

			\draw[->] (4, -0.5)--(4, -2.5) node[midway,right]{$\phi(x,y)=x$};
			\draw[->] (-4, -2.5)--(-4, -0.5) node[midway,left]{$\psi(t)=(t,t^2)$};

			\path (-8,0) -- (8,0);
		\end{tikzpicture}
	\end{centering}

	One verifies that this determines a bijection $\mathbb{A}^1\to V$.

	How do we connect these two correspondences? Consider the map 
	\begin{gather*}
		\phi^\#:\Gamma(\mathbb{A}^1) \to \Gamma(V) \\
		T \mapsto T\circ \phi. 
	\end{gather*}
	Intuitively, this sends the polynomial $T$ to the polynomial which picks out the first coordinate, i.e. the polynomial $X$. Now consider the map 
	\begin{gather*}
		\psi^\#: \Gamma(V) \to \Gamma(\mathbb{A}^1) \\
		\bar{f} \mapsto \bar{f}\circ \psi.
	\end{gather*}
	In particular, this sends the polynomial $Y$ (which picks out the second coordinate) to $T^2$.

	If we regard $\{1,X,Y\}$ as generators of $k[X,Y]=\Gamma(\mathbb{A}^2)$, then the correspondence on the level of rings is given by $X\mapsto T$ and $Y\mapsto T^2$.
\end{example}

Which rings arise as $\Gamma(V)$ for some algebraic set $V\subset\mathbb{A}^n$? For one, we know every $\Gamma(V)$ must hhave the following form:
\begin{equation*}
	\Gamma(V) = \frac{\Gamma(\mathbb{A}^n)}{I(V)} = \frac{k[X_1,\dots,X_n]}{I(V)},
\end{equation*}
which is in particular a finitely-generated $k$-algebra. From Corollary \ref{cor_iv_is_radical}, we know that $\mathbf{I}(V)=\sqrt{\mathbf{I}(V)}$. So in particular if $f\in \Gamma(V)$ is such that $f^N=0$ on $V$, i.e. $f^N\in \mathfrak{I}(V)$, then $f=0$, i.e. $f\in\mathfrak{I}(V)$. Since polynomial rings over a field are reduced, it then follows that $\Gamma(V)$ is reduced \footnote{i.e. it has no nilpotent elements}. To summarize:
\begin{itemize}
	\item $\Gamma(V)$ is finitely generated.
	\item $\Gamma(V)$ is reduced.
\end{itemize}
The big question is: if some $k$-algebra has these two properties, is it of the form $\Gamma(V)$?

\begin{definition}
	An \emph{affine $k$-algebra} is a finitely generated reduced $k$-algebra, i.e. 
	\begin{equation*}
		A \cong \frac{k[X_1,\dots,X_n]}{I}
	\end{equation*}
	for some radical ideal ideal $I\subset k[X_1,\dots, X_n]$.
\end{definition}

\begin{definition}
	Let $X$ be a topological space. We call a nonempty subset of $X$ \emph{irreducible} if $X$ cannot be written as a union of two proper closed subsets.
\end{definition}

\begin{remark}
	Note that spaces that have such a property are quite pathological!
\end{remark}

\begin{proposition}
\label{prop_y_irred_closed_then_iy_prime}
	An algebraic set $Y\subset\mathbb{A}^n$ is irreducible if and only if $\mathbf{I}(Y)$ is prime.
\end{proposition}
\begin{proof}
	First we will do the forward direction. Suppose $Y$ is irreducible. Suppose $fg\in \mathbf{I}(Y)$. We want to show that either $f\in \mathbf{I}(Y)$ or $g\in\mathbf{I}(Y)$. First, $fg\in\mathbf{I}(Y)$ implies $(fg)\subset\mathbf{I}(Y)$, and so $\mathbf{V}(\mathbf{I}(Y))\subset\mathbf{V}(fg)=\mathbf{V}(f)\cup\mathbf{V}(g)$. Since $Y\subset\mathbf{V}(\mathbf{I}(Y))$, we get 
	\begin{equation*}
		Y=Y\cap (\mathbf{V}(f)\cup\mathbf{V}(g))=(Y\cap\mathbf{V}(f))\cap(Y\cap \mathbf{V}(g))
	\end{equation*}
	Now by the definition of the Zariski topology, $\mathbf{V}(f)$ and $\mathbf{V}(g)$ are closed. Since $Y$ is an algebraic set, it is also closed. Hence the above expresses $Y$ as the union of two closed sets. By our assumption that $Y$ is irreducible, either:
	\begin{itemize}
		\item $Y=Y\cap\mathbf{V}(f)$, which implies $Y\subset\mathbf{V}(f)$ so $f|_Y=0$ so $f\in I(Y)$.
		\item analagous.
	\end{itemize}
	This proves the forward direction.

	For the reverse direction, suppose $\mathbf{I}(Y)=\mathfrak{p}$ is prime. We need to show $Y$ is irreducible. Since $Y$ is closed, by (\todo{ref}) we know $Y=\mathbf{V}(\mathbf{I}(Y))=\mathbf{V}(\mathfrak{p})$. Suppose that we can write $Y=Y_1\cup Y_2$, where $Y_1$ and $Y_2$ are closed in $Y$ (hence in $\mathbb{A}^n$). Then 
	\begin{equation*}
		\mathfrak{p}=\mathbf{I}(Y)=\mathbf{I}(Y_1\cup Y_2)=\mathbf{I}(Y_1)\cap\mathbf{I}(Y_2).
	\end{equation*}
	But $\mathfrak{p}$ is prime, so by (\todo{ref}) $\mathfrak{p}=\mathbf{I}(Y_1)$ or $\mathfrak{p}=\mathbf{I}(Y_2)$. But then either $Y=\mathbf{V}(\mathfrak{p})=\mathbf{V}(I(Y_1))=Y_1$ or $Y=\mathbf{V}(\mathfrak{p})=\mathbf{V}(I(Y_2))=Y_2$.
\end{proof}

\begin{definition}
	An \emph{affine algebraic variety} is an irreducible affine algebraic set.
\end{definition}

% affine algebraic varieties }}}1

\section{Hilbert basis theorem} % {{{1 

\begin{proposition}
\label{prop_acc_equiv_max}
	Let $(P,\leq)$ be a poset. Then the following are equivalent:
	\begin{itemize}
		\item (ascending chain condition, ACC) Every increasing sequence
			\begin{equation*}
				x_1 \leq x_2 \leq x_3 \leq \cdots 
			\end{equation*}
			stabilizes, i.e. there exists $N\in\mathbb{N}$ such that $x_N=x_{N+1}=\cdots$.
		\item Every nonempty subset of $P$ has a maximal element.
	\end{itemize}
\end{proposition}
\begin{proof}
	($1\Rightarrow 2$) We prove the contrapositive, so suppose there exists a nonempty subset $\Sigma\subset P$ which does not have a maximal element. Let $x_1\in\Sigma$. Then there exists $x_2\in\Sigma$ such that $x_1<x_2$ strictly. Repeating this procedure produces a sequence which does not stabilize.

	($2\Rightarrow 1$) Let $x_1\leq x_2\leq \cdots$ be an ascending chain. By assumption, the set $\{x_i\}_i$ has a maximal element, call it $x_N$. But then it must be that $x_N=x_{N+1}=\cdots$, i.e. the chain stabilizes. 
\end{proof}

We will want to apply Proposition \ref{prop_acc_equiv_max} to the situation where we have an $A$-module $M$ and our partially ordered set is the submodules of $M$ with ordering coming from inclusion.

\begin{definition}
	An $A$-module $M$ is called \emph{Noetherian} if it satisfies the ascending chain condition (equivelently, every collection of submodules of $M$ has a maximal element). A ring is called Noetherian if it is Noetherian as a module over itself.
\end{definition}

\begin{proposition}
	The following are equivalent:
	\begin{enumerate}
		\item $M$ is Noetherian.
		\item All submodules of $M$ are finitely-generated.
	\end{enumerate}
\end{proposition}
\begin{proof}
	($1\Rightarrow 2$) Suppose $M$ is Noetherian. Let $N\subset M$ be a submodule. Let $\Sigma$ be the set of all finitely generated submodules of $N$. Then $\Sigma$ is nonempty, since for instance it contains a cyclic submodule \footnote{i.e. a submodule generated by a single element (which lies in $N$)}. Since every submodule of $N$ is a submodule of $M$, it follows by the assumption on $M$ that $\Sigma$ has a maximal element $\bar{N}$. We claim that $\bar{N}=N$, which will show that $N$ is finitely generated. Suppose otherwise. Then there exists $x\in N-\bar{N}$. But then $\bar{N}+\langle x\rangle$ is finitely generated, violating the maximality of $\bar{N}$. Hence $N=\bar{N}$, and $N$ is finitely generated.

	($2\Rightarrow 1$) Now suppose that submodules of $M$ are finitely-generated. Let $M_1\subset M_2\subset\cdots$ be an ascending chain of submodules. Let 
	\begin{equation*}
		M_\infty = \bigcup_{n\geq 0}M_n.
	\end{equation*}
	Then $M_\infty$ is finitely-generated, since it is also a submodule. Call its set of generators $\{x_1,\dots,x_n\}$. Since each $x_i$ must be in some $M_{k_i}$, let $k=\max\{k_1,\dots,k_n\}$. But then $\{x_1,\dots,x_n\}\subset M_k$, which shows that $M_\infty\subset M_k\subset M_\infty$. Hence $M_k=M_{k+1}=\cdots$ which shows the chain stabilizes, hence $M$ is Noetherian.
\end{proof}

\begin{example}
	$\mathbb{Z}$ is Noetherian as a module over itself. Submodules of a ring regarded as a module over itself are just its ideals, and every ideal in $\mathbb{Z}$ is principal.
\end{example}

\begin{example}
	$\mathbb{Z}_{p^\infty}$, the $p$-primary part of $\mathbb{Q}/\mathbb{Z}$, is not Noetherian.
\end{example}

\begin{proposition}
	If 
	\begin{equation*}
		0 \to M' \to M \to M'' \to 0
	\end{equation*}
	is exact, then $M$ is Noetherian if and only if $M'$ and $M''$ are Noetherian.
\end{proposition} 
\begin{proof}
	HW \todo{hw}
\end{proof}

\begin{theorem}[Hilbert basis theorem]
	If $A$ is a Noetherian ring, then $A[X]$ is also Noetherian.
\end{theorem}
\begin{proof}
	For an ideal $E \subset A[X]$, define $C_nI\subset A$ to be the set of coefficents of $X^n$ for functions $f\in I$ of degree $\leq n$. In other words, it is the set of leading coefficents for degree $n$ polynomials in $I$ along with $0$.

	\begin{lemma}
		$C_nI\subset A$ is an ideal.
	\end{lemma}
	\begin{proof}
		Exercise.
	\end{proof}

	\begin{lemma}
		$C_0I\subset C_1I\subset C_2I\subset\cdots$ for any ideal $I\subset A[X]$.
	\end{lemma}
	\begin{proof}
		Exercise.
	\end{proof}
	
	\begin{lemma}
		Let $I,J\subset A[X]$ be ideals, with $I\subset J$. Then 
		\begin{enumerate}
			\item $C_iI\subset C_iJ$ for all $i$.
			\item If $C_iI=C_iJ$ for all $i$, then $I=J$.
		\end{enumerate}
	\end{lemma}
	\begin{proof}
		The first statement is clear after unwrapping definitions. For the second, suppose $C_iI=C_jJ$ for all $i$. Let $f\in J$. In light of the first statement, it suffices to show $f\in I$. We will do this by inducting on $\text{deg}(f)=n$.

		For $n=0$, we know $f\in I$ because then $f$ is a constant, so its leading coefficient is itself. Since $C_iI=C_iJ$, we know $f\in C_0I$. So then there must be a degree 0 polynomial $g\in I$ whose leading coefficient is $f$. But there is only one, namely $f$, i.e. $f=g$, so $f\in I$. 

		In the general case, let 
		\begin{equation*}
			f = a_n X^n + \cdots + a_1X + a_0 \in J,
		\end{equation*}
		with $a_n\neq 0$ (if $a_n=0$ then we are done by induction). This means $a_n\in C_nJ=C_nI$, and so there exists $g\in I$ of the form 
		\begin{equation*}
			g = a_nX^n + b_{n-1}X^{n-1} + \cdots + b_1X + b_0 \in I \subset J.
		\end{equation*}
		Then $f-g\in J$, and since $\text{deg}(f-g)<n$ strictly we have by induction that $f-g\in I$. But then $f=(f-g)+g\in I$ too.
	\end{proof}

	Let $I_0\subset I_1\subset\cdots$ be an ascending chain of ideals in $A[X]$. Consider the diagram 
	\begin{equation*}
		% https://tikzcd.yichuanshen.de/#N4Igdg9gJgpgziAXAbVABwnAlgFyxMJZABgBoBmAXVJADcBDAGwFcYkQBJAfWJAF9S6TLnyEUZAEzU6TVu24BGfoJAZseAkTILpDFm0ScuE5UPWitpYrtkGQAHXu0oEHAgFmRmlAoo397ADCPNy8HqrCGmLIvlI0enKGwcSKphHm3jGkOvG2QSHGaWpe0b7WuQGGjs6u7irFUUQSfhWJIMEKoUWRFijNcTKV7VydI90Z0c05g20d3CbhDb3IzeUzdtUubuMlROTZ-m2OAMZbdZ6NKPsDCRv2p7U7l8j7VK13D9t80jBQAObwIigABmACcIABbJBkEA4CBIXyw+hYRjsAAWEAgAGs0mDIQiaHCkM0kSj0ZiceE8VDECSiYh9qTUYYMdjceCaQAWQnwxAAVkJyOZIFZlJU1KQAthvIAbIKySyKez8Yg5dKkAB2eXC0XKmlq+kKGE4IXktlUjmSnlIACc2rNYpBlsQ3PViAAHPbFebxc6jdbEApESaFSKlRaVf63Qo6abvY6QBLEHbo4yQzrw76VZ7o670w69Ugc-SU-n44XkwGo2Ww+bKHwgA
\begin{tikzcd}
\vdots              & \vdots                                 & \vdots                                 &        \\
I_2 \arrow[u, hook] & C_0I_2 \arrow[u, hook] \arrow[r, hook] & C_1I_2 \arrow[u, hook] \arrow[r, hook] & \cdots \\
I_1 \arrow[u, hook] & C_0I_1 \arrow[u, hook] \arrow[r, hook] & C_1I_1 \arrow[r, hook] \arrow[u, hook] & \cdots \\
I_0 \arrow[u, hook] & C_0I_0 \arrow[u, hook] \arrow[r, hook] & C_1I_0 \arrow[r, hook] \arrow[u, hook] & \cdots
\end{tikzcd}
	\end{equation*}
	If we can show that the square on the right stabilizes ``uniformly'' at a certain height, then so will $I_0\subset I_1\subset\cdots$ by the previous lemma. We can stabilize each column individually, but we need to show that there is a single $N>0$ after which all columns stabilize.

	Consider the chain formed on the diagonal, i.e. 
	\begin{equation*}
		C_0I_0\subset C_1I_1\subset\cdots
	\end{equation*}
	This is an ascending chain in $A$ by the above lemma, hence stabilizes at some index $k$. But this means all arrows in above and right of $(k,k)$ are equalities, by the principle that $A\subset B\subset A$ implies $A=B$:
	\begin{equation*}
		% https://tikzcd.yichuanshen.de/#N4Igdg9gJgpgziAXAbVABwnAlgFyxMJZABgBoAmAXVJADcBDAGwFcYkQBhAfQGsBJXiAC+pdJlz5CKMgEZqdJq3bd+XYDwDUMocNEgM2PASJli8hizaIQAHRu0oEHAhFjDkojIrnFVzms1tAR5dNwljFC85Ggsla251LSEBRO1Q-XEjKWQvMxjfdjsHJxc9A3Ds8lJohUtCmwBjR2d08qyiKqp8uus7JpLheRgoAHN4IlAAMwAnCABbJDIQHAgkL2X6LEZ2AAsICBDXEBn5tZoVpCqNrd39w70ThcR1i8QAFnPN7es9g-THpAfZarRAAVk+Nx+d3+syeQNeADYId8QL97lNYUgAMznEEAdmRtz+RwBiBxwMBhKhxIemMQS1e5JwXyJ6OOdIZIKBjCwYD8jmYACNGGwaDsYPQoOxIHyQFTwAQ2EJKEIgA
\begin{tikzcd}
\vdots                                                                 & \vdots                                         &        \\
C_kI_{k+1} \arrow[u, hook] \arrow[r, hook]                             & C_{k+1}I_{k+1} \arrow[u, hook] \arrow[r, hook] & \cdots \\
C_kI_k \arrow[u, hook] \arrow[r, hook] \arrow[ru, Rightarrow, no head] & C_{k+1}I_k \arrow[r, hook] \arrow[u, hook]     & \cdots
\end{tikzcd}
	\end{equation*}
	Now we only need to uniformly stabilize the first $k$ columns. Since they each individually stabilize, and there's finitely many of them, we can just take the max of the indices after which they stabilize.
\end{proof}

% Hilbert basis theorem }}}1

\section{Nullstellensatz} % {{{1 

For now let $k$ be a field. Eventually we will require it to be algebraically closed.

\begin{example}
	Recall that $k[X]$ is the free $k$-algebra on the one-element set $\{X\}$ (as an algebra it is generated by one element, but as a module it is generated by countably infinite many). Consider the diagram 
	\begin{equation*}
		% https://tikzcd.yichuanshen.de/#N4Igdg9gJgpgziAXAbVABwnAlgFyxMJZABgBpiBdUkANwEMAbAVxiRAB13gANTgXxB9S6TLnyEUARnJVajFmwDWybhUHCQGbHgJFpk2fWatEIAIKDZMKAHN4RUADMAThAC2SMiBwQk0ucZsWOpOrh6IXj5IAEzURgqmnIxoABZ0INQMdABGMAwACqI6EiDOWDYpOCEgLu5+1FGIsQEJHOwwaNgMBAD6SQyp6XwUfEA
\begin{tikzcd}
\{X\} \arrow[r, "i"] \arrow[rd, "\alpha"'] & {k[X]} \arrow[d, "\epsilon_\alpha"] \\
                                           & A                                  
\end{tikzcd}
	\end{equation*}
	where $A$ is any $k$-algebra. Here $\alpha$ is any function. Since it is a function out of a one-element set, it amounts to picking out an element in $A$. To make this diagram commute, $\epsilon_\alpha$ must send $X\mapsto \alpha$, which on a polynomial amounts to evaluating that polynomial at the point $\alpha$.

	Now suppose we take $A=K$ to be monogenic, i.e. generated by one elment $\alpha$. There are two cases. 
	\begin{enumerate}
		\item $\epsilon_\alpha$ is injective. Then since it also maps to the generater of $K$, it's surjective, hence $K\cong k[X]$. But then $K$ is not a field, and in particular is not an algebraic extension of $k[X]$. We may think of this as the case where $\alpha$ is transcendental.
		\item $\epsilon_\alpha$ is not injective. Then $\alpha$ is the root of some polynomial, hence ``by definition'' $K=k(\alpha)$ is a finite algebraic extension of $k$.
	\end{enumerate}
\end{example}

Zariski's lemma can be seen as a multivariable analogue of this property:

\begin{theorem}[Zariski's lemma]
	Let $K$ be a finitely generated $k$-algebra. If $K$ is a field, then it is a finite algebraic extension of $k$.
\end{theorem}

We now assume $k$ is algebraically closed.

\begin{theorem}[weak Nullstellensatz]
	Let $I\subsetneq k[X_1,\dots,X_n]$ be a proper ideal. Then $\mathbf{V}(I)\neq\emptyset$, i.e. there exists at least one point in $\mathbb{A}^n_k$ which on which all of $I$ vanishes.
\end{theorem}
\begin{proof}
	The idea is to reduce to considering maximal ideals. The quotient of these with the polynomial ring is a field, hence a finite algebraic extension by Zariski's lemma. Since we algebraically closed, this field must be $k$ itself. We then pull back each inderminate to $k$ to find a point $\underline{a}$ such that $\mathfrak{m}_{\underline{a}}\subset\mathfrak{m}$, hence the two are equal, hence $\underline{a}\in\mathbf{V}(\mathfrak{m})$.

	Since $I$ is proper, it's contained in a maximal ideal $\mathfrak{m}$. Then $\mathbf{V}(\mathfrak{m})\subset\mathbf{V}(I)$, so it suffices to consider maximal ideals.

	Let $K\coloneqq k[X_1,\dots,X_n]/\mathfrak{m}$. Consider the composite 
	\begin{equation*}
		% https://tikzcd.yichuanshen.de/#N4Igdg9gJgpgziAXAbVABwnAlgFyxMJZABgBpiBdUkANwEMAbAVxiRAGsQBfU9TXfIRQBGclVqMWbdsgAaAfVEAdJVAg44pBWArdeIDNjwEio4ePrNWiEAGlu4mFADm8IqABmAJwgBbJGQgOBBIAEzUDFhg1iBQdHAAFk4g1JZSNipoCVgpIAx0AEYwDAAK-MZCIF5Yzgk4ep4+-oiBwUiiElZsOTyNfu3UbYjhnekgmT0UXEA
\begin{tikzcd}
k \arrow[rd, "\phi"', dashed] \arrow[r, "i"] & {k[X_1,\dots,X_n]} \arrow[d, "\pi"] \\
                                             & K                                  
\end{tikzcd}
	\end{equation*}
	By Zariski's lemma, $K$ is a finite algebraic extension of $k$. Since $k$ is algebraiclly closed, $K\cong k$ and in particular $\phi$ is an isomorphism. 

	Let $a_i=\phi^{-1}(\bar{X_i})$. This means $a_i + \mathfrak{m} = X_i + \mathfrak{m}$, i.e. $X_i - a_i\in\mathfrak{m}$ for all $i$. The the maximal ideal $m_{\underline{a}}=(X_1-a_1,\dots,X_n-a_n)$ is contained in $\mathfrak{m}$, hence $\mathfrak{m}_{\underline{a}}=\mathfrak{m}$. But certainly $\underline{a}\in\mathbf{V}(\mathfrak{m}_{\underline{a}})$, so we're done. 
\end{proof}

\begin{corollary}
	If $k$ is algebraically closed, then \textit{all} maximal ideals are of the form $\mathfrak{m}_{\underline{a}}=(X_1-a_1,\dots,X_n-a_n)$.
\end{corollary}

\begin{theorem}[strong Nullstellensatz]
	Let $\mathfrak{a}\subset k[X_1,\dots,X_n]$. Then $\mathbf{I}(\mathbf{V}(\mathfrak{a}))=\sqrt{\mathfrak{a}}$.
\end{theorem}
\begin{proof}
	($\supset$) Follows by Corollary \ref{cor_iv_is_radical}. 

	($\subset$) Let $h\in\mathbf{I}(\mathbf{V}(\mathfrak{a}))$ be nonzero. We want to show $h\in\sqrt{\mathfrak{a}}$. To that end, consider $\tilde{\mathfrak{a}}\in k[X_1,\dots,X_n][Y]$ given by 
	\begin{equation*}
		\tilde{\mathfrak{a}} \coloneqq \langle \mathfrak{a}\cup\{1-hY\}\rangle\subset k[X_1,\dots,X_n][Y],
	\end{equation*}
	i.e. the ideal generated by $\mathfrak{a}$ and $1-hY$. 

	We claim $\mathbf{V}(\tilde{\mathfrak{a}}))\subset\mathbb{A}^{n+1}_k$ is empty. Suppose otherwise, i.e. there exists an element $(a_1,\dots,a_n,b)\in\mathbf{V}(\tilde{\mathfrak{a}})$. Then since $\mathfrak{a}\subset\tilde{\mathfrak{a}}$, we have that $\tilde{f}(a_1,\dots,a_n,b)=f(a_1,\dots,a_n,b)=0$ for any $f\in\mathfrak{a}$. But then $(a_1,\dots,a_n)\in\mathbf{V}(\mathfrak{a})$. But then $0 = (1-hY)(a_1,\dots,a_n,b) = 1 - h(a_1,\dots,a_n)b=1$, a contradiction. This proves the claim.

	Now, by the weak Nullstellensatz we know that $\tilde{\mathfrak{a}} = k[X_1,\dots,X_n][Y]$, and in particular $1\in\tilde{\mathfrak{a}}$. Then using the fact that $\tilde{\mathfrak{a}}=\langle\mathfrak{a}\cup\{1-hY\}\rangle$, we can write 
	\begin{equation*}
		1 = \sum_{i=1}^r F_i(X_1,\dots,X_n,Y)g_i(X_1,\dots,X_n) + F(X_1,\dots,X_n)(1-h(X_1,\dots,X_n)Y)
	\end{equation*}
	for some $g_i\in\mathfrak{a}$ and some $F_i\in k[X_1,\dots,X_n][Y]$. Consider the Laurent ring $k[X_1,\dots,X_n][Y,1/Y]$ and the evaluation map 
	\begin{gather*}
		\epsilon: k[X_1,\dots,X_n][Y] \to k[X_1,\dots,X_n][Y,1/Y] \\
		Y \mapsto \frac{1}{Y}.
	\end{gather*}
	In this new ring, 
	\begin{equation*}
		1 = \sum_{i=1}^r F_i(X_1,\dots,X_n,\frac{1}{Y})g_i(X_1,\dots,X_n) + F(X_1,\dots,X_n,\frac{1}{Y})(1-h(X_1,\dots,X_n)\frac{1}{Y}).
	\end{equation*}
	We may multiply both sides by $Y^N$ for some large enough $N$ such that 
	\begin{equation*}
		Y_N = \sum_{i=1}^r G_i(X_1,\dots,X_n,Y)g_i(X_1,\dots,X_n) + G(X_1,\dots,X_n,Y)(Y-h(X_1,\dots,X_n))
	\end{equation*}
	for some $G_i \in k[X_1,\dots,X_n,Y]$.

	Now consider the evaluation map 
	\begin{gather*}
		\eta: k[X_1,\dots,X_n][Y] \to k[X_1,\dots,X_n] \\
		Y \mapsto h.
	\end{gather*}
	Then the above becomes 
	\begin{align*}
		h^N = \sum_{i=1}^r & G_i(X_1,\dots,X_n,h(X_1,\dots,X_n))g_i(X_1,\dots,X_n) \\
		+& G(X_1,\dots,X_n,h(X_1,\dots,X_n))(0).
	\end{align*}
	Since $g_i\in\mathfrak{a}$, it follows that $h^N\in\mathfrak{a}$, so $h\in\mathfrak{a}$.
\end{proof}

\begin{corollary}
	Let $k$ be an algebraically closed field. Then there exist inverse bijections 
	\begin{equation*}
		\left\{\parbox{9em}{\centering{algebraic subsets $X\subset\mathbb{A}^n_k$}}\right\} \stackrel[\mathbf{V}]{\mathbf{I}}{\rightleftarrows} \left\{\parbox{7em}{\centering{radical ideals $\mathfrak{a}\subset\Gamma(\mathbb{A}^n_k)$}}\right\}
	\end{equation*}
\end{corollary}

\begin{corollary}
	If $\mathfrak{a},\mathfrak{b}\subset k[X_1,\dots,X_n]$ are ideals, then $\mathbf{V}(\mathfrak{a})=\mathbf{V}(\mathfrak{b})$ if and only if $\sqrt{a}=\sqrt{b}$.
\end{corollary}

% Nullstellensatz }}}1

\section{affine schemes: as spaces} % {{{1 

Our goal will be to realize a commutative ring $A$ as the ``ring of functions'' on some space (which will be $\text{Spec}(A)$).

\subsection{summary so far} % {{{2 

Let $k$ be an algebraically closed field.

\begin{enumerate}
	\item Points $\underline{a}=(a_1,\dots,a_n)\in\mathbb{A}_k^n$ correspond naturally to maximal ideals $\mathfrak{m}_{\underline{a}}=(X_1-a_1,\dots,X_n-a_n)\subset\Gamma(\mathbb{A}^n)$.
	\item A polynomial map $F=(F_1,\dots,F_n):\mathbb{A}^n\to\mathbb{A}^n$, where $F_i\in k[X_1,\dots,X_n]$, induces a $k$-algebra map 
		\begin{gather*}
			F^\#:\Gamma(\mathbb{A}^m)\to\Gamma(\mathbb{A}^n) \\
			Y_j \mapsto F_j 
		\end{gather*}
		ref \todo{ref}
	\item We can recover $F$ from $F^\#$ in the following way: $F^\#$ induces a map 
		\begin{gather*}
			\tilde{F}: \text{Spec}_m(k[X_1,\dots,X_n]) \to \text{Spec}_m(k[Y_1,\dots,Y_m]) \\
			\mathfrak{m} \mapsto (F^\#)^{-1}(\mathfrak{m}),
		\end{gather*}
		where we then identity $\text{Spec}_m(k[X_1,\dots,X_n])\cong \mathbb{A}^n$ and $\text{Spec}_m(k[Y_1,\dots,Y_m])\cong\mathbb{A}^m$.
	\item Let $\mathfrak{a}\subset A=k[X_1,\dots,X_n]$ be an ideal. Consider the bijection 
		\begin{gather*}
			\mathbb{A}^n\leftrightarrow \text{Spec}_m(k[X_1,\dots,X_n]) \\
			\underline{a} \mapsto \mathfrak{m}_{\underline{a}}.
		\end{gather*}
		Then $\underline{a}\in\mathbf{V}(\mathfrak{a})$ if and only 
		\begin{align*}
			\underline{a}\in\mathbf{V}(\mathfrak{a})
			\Leftrightarrow& \{\underline{a}\}\subset \mathbf{V}(\mathfrak{a}) \\
			\Leftrightarrow& \mathbf{I}(\{\underline{a}\})\supset\mathbf{I}(\mathbf{V}(\mathfrak{a})) \\
			\Leftrightarrow& \mathfrak{m}_{\underline{a}}\supset\sqrt{\mathfrak{a}} \\
			\Leftrightarrow& \mathfrak{m}_{\underline{a}}\supset\mathfrak{a}.
		\end{align*}
\end{enumerate}

% summary so far }}}2

\subsection{Zariski topology} % {{{2 

\begin{definition}
	Let $A$ be a ring. For a subset $E\subset A$, define 
	\begin{equation*}
		V(E) \coloneqq \{\mathfrak{p}\in\text{Spec}(A) : \mathfrak{p}\supset E \}.
	\end{equation*}
	Note $\mathfrak{p}\supset E$ iff $\mathfrak{p}$ contains the ideal generated by $E$, so we may as well consider $E$ to be an ideal above.
\end{definition}

\begin{para}
The idea now is to view an element $f\in A$ as a ``function'' on $\text{Spec}(A)$ with values in a field: the value of $f\in A$ at the point $\mathfrak{p}\in\text{Spec}(A)$ is the image of $f$ under the natural maps to the residue field: 
\begin{equation*}
	A \overset{\pi}{\longrightarrow} A/\mathfrak{p} \overset{i}{\longrightarrow} \kappa(\mathfrak{p})\coloneqq \text{Frac}(A/\mathfrak{p}).
\end{equation*}

This is a generalization of usual function evaluation: for $f\in\Gamma(\mathbb{A}^n)$, the value of $f$ at $\underline{a}$ could be considered as the image of $f$ under the map 
\begin{gather*}
	k[X_1,\dots,X_n] \to k[X_1,\dots,X_n]/\mathfrak{m}_{\underline{a}} \to \text{Frac}(k[X_1,\dots,X_n]/\mathfrak{m}_{\underline{a}}) = k \\
	f \mapsto f(\bar{X_1},\dots,\bar{X_n}) = f(a_1,\dots,a_n).
\end{gather*}
The simplification being that since $\mathfrak{m}_{\underline{a}}$ is maximal, we already have a field before we take the field of fractions (taking the field of fractions does nothing).

In this context, the notation in the above notation is justified: $V((f))$ consists of prime ideals containing $(f)$, hence $f$ will be killed in the quotient above, hence the value of $f$ at those primes is 0. More generally, for a subset $E\subset A$, note 
\begin{equation*}
	V(E) = \bigcap_{f\in E}V(f) = \bigcap_{f\in E} \{\mathfrak{p}\in\text{Spec}(A) : f\in\mathfrak{p}\} = \{\mathfrak{p}\in\text{Spec}(A): E\supset\mathfrak{p}\}
\end{equation*}
and we recover the definition. 
\end{para}

\begin{para}
The sets $V(E)$ will be the closed sets in the topology we define on $\text{Spec}(A)$. This topology will also be called the Zariski topology for the following reason: recall that $\underline{a}\in\mathbf{V}(\mathfrak{a})$ iff $\mathfrak{m}_{\underline{a}}\supset\mathfrak{a}$. Hence 
\begin{equation*}
	V(A)\cap\text{Spec}_m(\mathbb{A}^n) = \mathbf{V}(\mathfrak{a})
\end{equation*}
Hence the earlier Zariski topology is nothing but the subspace topology of the new Zariski topology on $\text{Spec}(A)$, specifically the one on the subspace $\text{Spec}_m(A)\cong\mathbb{A}^n$.
\end{para}

\begin{proposition}
	Let $A$ be a ring, let $X=\text{Spec}(A)$.
	\begin{enumerate}
		\item (order reversing) If $E_1\subset E_2\subset A$, then $V(E_1)\supset V(E_2)$.
		\item $V(\emptyset) = X$ and $V(A)=\emptyset$.
		\item $V(E)=V\langle E\rangle$
		\item For ideals $\mathfrak{a},\mathfrak{b}\subset A$, we have $V(\mathfrak{a}\cap\mathfrak{b})=V(\mathfrak{a})\cup V(\mathfrak{b})$.
		\item For ideals $\mathfrak{a}_\alpha\subset A$, we have $V(\sum_\alpha \mathfrak{a}_\alpha) = \bigcap_\alpha V(\mathfrak{a}_\alpha)$.
	\end{enumerate}
\end{proposition}

\begin{definition}
	The \emph{Zariski topology} on $X=\text{Spec}(A)$ is the topology whose closed sets are the sets $V(\mathfrak{a})$ for some ideal $\mathfrak{a}\subset A$.
\end{definition}

\begin{remark}
	Recall from (\todo{ref}) that points in $\mathbb{A}^n$ correspond to maximal ideals of $A$. The set of prime ideals include the maximal ideals; so this Zariski topology includes all the points of $\mathbb{A}^n$ but also certain ``smeared out'' points, e.g. varieties such as $\mathbf{V}(Y-X^2)$.
\end{remark}

\begin{proposition}
	The Zariski open sets 
	\begin{equation*}
		D(f) = X - V(f) = \{\mathfrak{p}\in\text{Spec}(A) : f\not\in \mathfrak{p} \},
	\end{equation*}
	called \emph{distinguished open sets}, form a basis for the Zariski topology on $X$.
\end{proposition}
\begin{proof}
	We want to show than an arbitrary open set $U\subset X$ is a union of distinguished open sets. Let $U=X-V(\mathfrak{a})$ for some $\mathfrak{a}\subset A$. Let $\{f_i\}_i$ be a set of generators of $\mathfrak{a}$ (perhaps it is the set of all elements in $\mathfrak{a}$). Then $\mathfrak{a} = \sum_i (f_i)$, implying
	\begin{equation*}
		V(\mathfrak{a}) = V(\sum_i (f_i)) = \bigcap_i V(f_i),
	\end{equation*}
	so 
	\begin{equation*}
		U = X-V(\mathfrak{a}) = X-\bigcap_i V(f_i) = \bigcup_i (X-V(f_i)) = \bigcup_i D(f_i).
	\end{equation*}
	This shows what we want.
\end{proof}

\begin{proposition}
	$D(fg) = D(f) \cap D(g)$.
\end{proposition}
\begin{proof}
	HW.
\end{proof}

\begin{proposition}
	$X=\Spec(A)$ is quasicompact\footnote{i.e. compact but not necessarily Hausdorff; apparantly Bourbaki used ``compact'' to refer to compact Hausdorff spaces}.
\end{proposition}
\begin{proof}
	It suffices to show an open cover of $X$ by basic open sets admits a finite subcover (since any open set is a union of basic open sets). Say $U=\{D(f_\alpha)\}_{\alpha\in\Lambda}$ covers $X$. Then 
	\begin{equation*}
		\emptyset = X - \bigcup_{\alpha}D(f_\alpha) = \bigcap_\alpha (X-D(f_\alpha)) = \bigcap_\alpha V(f_\alpha) = V(\sum_\alpha(f_\alpha)),
	\end{equation*}
	so $\sum_\alpha(f_\alpha) = A$. Thus there exist $\alpha_1,\dots,\alpha_n\in\Lambda$ and $c_1,\dots,c_n\in A$ such that 
	\begin{equation*}
		1 = c_1f_{\alpha_1} + \cdots + c_nf_{\alpha_n}.
	\end{equation*}
	So $A = (f_{\alpha_1}) + \cdots + (f_{\alpha_n})$. So 
	\begin{equation*}
		\emptyset = V(A) = V(\sum_{i=1}^n (f_{\alpha_i})) =\bigcap_{i=1}^n V(f_{\alpha_i}).
	\end{equation*}
	So 
	\begin{equation*}
		X = X-\emptyset = X-\bigcap_{i=1}^n V(f_{\alpha_i}) = \bigcup_{i=1}^n (X - V(f_{\alpha_i})) = \bigcup_{i=1}^n D(f_{\alpha_i}).
	\end{equation*}
	This exhibits a finite subcover.
\end{proof}

\begin{proposition}
	$X = \Spec(A)$ is disconnected if and only if $A$ admits a ring product decomposition $A\cong A_1\times A_2$.
\end{proposition}
\begin{proof}
	For the reverse direction, suppose $A=A_1\times A_2$. By (\todo{ref hw}), 
	\begin{align*}
		\Spec(A) 
		=& \{\mathfrak{p}_1\times A_2 : \mathfrak{p}_1\in\Spec(A_1)\} \sqcup \{A_1\times\mathfrak{p}_2 : \mathfrak{p}_2\in\Spec(A_2)\} \\
		=& V(0\times A_2) \sqcup V(A_1\times 0).
	\end{align*}
	This expresses $X$ as the disjoint union of two nontrivial closed sets, hence $X$ is disconnected.

	For the forward direction, suppose $X=V(\mathfrak{a})\sqcup V(\mathfrak{b})$ for some ideal $\mathfrak{a},\mathfrak{b}\subset A$. Then 
	\begin{equation*}
		X = V(\mathfrak{a})\cup V(\mathfrak{b}) = V(\mathfrak{a}\cap\mathfrak{b})
	\end{equation*}
	implying $\mathfrak{a}\cap \mathfrak{b}\subset\text{Nil}(A)$, since it lies inside every prime ideal and the nilradical can be characterized as the intersection of all prime ideals (\todo{ref}). Also 
	\begin{equation*}
		\emptyset = V(\mathfrak{a})\cap V(\mathfrak{b}) = V(\mathfrak{a}+\mathfrak{b}),
	\end{equation*}
	implying $\mathfrak{a}+\mathfrak{b}=A$. Hence there exists $a\in\mathfrak{a}$ and $b\in\mathfrak{b}$ such that $a+b=1$. Pulling these two observations together, we can find $a,b$ such that $a+b=1$ and $(ab)^N=0$ for some $N$.

	We now claim $(a^N,b^N)=A$, where the notation denotes the ideal generated by the two elements. Suppose not. Then there exists a maximal ideal $\mathfrak{m}\supset (a^N,b^N)$. But $a^N\in\mathfrak{m}$ implies $a\in\mathfrak{m}$, and similarly if $b^N\in\mathfrak{m}$ then $b\in\mathfrak{m}$. But then $(a,b)=A\subset\mathfrak{m}$, which is a contradiction.

	So there exists $c,d\in A$ such that $1=ca^N+db^N$. Write $\alpha\coloneqq ca^N$ and $\beta=db^N$ for convenience. Then 
	\begin{equation*}
		\alpha\beta = ca^N\cdot db^N = cd(ab)^N=0.
	\end{equation*}
	Now 
	\begin{equation*}
		\alpha = \alpha\cdot 1 = \alpha(\alpha+\beta) = \alpha^2 + \alpha\beta = \alpha^2,
	\end{equation*}
	so $\alpha$ is idempotent. Similarly $\beta=\beta^2$. Hence $\alpha,\beta$ are a complete set of orthogonal idempotents, so by (\todo{ref}) $A\cong A_1\times A_2=A\alpha\times A\beta$.
\end{proof}

% Zariski topology }}}2

\subsection{upgraded Galois connection} % {{{2 

\begin{para}
	(\todo{ref}) shows us how $V(-)$ extended our previous $\mathbf{V}(-)$. How should we extend $\mathbf{I}(-)$? For a subset $S\subset X=\Spec(A)$, we want to think of this as the set of elements in $A$ on which $S$ vanishes. In the case where $S$ is a single element:
	\begin{align*}
		I(\{\mathfrak{p}\})
		=& \{f\in A : f(\mathfrak{p})=0\} \\
		=& \{f\in A : f\in\mathfrak{p}\} = \mathfrak{p},
	\end{align*}
	where we have used (\todo{ref}). The natural way to extend this is 
	\begin{align*}
		I(S) 
		=& \{f\in A : f(\mathfrak{p})=0 \text{ for all }\mathfrak{p}\in S\} \\
		=& \bigcap_{\mathfrak{p}\in S} \{f : f(\mathfrak{p})=0 \} = \bigcap_{\mathfrak{p}\in S}\mathfrak{p} \\
		=& \bigcap S.
	\end{align*}
	This motivates the following:
\end{para}

\begin{definition}
	For $S\subset X$, define (the ideal) 
	\begin{equation*}
		I(S) \coloneqq \bigcap_{\mathfrak{p}\in S}\mathfrak{p} = \bigcap S\subset A.
	\end{equation*}
\end{definition}

\begin{corollary}
	\hfill 
	\begin{enumerate}
		\item If $E\subset A$, then $V(E)\subset X$ is closed.
		\item If $S\subset X$, then $I(S)$ is an intersection of prime ideals, hence radical.
		\item $I,V$ are order reversing.
		\item $IV$ and $VI$ are inflationary.
		\item $\im(V)=\{\text{closed subsets of }X\}$.
		\item (formal Nullstellensatz) $\im I=\{\text{radical ideals of }A\}$.
	\end{enumerate}
\end{corollary}
\begin{proof}
	\hfill 
	\begin{enumerate}
		\item x 
		\item x 
		\item x 
		\item Note
			\begin{align*}
				I(V(\mathfrak{a})) 
				=& I(\{\mathfrak{p}\in\Spec(A) : \mathfrak{p}\supset\mathfrak{a}\}) \\
				=& \bigcap \{\mathfrak{p}\in\Spec(A) : \mathfrak{p}\supset\mathfrak{a}\} \\
				=& \sqrt{\mathfrak{a}}\supset\mathfrak{a}.
			\end{align*}
		\item x 
		\item The forward inclusion follows from (2). For the reverse inclusion, suppose $\mathfrak{a}\subset A$ is radical. Then $\mathfrak{a}=\sqrt{\mathfrak{a}}$, hence by the proof of (4) $I(V(\mathfrak{a}))=\sqrt{\mathfrak{a}}$.
	\end{enumerate}
\end{proof}

\begin{para}
	We thus have the following correspondences:
	\begin{itemize}
		\item Zariski closed subsets of $\Spec(A)$ and radical ideals of $A$
		\item irreducible closed subsets of $\Spec(A)$ and prime ideals of $A$
		\item closed points of $\Spec(A)$ and maximal ideals of $A$
	\end{itemize}
\end{para}

% upgraded Galois connection }}}2

\subsection{Noetherian spaces} % {{{2 

\begin{definition}
	A space $X$ is \emph{Noetherian} if it satisfies the ascending chain condition (ACC) on open sets.
\end{definition}

\begin{remark}
	This is equivalent to satisfying the descending chain condition (DCC) on closed sets. We will often use this property instead, because the closed sets we work with are easier to describe than the open ones.
\end{remark}

\begin{proposition}
\label{prop_a_noetherian_speca_noetherian}
	If $A$ is a Noetherian ring, then $\Spec(A)$ is a Noetherian space.
\end{proposition}
\begin{proof}
	Let $A$ be a Noetherian ring. Let $X_1\supset X_2\supset\cdots$ be a descending chain of closed suspaces of $X$. Then $I(X_1)\subset I(X_2)\subset\cdots$ is an ascending chain of ideals in $A$, hence stabilizes. So there exists $N$ such that $I(X_N)=I(X_{N+1})=\cdots$. Hence also $VI(X_N)=VI(X_{N+1})=\cdots$. Thus $X_N=X_{N+1}=\cdots$, since $VI$ is the identity on closed subsets. 
\end{proof}

\begin{proposition}
	Let $X$ be a Noetherian space. Then 
	\begin{enumerate}
		\item $X$ is a finite union $X=X_1\cup\cdots\cup X_n$ of irreducible closed subspaces.
		\item If the above union (decomposition) is irredundant\footnote{i.e. we can't remove any $X_i$, i.e. $X_i\not\subset X_j$ for any $i\neq j$}, then it is unique up to permutation.
	\end{enumerate}
\end{proposition}
\begin{proof}
	\hfill 
	\begin{enumerate}
		\item Let $\Sigma$ denote the set of nonempty, closed subsets of $X$ which are note a finite union of irreducible closed subsets. It suffices to show $\Sigma$ is empty. Suppose otherwise. Then, since $X$ is Noetherian, $\Sigma$ has a minimal element $X_M$. By assumption $X_M$ is not irreducible, hence can be written as $X_M=X_1\sqcup X_2$ for some nontrival closed $X_1,X_2\subset X_M$. Then $X_1,X_2\not\in\Sigma$ by minimality, hence are themselves both finite unions of closed irreducible subspaces. But then so is $X_M$, a contradiction.
		\item Suppose $X=X_1\cup\cdots\cup X_n$ and $X=Y_1\cup\cdots\cup Y_m$ are both irredundant decompositions. Then for each $i$ 
			\begin{equation*}
				X_i = (X_i \cap X) = X_i \cap (Y_1\cup\cdots\cup Y_m) = (X_i\cap Y_1) \cup \cdots \cup (X_i\cap Y_m),
			\end{equation*}
			which expresses $X_i$ as a finite union of closed sets. Since $X_i$ is irreducible, it must be that $X_i=X_i\cap Y_j$ for some $j$, hence $X_i\subset Y_j$. We can thus define a function $\alpha$ such that $X_i\subset Y_{\alpha(i)}$ for all $i=1,\dots,n$. We can analagously get a function $\beta$ such that $Y_j\subset X_{\beta(j)}$ for all $j=1,\dots,m$. Then $X_i\subset Y_{\alpha(i)}\subset X_{\beta(\alpha(i))}$, and by the irredundancy assumption it must be that $\beta\circ\alpha=1$. Similarly $\alpha\circ\beta=1$. Hence $m=n$, and $X_i=Y_j$ for all $i$ and some $j=1,\dots,n$.
	\end{enumerate}
\end{proof}

\begin{corollary}
	The $X_i$ appearing in the (irredundant) decomposition of $X$ are the maximal irreducible closed subsets of $X$.
\end{corollary}

\begin{corollary}
	Let $A$ be a Noetherian ring. Then 
	\begin{enumerate}
		\item any radical ideal $\mathfrak{a}\subset A$ is a finite intersection of prime ideals (recall every radical ideal in a not-necessarily-Noetherian ring is a not-necessarily-finite intersection, \todo{ref}).
		\item If the above intersection is irredundant, then it is unique up to ordering.
	\end{enumerate}
\end{corollary}
\begin{proof}
	$V(\mathfrak{a})$ is a Noetherian space, since it is a closed subset of $\text{Spec}(A)$ which is a Noetherian space by Proposition \ref{prop_a_noetherian_speca_noetherian}. Those it has an irredundant decomposition 
	\begin{equation*}
		V(\mathfrak{a}) = V_1 \cup \cdots \cup V_n
	\end{equation*}
	whose components $V_i$ are unique up to ordering. Also $V_i\subset V(\mathfrak{a})$ is irreducible and closed, hence irreducible and closed in $\Spec(A)$. Thus $\mathfrak{p}_i = I(V_i)$ is prime, by Proposition \ref{prop_y_irred_closed_then_iy_prime}. Also note $V(\mathfrak{p}_i) = VI(V_i) = V_i$, so 
	\begin{equation*}
		\mathfrak{a} = \sqrt{\mathfrak{a}} = IV(\mathfrak{a}) =I(V_1\cup\cdots\cup V_n) = I(V_1)\cap\cdots I(V_n) = \mathfrak{p}_1\cap\cdots\cap\mathfrak{p}_n,
	\end{equation*}
	and uniqueness follows.
\end{proof}

% Noetherian spaces }}}2

% affine schemes }}}1

\section{singularities} % {{{1 

\begin{example}
	Consider the variety $V=\mathbf{V}(Y^2-X^3)\subset\mathbb{A}^2$, where as usual we are over an algebraically-closed field $k$. As we have mentioned before, the coordinate ring is 
	\begin{equation*}
		A \coloneqq \Gamma(V) = \frac{k[X,Y]}{(Y^2-X^3)}.
	\end{equation*}
	Let $x,y$ be the elements of $\Gamma(V)$ which pick out the 1st and 2nd coordinate, respectively. These are the cosets of the generators of $k[X,Y]$, hence are generators of $\Gamma(V)$. Consider the parameterization $\alpha:\mathbb{A}^1\to V$ be $\alpha(t)=(t^2,t^3)$. The picture is as follows:

	\begin{centering}
		\begin{tikzpicture}
			\draw [thick] (-.5, 0) -- (3, 0);
			\draw [thick] (0, -2) -- (0, 2);
			\node[] at (0.5, -0.5) {$\mathbb{A}^2$};
			\draw (0,0) .. controls (1,0) and (1.6,.5) .. (2,2);
			\draw (0,0) .. controls (1,0) and (1.6,-.5) .. (2,-2);

			\node[] at (0.5,-3.5) {$\mathbb{A}^1$};
			\draw [thick] (-3, -3) -- (3, -3);

			\draw[->] (-4, -2.5)--(-4, -0.5) node[midway,left]{$\alpha(t)=(t^2,t^3)$};

			\path (-8,0) -- (8,0);
		\end{tikzpicture}
	\end{centering}

	Evidently there is something going on at $V$ at the origin. Is this detected algebraically? Clearly $x$ and $y$ are related, namely $x^3=y^2$. In particular, if we let $t=\frac{y}{x}\in\text{Frac}(A)$, then $t$ satisfies the monic polynomial $z^2-x\in A[z]$:
	\begin{equation*}
		t^2 - x = \frac{y^2}{x^2}-x = \frac{x^3}{x^2} - x = x - x = 0.	
	\end{equation*}
	Thus $t$ is integral, over $A$. However $t$ is not in $A$, which shows that $A$ is not integrally closed. The failure of $A$ to be integrally closed can be shown to indicate the existence of singularities.
	
	But we can map $\mathbb{A}^1$ bijectively onto $V$ via $\alpha$, and yet $\mathbb{A}^1$ apparently does not have singularities. In some sense the object $\mathbb{A}^1$ has ``resolved'' the singularities of $V$. How is this detected algebraically? Given our discussion, we might expect this to present itself in the fact that $\Gamma(\mathbb{A}^1)$ is integrally closed, but somehow related to $A\coloneqq\Gamma(V)$.

	The ring we are looking for is the integral closure of $A$, i.e. we might expect $\gamma(\mathbb{A}^1)=k[T]$ to be the integral closure of $A\coloneqq\Gamma(V)$. Is this the case? Let's try to construct the integral closure of $A$. We can first adjoint the element $t$ to get $A[t]$. We have from above that $t^2=x$, but also 
	\begin{equation*}
		t^3 = \frac{y^3}{x^3}=\frac{y^3}{y^2}=y.
	\end{equation*}
	The effect is that any polynomial in $A[t]$ can be rewritten as a polynomial in $k[t]$ via the substitutions $x\mapsto t^2$ and $y\mapsto t^3$. Note that this is very similar to the situation on coordinate rings:
	\begin{gather*}
		\alpha^\sharp: \Gamma(V) \to \Gamma(\mathbb{A}^1)=k[T] \\
		x \mapsto T^2 \\
		y \mapsto T^3
	\end{gather*}
	which might suggest where this is going. Now $A[t]$ surjects onto $k[t]\cong k[T]$, which is integrally closed since it is a UFD. But also $A[t]$ is integral over $A$, so it must be that $A[t]$ is the integral closure of $A$, as is (the isomorphic image of) $k[T]$, which is precisely the coordinate ring of $\mathbb{A}^1$. 
\end{example}

\begin{definition}
	Let $V=\mathbf{V}(f)\subset\mathbb{A}^2$, where $f\in k[X,Y]$ is irreducible. We say that $V$ is \emph{nonsingular} if, for all $\underline{a}\in V$,
	\begin{equation*}
		\frac{\partial f}{\partial x}(\underline{a})\neq 0 \quad \text{or} \quad \frac{\partial f}{\partial y}(\underline{a})\neq 0.
	\end{equation*}
\end{definition}

\begin{remark}
	Compare with the hypothesis for the implicit function theorem for smooth manifolds.
\end{remark}

\begin{proposition}
	$V=\mathbf{V}(f)$ is nonsingular if and only if $A\coloneqq\Gamma(V)$ is integrally closed.
\end{proposition}

\begin{proposition}
	Let $V=\mathbf{V}(f)$ as above. Let $\overline{A}$ be the integral closure of $A$. Then there exists a nonsingular curve $\overline{V}\subset\mathbb{A}^n$ (for some $n$) with $\Gamma(\overline{V})=\overline{A}$ and a polynomial map $F:\overline{V}\to V$ inducducing 
	\begin{equation*}
		F^\sharp: \Gamma(V)=A \to \overline{A}=\Gamma(\overline{V}).
	\end{equation*}
\end{proposition}

\begin{example}
	$V=\mathbf{V}(Y^2-X^3-X^2)$
\end{example}

% singularities }}}1

\section{dimension} % {{{1 

As usual we work over an algebraically-closed field $k$.

\begin{definition}
	Let $X$ be a Noetherian topological space. The \emph{dimension} of $X$ is defined as 
	\begin{equation*}
		\text{dim}(X) \coloneqq \sup_{n\in\mathbb{N}} C_n,
	\end{equation*}
	where $C_n$ is a chain $X_0\subsetneq X_1 \subsetneq \cdots\subsetneq X_n$ of subspaces in $X$ such that each $X_i$ is closed and irreducible. The empty set is not irreducible by convention.
\end{definition}

\begin{definition}
	Let $A$ be a Noetherian ring, let $\mathfrak{p}\in\Spec(A)$. The \emph{height} of $\mathfrak{p}$ is defined as 
	\begin{equation*}
		\text{ht}(\mathfrak{p})\coloneqq\sup_{n\in\mathbb{N}}C_n,
	\end{equation*}
	where $C_n$ is a chain $\mathfrak{p}_0\subsetneq\mathfrak{p}_1\subsetneq\cdots\subsetneq\mathfrak{p}_n$ of prime ideals in $\mathfrak{p}$.

	The \emph{Krull dimension} of $A$ is 
	\begin{equation*}
		\text{dim}(A) \coloneqq \sup_{\mathfrak{p}\in\Spec(A)}\text{ht}(\mathfrak{p}).
	\end{equation*}
\end{definition}

\begin{proposition}
	Let $X\subset\mathbb{A}^n$ be an affine variety. Then 
	\begin{equation*}
		\text{dim}(X) = \text{dim}(\Gamma(X)).
	\end{equation*}
\end{proposition}
\begin{proof}
	By (ref), closed irreducible subsets of $X$ correspond bijectively to prime ideals of $\Gamma(X)$.
\end{proof}

\begin{example}
	The dimension of a field is 0, since the only prime ideal is $(0)$. The dimension of $\mathbb{Z}$ is 1, since the only prime ideals are those generated by a prime number $p$, and the only ideal contained in them is $(0)$.	
\end{example}

\begin{proposition}
	\begin{equation*}
		\text{dim}(k[X_1,\dots,X_n]) = \text{dim}(\mathbb{A}^n) \geq n.
	\end{equation*}
\end{proposition}
\begin{proof}
	The equality follows from the above proposition. For the inequality, not the we have a chain 
	\begin{equation*}
		(0) \subsetneq (X_1) \subsetneq \cdots \subsetneq (X_1,\dots, X_n)
	\end{equation*}
	of length $n$, so $\text{ht}(X_1,\dots,X_n)\geq n$ so the dimension is at least $n$.
\end{proof}

\begin{remark}
	There are some pathological examples, due to Nagata, where... a Noetherian ring with infinite dimension.
\end{remark}

\begin{para}
	There is another natural notion of the dimension of an affine domain $B$ over $k$, namely the transcendence degree 
	\begin{equation*}
		\text{dim}'(B) = \text{trdeg}_k(\text{Frac}(B)) = \text{tr}_k(B).
	\end{equation*}
	The coordinate ring of a variety $X$ is an affine domain, so this presents 2 potentially distinct notions of dimension: the dimension of $X$ as a Noetherian space (equivalently the Krull dimension of $\Gamma(X)$) and the transcendence degree of $\Gamma(X)$.
\end{para}

\begin{proposition}
	Let $A\subset B$ be an integral extension. Then $\text{dim}(B) = \text{dim}(A)$.
\end{proposition}
\begin{proof}
	($\leq$) Let $\mathfrak{q}\in\Spec(B)$, and suppose we have a chain 
	\begin{equation*}
		\mathfrak{q}_0\subsetneq\cdots\subsetneq\mathfrak{q}_r=\mathfrak{q}
	\end{equation*}
	of prime ideals in $B$. Let $\mathfrak{p}=\mathfrak{q}\cap A=\mathfrak{q}^c$, and similarly define $\mathfrak{p}_0,\dots,\mathfrak{p}_r$ so that $\mathfrak{p}_i=\mathfrak{q}_i^c$. Then 
	\begin{equation*}
		\mathfrak{p}_0\subsetneq\mathfrak{p}_1\subsetneq\cdots\subsetneq\mathfrak{p}_r 
	\end{equation*}
	is a chain of prime ideal in $A$, where the inclusions are strict by the incomparability Cohen-Seidenberg theorem. So $\text{ht}(\mathfrak{q})\leq\text{ht}(\mathfrak{p})=\text{ht}(\mathfrak{q}\cap A)$. Then 
	\begin{equation*}
		\text{dim}(B) = \sup_{\mathfrak{q}\in\Spec(B)}\text{ht}(\mathfrak{q})\leq\sup_{\mathfrak{q}\in\Spec(B)}\text{ht}(\mathfrak{q}\cap A)\leq\sup{_{\mathfrak{p}\in\Spec(A)}}\text{ht}(\mathfrak{p})=\text{dim}(A).
	\end{equation*}
	So $\text{dim}(B)\leq\text{dim}(A)$.

	($\geq$) Let $\mathfrak{p}\in\Spec(A)$. Let 
	\begin{equation*}
		\mathfrak{p}_0\subsetneq\mathfrak{p}_1\subsetneq\cdots\subsetneq\mathfrak{p}_r=\mathfrak{p}
	\end{equation*}
	be a chain of primes in $A$. By the lying over theorem, there exists $\mathfrak{q}_0\in\Spec(B)$ such that $A\cap \mathfrak{q}_0=\mathfrak{p}_0$. By the going up theorem we can extend this to a chain 
	\begin{equation*}
		\mathfrak{q}_0\subsetneq\mathfrak{q}_1\subsetneq\cdots\subsetneq q_r
	\end{equation*}
	of primes in $B$ where $\mathfrak{q}_i^c=\mathfrak{p}_i$, and where the inclusions are strict since $\mathfrak{q}_i\cap A=\mathfrak{p}_i\neq\mathfrak{p}_{i+1}=\mathfrak{q}_{i+1}\cap A$. Thus any chain in $A$ lifts to a chain in $B$ of the same length, so for all $\mathfrak{p}\in\Spec(A)$ there exists $\mathfrak{q}\in\Spec(B)$ such that $A\cap \mathfrak{q}=\mathfrak{p}$ and $\text{ht}(\mathfrak{q})\geq\text{ht}(\mathfrak{p})$. Thus 
	\begin{equation*}
		\text{dim}(A)=\sup_{\mathfrak{p}\in\Spec(A)}\text{ht}(\mathfrak{p})\leq\sup_{\mathfrak{q}\in\Spec(B)}\text{ht}(\mathfrak{q})=\text{dim}(B).
	\end{equation*}
	So $\text{dim}(A)\leq\text{dim}(B)$.
\end{proof}

\begin{theorem}
	Let $B$ be an affine domain. Then $\text{dim}(B)=\text{trdeg}_k(B)$.
\end{theorem}
\begin{proof}
	We induct on $d=\text{trdeg}_k(B)$. By Noether normalization, there exists $b_1,\dots,b_d\in B$ such that the $b_i$ are algebraically independent and we have a factorization $k\to k[b_1,\dots,b_d]\to B$ where the first map is polynomial and the second is integral. We write $A=k[b_1,\dots,b_d]$.

	($d=0$) Suppose $B$ is integral over $A$. By (ref) $B$ is a field if and only if $k$ is a field, and so $B$ is a field and hence has dimension 0. But also $B=A$ is a polynomial extension over $k$, hence $B$ is algebraic over $k$, so the $\text{trdeg}_k(B)=0$.

	($d>0$) Suppose the result holds for domains with $\text{trdeg}_k < d$. By (12.6),
	\begin{equation*}
		\text{dim}(B)=\text{dim}(A)\geq d = \text{trdeg}_k(A) = \text{trdeg}_k(B),
	\end{equation*}
	where the inequality is from (12.5) and the equality is the induction hypothesis. For the reverse inequality, it evidently suffices to show $\text{dim}(A)\leq d$. Suppose otherwise, i.e. $\text{dim}(k[T_1,\dots,T_d])>d$. Then there exists  achain 
	\begin{equation*}
		(0) \subsetneq \mathfrak{p}_1\subsetneq\cdots\subsetneq\mathfrak{p}_{d+1}
	\end{equation*}
	of primes of length $d+1$ in $A=k[T_1,\dots,T_d]$. Let $\overline{A}=A/\mathfrak{p}_1$. This is a domain, since $\mathfrak{p}_1$ is prime. We claim that $\text{trdeg}_k(\overline{A})<\text{trdeg}_k(A)=d$. Let us show how the claim completes the proof, and then we will prove the claim. If the claim holds, then by induction hypothesis $\text{dim}(\overline{A})=\text{trdeg}(\overline{A})<d$. But in $\overline{A}$, we have a chain $0=\mathfrak{p}_1/\mathfrak{p}_1\subsetneq\cdots\subsetneq\mathfrak{p}_{d+1}/\mathfrak{p}_{d}$ (ref correspondence of ideals in quotient), which is a contradiction (it has length $d$).

	Now let us prove the claim. Let $0\neq f\in\mathfrak{p}_1$, so that $f\not\in k$. So $f$ is transcendental over $k$. Then $\{f\}$ can be enlarged to a transcendence basis $\{f=f_1,\dots,f_d\}$ of $k[T_1,\dots,T_d]$. Then $A=k[f_1,\dots,f_d]$, so $\overline{A}=k[\overline{f_2},\dots,\overline{f_d}]$, where we have used the fact that $\overline{f}=0$ since $f\in\mathfrak{p}_1$. So $\{\overline{f_2},\dots,\overline{f_d}\}$ at least contains a transcendence basis of $\overline{A}$, so $\text{trdeg}_k(\overline{A})\leq d-1<d=\text{trdeg}_k(A)$.
\end{proof}

\begin{para}
	We give a geometric interpretation of Noether normalization. Let $X\subset\mathbb{A}^n$ be an affine variety, so that 
	\begin{equation*}
		\Gamma(X)=\frac{k[X_1,\dots,X_n]}{\mathbf{I}(X)}
	\end{equation*}
	is an affine domain. Let $d=\text{dim}(X)\text{dim}(\Gamma(X))$. By Noether normalization, $B\coloneqq\Gamma(X)$ is integral over some polynomial algebra $A=k[b_1,\dots,b_d]\cong k[T_1,\dots,T_d]$. This is the coordinate ring of $\mathbb{A}^d$ for some $d\leq n$.
\end{para}

% dimension }}}1

\section{sheaves} % {{{1 

\begin{para}
	Let $X$ be a topological space. For any open $U\subset X$, we have an associated $\mathbb{R}$-algebra 
	\begin{equation*}
		C_X(U) \coloneqq \{ \text{continuous functions } U\to\mathbb{R} \}.
	\end{equation*}
	For open inclusions $V\subset U\subset X$, we have an induced ring map 
	\begin{equation*}
		\text{res}^U_V:C_X(U) \to C_X(V)
	\end{equation*}
	sending a function on the larger open set to its restriction on the smaller one.

	Letting $\text{Ouv}(X)$ be the poset category of open sets of $X$, with the partial ordering being inclusion, then $C_X$ defines a contravariant functor $\text{Ouv}(X)\to\mathbb{R}\text{Alg}$.
\end{para}

\begin{definition}
	Let $X$ be a topological space. A \emph{presheaf} on $X$ with values in a category $\cat{C}$ is a contravariant functor $\mathscr{F}:\text{Ouv}(X)\to\cat{C}$.
\end{definition}

\begin{para}
	If $V\hookrightarrow U$ is an arrow in $\text{Ouv}(X)$, then there is the induced map $\text{res}^U_V:\mathscr{F}(U)\to\mathscr{F}(V)$ (note the abuse of notation). A morphism of presheaves $\phi:\mathscr{F}\to\mathscr{G}$ on $X$ is a natural transformation: if $V\hookrightarrow U$, this diagram will commute:
	\begin{equation*}
		% https://tikzcd.yichuanshen.de/#N4Igdg9gJgpgziAXAbVABwnAlgFyxMJZABgBpiBdUkANwEMAbAVxiRAB12BbOnACzgBjAE7AAYgF8AFAFUAlCAml0mXPkIoAjOSq1GLNpx78hogOLT5i5SAzY8BImU276zVog7deAkeOkAagpKKvbqRNou1G4GnkY+psAWUkGKujBQAObwRKAAZsIQXEhkIDgQSNp67obsaHxYAPoy1vmFxYgATNTlSADM0foeXvVNAa0gBUUlPRVdgzVx7DgwAB44wMLwEiDUDHQARjAMAAqqDhogwliZfDgTUx1VvYgD1bFeK+ub22kSQA
\begin{tikzcd}
\mathscr{F}(U) \arrow[r, "\phi_U"] \arrow[d, "\text{res}"'] & \mathscr{G}(U) \arrow[d, "\text{res}"] \\
\mathscr{F}(V) \arrow[r, "\phi_V"]                          & \mathscr{G}(V)                        
\end{tikzcd}
	\end{equation*}
\end{para}

\begin{para}
	Returning to the example from before, we see that $C_X$ is a presheaf of $\mathbb{R}$-algebras on $X$. Note $C_X$ has two further properties arising from the local nature (!) of continuity:
	\begin{enumerate}[label=S\arabic*.,ref=S\arabic*]
		\item\label{sheaf_local_determination} Let $U\subset X$ be open, and let $\mathcal{U}=\{U_\alpha\}_\alpha$ be an open cover of $U$. If $f,g\in C_X(U)$ are such that $f|_{U_\alpha}=g|_{U_\alpha}$ for all $\alpha$, then $f=g$. In other words, the following map is injective: 
			\begin{equation*}
				(\text{res}^U_{U_\alpha})_\alpha: C_X(U) \to \prod_{U_\alpha\in\mathcal{U}} C_X(U_\alpha)
			\end{equation*}

		\item\label{sheaf_patching} Let $U\subset X$ be open, and let $\mathcal{U}=\{U_\alpha\}_\alpha$ be an open cover of $U$. Suppose $\{f_\gamma\in C_X(U_\alpha)\}_{U_\gamma\in \mathcal{U}}$ is a collection such that $f_\alpha|_{U_\alpha\cap U_\beta}=f_\beta|_{U_\alpha\cap U_\beta}$ for all $\alpha,\beta$. Then there exists $f\in C_X(U)$ such that $f|_{U_\alpha}=f_\alpha$ for all $\alpha$. In other words, the following is an equalizer:
			\begin{equation*}
				% https://tikzcd.yichuanshen.de/#N4Igdg9gJgpgziAXAbVABwnAlgFyxMJZABgBpiBdUkANwEMAbAVxiRAGEB9ADQAoBVAJQgAvqXSZc+QigCM5KrUYs2AAgA66tACdonYP06aA5nQC2ZupqxhNlnAAsAxowMiRqrn0ObGaB3TCYhLYeAREAEwK1PTMrIggmjp6BkbqfgGkPuoARjA4gda26vbOrvwimnhm8HZ0ji4Mbh5eAmkZVuouaKrZeQVBijBQxvBEoABmumZIZCA4EEjySnFsvFUwAB44wNrwIgB6hqkm5pYigvrZphadNnUN5e6i4iBTEDOIywtIUSBwDiwExwSAAtMtYioElADsQXpNpktqD9EH8AUCQV8Ysp4iAYbIQNQGHQ8gwAAqSMIyEDaLDGBwgkQUERAA
\begin{tikzcd}
	C_X(U) \arrow[r, "(\text{res}^U_{U_\gamma})_{U_\gamma\in\mathcal{U}}"] &  {\displaystyle \prod_{U_\gamma\in\mathcal{U}} C_X(U_\alpha)} \arrow[r, "d^0", shift left] \arrow[r, "d^1"', shift right] & {\displaystyle \prod_{(U_\alpha,U_\beta)\in\mathcal{U}\times\mathcal{U}} C_X(U_\alpha\cap U_\beta)}
\end{tikzcd},
			\end{equation*}
			where the maps $d^0, d^1$ are defined by the universal property of products in the following way:
			\begin{equation*}
				% https://tikzcd.yichuanshen.de/#N4Igdg9gJgpgziAXAbVABwnAlgFyxMJZABgBoBGAXVJADcBDAGwFcYkRgAdTqLONRvQCecHEMYwABNzQAnaAH1gAVQXcA5vQC2W+tyxhuunAAsAxkxUBfK5IDCCgBoAKVRu26AlFZBXS6TFx8QhQyYmo6JlZ2Bxc3TiY0E3pPX38QDGw8AiJyUnCaBhY2RBBY1zUExiS9Tgs0SXiAIxgcFLSArODcigii6NKuHj4BYVFxKRl5KCUK7kTk0kbKlrbPfUNOY3NLZStuPC14I3pTC0ZrK3L4hdr65e5VlJ8-TqCc0NIAJj6okrKnHNOE9Uq8MoFsiFkHkfoU-jFATdqsluPdmq12lYIjAoOp4ERQAAzeRaJBkEA4CBIAAscOK7BkWBWGJANEELUYAAUId1SrIsOoTDgOiBiRBSYhyZSkHlIvTSozKrcRWKJbLpYgvnSBiADjAAB44YCyeAvdKqmk0DUAVm1-z1huNppVJKQAGYrVTELa5TrFUNbktHhizUTXYgPRSvVrffbOGgmQHkfQg8CQ6yQOyYFyeR8QPzBcKwRbJZ6kF9i+GpV7rZXxWSyxGaHATFhCcLEABaWX9f5QAB6xAzjAMffoLZxLvrpaj7ubrfbMrZo-YUHHJkndtX-fIw-oHO5XTzBaFvkoViAA
\begin{tikzcd}
C_X(U_\alpha) \arrow[r, "\text{res}"]                                                                                                                                                                          & C_X(U_\alpha\cap U_\beta)                                                                                                                                                    \\
{\displaystyle \prod_{U_\gamma\in\mathcal{U}} C_X(U_\gamma)} \arrow[d, "\pi_\beta"'] \arrow[u, "\pi_\alpha"] \arrow[ru] \arrow[rd] \arrow[r, "d^0", dashed, shift left] \arrow[r, "d^1"', dashed, shift right] & {{\displaystyle \prod_{(U_\alpha, U_\beta)\in\mathcal{U}\times\mathcal{U}}C_X(U_\alpha\cap U_\beta)}} \arrow[d, "{\pi_{\alpha, \beta}}"] \arrow[u, "{\pi_{\alpha, \beta}}"'] \\
C_X(U_\beta) \arrow[r, "\text{res}"]                                                                                                                                                                           & C_X(U_\alpha\cap U_\beta)                                                                                                                                                   
\end{tikzcd}
			\end{equation*}
	\end{enumerate}
\end{para}

\begin{definition}
	A \emph{sheaf} is a presheaf satisfying (\ref{sheaf_local_determination}) and (\ref{sheaf_patching}).
\end{definition}

\begin{proposition}
	If the objects of $\mathcal{C}$ are abelian groups, then the sheaf property can be expressed as the exactness of the following sequence:
	\begin{equation*}
		% https://tikzcd.yichuanshen.de/#N4Igdg9gJgpgziAXAbVABwnAlgFyxMJZABgBpiBdUkANwEMAbAVxiRGJAF9T1Nd9CKAIzkqtRizYAdKQFs6OABZwAxgCdgAMU4AKAKoBKLjxAZseAkQBMo6vWatEIYDKhY4aBnQCecHN4YYAAIZNDVoAH1gPQiZAHM6WXkZLDAZeSUVRmjOTnSFZXUtXRj4xPkDTmNecwEiAGZbcQc2Fyk3Dy9ff0CQqTDI4H1YqUY0RTpSINKpACMYHDoDFLS5AqyGHJk8WXh8zOy9XL6Mwo1tYZkxiZkstGmR+cXKrjEYKDj4IlAAM3DZJBkEA4CBIITcX7-MHUEFIGzNSROHTbGAADxwwDU8E4RghID+EABiHhsMQjRAcEUWB+OEBdgkjhAUAAesQALQsoSvThAA
\begin{tikzcd}
0 \arrow[r] & \mathscr{F}(U) \arrow[r, "(\text{res})"] & {\displaystyle \prod_{U_\gamma\in\mathcal{U}}\mathscr{F}(U_\gamma)} \arrow[r, "d^0-d^1"] & {{\displaystyle \prod_{(U_\alpha, U_\beta)\in\mathcal{U}\times\mathcal{U}} \mathscr{F}(U_\alpha\cap U_\beta)}}
\end{tikzcd}
	\end{equation*}
	where the maps send:
	\begin{gather*}
		(\text{res})(f) \mapsto (f|_{U_\gamma}), \\
		(d^0-d^1)((f_\gamma)_\gamma) \mapsto (f_\alpha|_{U_\alpha\cap U_\beta}-f_\beta|_{U_\alpha\cap U_\beta})_{\alpha,\beta}.
	\end{gather*}
\end{proposition}

\begin{example}
	\hfill 
	\begin{enumerate}
		\item $C_X$ is a sheaf on $X$.
		\item If $X$ is a smooth manifold, then $C_X^\infty$, defined on open $U\subset X$ by 
			\begin{equation*}
				C_X^\infty(U)\coloneqq \{\text{smooth functions }U\to\mathbb{R} \},
			\end{equation*}
			is a sheaf on $X$.
		\item If $X$ is a topological space, then $BC_X$, defined on open $U\subset X$ by 
			\begin{equation*}
				BC_X(U)\coloneqq \{\text{bounded continuous functions }U\to\mathbb{R} \},
			\end{equation*}
			is a presheaf that may not be a sheaf. For consider the case $X=\mathbb{R}$, and let $\mathcal{U}=\{U_n=(-n, n)\}_{n\in\mathbb{N}}$ be an open cover. Consider $f_n\in BC_\mathbb{R}(U_n)$ defined by $f_n(x)=x$. Then there is no bounded continuous function on $X$ such that $f|_{U_n}=f_n$. So boundedness is not detectable locally.
		\item Let $X$ be a reducible space, so $X=X_1\cup X_2$ with $X_1,X_2$ proper closed subsets of $X$. Let $U_1=X-X_1$ and $U_2=X-X_2$. Then $U_1\cap U_2=\emptyset$. Define a presheaf $\underline{\mathbb{R}}$ on $X$ by 
			\begin{equation*}
				\underline{\mathbb{R}}(U)=\{\text{constant functions }U\to\mathbb{R} \}
			\end{equation*}
			for any $U\subset X$ open. This is not a sheaf. Let $f_1\in\underline{\mathbb{R}}(U_1)$ be $f_1=1$, and let $f_2\in\underline{\mathbb{R}}(U_2)$ be $f_2=2$. Let $U=U_1\cup U_2$. Then $f_1,f_2$ satisfy patching (vacuosly), but no constant function on $U$ restricts to them. Hence constancy is not locally detectable.
	\end{enumerate}
\end{example}

\begin{definition}
	A morphism of sheaves on $X$ is a morphism of the underlying presheaves.
\end{definition}

\begin{example}
	Let $E$ be a topological space, and let $\pi:E\to X$ be a continuous surjection. Define a sheaf $\Gamma(-, E)$ on $X$ by 
	\begin{equation*}
		\Gamma(U, E) \coloneqq \{S: U\to E : S \text{ is continuous and }\pi\circ S=1_U \}.
	\end{equation*}
	These are the ``sections''. 

	In fact, we will see that \textit{any} sheaf $\mathscr{F}$ on $X$ is of this form. In particular, there exists a space $E=\acute{E}t(\mathscr{F})$, called the espace \'etal\'e, for which $\mathscr{F}=\Gamma(-, E)$. Hence for open $U\subset X$, we will often write $\mathscr{F}(U)=\Gamma(U,\mathscr{F})=H^0(U,\mathscr{F})$.
\end{example}

\begin{definition}
	A \emph{ringed space} is a pair $(X,\mathcal{O}_X)$ consisting of a space $X$ and a sheaf of rings $\mathcal{O}_X$ on $X$, called the \emph{structure sheaf}.
\end{definition}

\begin{example}
	\hfill 
	\begin{itemize}
		\item $(X, C_X)$ as above.
		\item $(X, C^\infty_X)$ where $X$ is a smooth manifold, as above.
		\item For $X\subset\mathbb{A}^n$ an affine algebraic set, $(X,\mathcal{O}_X)$ is a ringed space, where $\mathcal{O}_X$ is defined on open $U\subset X$ as 
			\begin{equation*}
				\mathcal{O}_X(U) \coloneqq \{\text{regular functions }U\to\mathbb{K} \}.
			\end{equation*}
	\end{itemize}
\end{example}

\begin{para}
	Let's consider what a map of ringed spaces is/should be. Taking our motivation from smooth manifolds, let $X,Y$ be smooth manifolds and consider a smooth map $\phi:X\to Y$ between them. If $V\subset Y$ is open, then $\phi^{-1}(V)\subset X$ is open by continuity. So if $f\in C^\infty_Y(V)$, then there exists a smooth function 
	\begin{equation*}
		\phi^\sharp(f)=f\circ\phi:\phi^{-1}(V)\to\mathbb{R},
	\end{equation*}
	where $\phi^\sharp(f)\in C^\infty_X(\phi^{-1}(V))$. We thus have a map of $\mathbb{R}$-algebras
	\begin{equation*}
		\phi^\sharp_V: C_Y^\infty(V) \to C^\infty_X(\phi^{-1}(V)).
	\end{equation*}
	If $W\subset V\subset Y$ is open, then $\phi^{-1}(W)\subset\phi^{-1}(V)$ and the following diagram commutes:
	\begin{equation*}
		% https://tikzcd.yichuanshen.de/#N4Igdg9gJgpgziAXAbVABwnAlgFyxMJZABgBpiBdUkANwEMAbAVxiRAGEA9AHW6zABmOAJ4B9AJoAKAGoBKEAF9S6TLnyEUARnJVajFmy69+QsQA1JvNAAssnYAFpNCmbPlKV2PASJlNu+mZWRA4JSQB1d2UQDC91Im1-akCDEPZRMx4+QRFLbhs7R2cIt0VdGCgAc3giUAEAJwgAWyQyEBwIJG09ILYrWyy4azp6tFFpRWiG5q7qDqQAZmT9YJBeHBgADxxgevgFSbrGlsQ2+cQAJmXekPWtnb24A48QaZOr9s7EJZ7UtfyBrwhiMxuEygogA
\begin{tikzcd}
C^\infty_Y(V) \arrow[r, "\phi^\sharp_V"] \arrow[d, "\text{res}"] & C^\infty_X(\phi^{-1}(V)) \arrow[d, "\text{res}"] \\
C_Y(W) \arrow[r, "\phi^\sharp_W"]                                & C_X^\infty(\phi^{-1}(W))                        
\end{tikzcd}.
	\end{equation*}
	In fact, this is a diagram of sheaves in the following sense: we can define a ``pushforward'' sheaf $\phi_\ast C^\infty_X$ (on $Y$, of $C^\infty_X$) by letting, for open $V\subset Y$, 
	\begin{equation*}
		(\phi_\ast C^\infty_X)(V) = C_X^\infty(\phi^{-1}(V)).
	\end{equation*}
	So it seems $\phi^\sharp$ is a map of sheaves $C^\infty_Y\to\phi_\ast C^\infty_X$.
\end{para}

\begin{definition}
	Let $\phi:X\to Y$ be a continuous map of spaces. Let $\mathscr{F}$ be a sheaf on $X$. Define the \emph{pushforward sheaf} or \emph{direct image sheaf} $\phi_\ast(\mathscr{F})$ as the sheaf on $Y$ given by, for $V\subset Y$ open,
	\begin{equation*}
		(\phi_\ast\mathscr{F})(V)=\mathscr{F}(\phi^{-1}(V)).
	\end{equation*}
\end{definition}

\begin{definition}
	Let $(X,\mathcal{O}_X)$ and $(Y,\mathcal{O}_Y)$ be ringed spaces. A morphism $(X,\mathcal{O}_X)\to(Y,\mathcal{O}_Y)$ is a pair $(\phi, \phi^\sharp)$, where $\phi:X\to Y$ is continuous and $\phi^\sharp:\mathcal{O}_Y\to\phi_\ast\mathcal{O}_X$ is a map of sheaves.
\end{definition}

\begin{definition}
	Let $(X,\mathcal{O}_X)$ be a ringed space, and let $U\subset X$ be open. Then $U$ inherits a natural structure of a ringed space 
	\begin{equation*}
		(U, \mathcal{O}_X|_{\text{Ouv}(U)}) = (U,\mathcal{O}_X|_U) = (U,\mathcal{O}_U)
	\end{equation*}
	through which there is an \emph{open immersion}
	\begin{equation*}
		(U,\mathcal{O}_X|_U) \hookrightarrow (X,\mathcal{O}_X).
	\end{equation*}
\end{definition}

\begin{example}
	Smooth manifolds are defined to have some local property which patches together in some compatible way. With regards to having a local property, we can say that there is a \textit{local model} for smooth manifolds, namely the ringed space $(U,C^\infty_U)$ where $U\subset\mathbb{R}^n$ is open. With regards to the compatibility of these local models, that is built into the definition of a sheaf. This motivates the following:

	\begin{definition}
		A smooth manifold is a ringed space $(X,\mathcal{O}_X)$, where $X$ is a second-countable Hausdorff space that is covered by open sets $U_\alpha\subset X$ such that, for each $\alpha$, $(\mathcal{U}_\alpha, \mathcal{O}_X|_{\mathcal{U}_\alpha})$ is ringed-space isomorphic to a local model $(U_\alpha, C^\infty_{U_\alpha})$.
	\end{definition}
\end{example}

\subsection{stalks and germs} % {{{2 

\begin{proof}
	Let $\mathscr{F}$ be a sheaf on $X$. Consider the neighborhood filter $\Lambda_x$ at $x\in X$ (i.e. the poset of open heighborhoods, ordered by reverse inclusion). This is a directed set (i.e. every pair is dominated by an element, e.g. if $U_1,U_2\in\Lambda_x$ then $U_1\cap U_2\in\Lambda_x$ and $U_1,U_2\leq U_1\cap U_2$). Then $\mathscr{F}$ restricts to a diagram $\Lambda_x\to\mathcal{C}$.
\end{proof}

\begin{definition}
	The \emph{stalk} of $\mathscr{F}$ at $x\in X$ is 
	\begin{equation*}
		\mathscr{F}_x\coloneqq \varinjlim_{U\ni x}\mathscr{F}(U).
	\end{equation*}
	The legs of the colimit cone $\mathscr{F}(U)\to \mathscr{F}_x$ are denoted 
	\begin{gather*}
		\text{res}^U_x: \mathscr{F}(U) \to \mathscr{F}_x \\
		s \mapsto s|_x
	\end{gather*}
	and sends $s\in\mathscr{F}(U)$ to its equivalence class $s|_x$. We call the equivalence class $s|_x$ the \emph{germ} of $s$ at $x$.
\end{definition}

\begin{remark}
	When $(X,\mathcal{O}_X)$ is a ringed space, we write the stalk as $\mathcal{O}_{X,x}$.
\end{remark}

\begin{example}
	Let $X$ be a smooth manifold with structure sheaf $\mathcal{O}_X=C^\infty_X$. For $U\subset X$ open and $x\in U$, we have the evaluation maps 
	\begin{gather*}
		\epsilon^U_x:\mathcal{O}_X(U) \to \mathbb{R} \\
		f \mapsto f(x).
	\end{gather*}
	In particular, these determine a cone under $\mathcal{O}_X:\Lambda_x\to\tcat{Ring}$ with apex $\mathbb{R}$:
	\begin{equation*}
		(\epsilon_x^U:\mathcal{O}_X(U)\to\mathbb{R})_{U\in\Lambda_x}.
	\end{equation*}
	By the universal mapping property of $\varinjlim$, there exists a unique ring map 
	\begin{equation*}
		\epsilon_x:\varinjlim_{U\ni x}\mathcal{O}_X(U) = \mathcal{O}_{X,x}\to\mathbb{R}
	\end{equation*}
	such that, for each open $U\subset X$ containing $x$ the following diagram commutes:
	\begin{equation*}
		% https://tikzcd.yichuanshen.de/#N4Igdg9gJgpgziAXAbVABwnAlgFyxMJZABgBoBGAXVJADcBDAGwFcYkQAdDgW3pwAsAxk2AB5AL4B9YAA1SAD3EhxpdJlz5CKMsWp0mrdl14DhjMVJkAKAKoBKZapAZseAkXIU9DFm0ScePn4AI2DgACUlcT0YKABzeCJQADMAJwhuJE8QHAgkACYaH0N-Lhg0bEYCAD0bSXkQGkZ6YJhGAAV1Ny0QRhhknEcU9MzEMhy8xEL9XyMOcsqCesberDA-ECh6OH5YlebWjq7NdlSsOP5BlWGMrJpcpHHija4cGHkcYFT4JSaWts6rhO-jOFyulHEQA
\begin{tikzcd}
\mathcal{O}_X(U) \arrow[rd, "\epsilon^U_x"] \arrow[d, "\text{res}"'] &            \\
{\mathcal{O}_{X,x}} \arrow[r, "\epsilon_x"', dashed]                 & \mathbb{R}
\end{tikzcd}
	\end{equation*}
	Now let $\mathfrak{m}_x=\ker(\epsilon_x)\subset\mathcal{O}_{X,x}$. This is automatically an ideal, but note also that since $\epsilon_x^U$ is surjective (it contains the constant functions), it must be that $\epsilon_x$ is surjective as well, so 
	\begin{equation*}
		\mathcal{O}_{X,x} / \mathfrak{m}_x \cong\mathbb{R}
	\end{equation*}
	which is a field, hence $\mathfrak{m}_x$ is actually a maximal ideal. 

	In fact, we claim $\mathfrak{m}_x$ is the unique maximal ideal in $\mathcal{O}_{X,x}$, i.e. that $\mathcal{O}_{X,x}$ is a local ring. To see this, it suffices to show (3.11) that all nonunits are contained in $\mathfrak{m}_x$, i.e. $\mathcal{O}_{X,x}-\mathfrak{m}_x\subset\mathcal{O}_{X,x}^\times$. Well, if $f_x\in\mathcal{O}_{X,x}-\mathfrak{m}_x$ then $f_x$ is the germ $f|_x$ of some $f\in\mathcal{O}_X(U)$ for some neighborhood $U$ of $x$, and furthermore $f_x\not\in \mathfrak{m}_x$ implies $0\neq\epsilon_x(f|_x)=f(x)$. This implies, since $f$ is continuous, that $f$ is nowhere zero on some neighborhood $V$ of $x$. In particular $f$ is invertible on $V$, so $f\in\mathcal{O}_X(V)^\times$. But the restriction $\mathcal{O}_X(V)\to\mathcal{O}_{X,x}$ is a ring map, hence sends units to units (?), hence $f|_x\in\mathcal{O}_{X,x}^\times$. This shows what we want, so $\mathcal{O}_{X,x}$ is local. Crucially, we recover $\mathbb{R}$ as $\mathbb{R}=\mathcal{O}_{X,x}/\mathfrak{m}_x$. 

	What we really used here was the continuity of $f$: the assumption that $f$ didn't vanish at $x$ (hence wasn't in $\mathfrak{m}_x$) implied by continuity that there existed an open neighborhood of $x$ on which $f$ was nowhere 0. As we will see, in the absence of continuity we will need to bake this into the definition.
\end{example}

\begin{definition}
	A \emph{locally ringed space} is a ringed space $(X,\mathcal{O}_X)$ such that $\mathcal{O}_{X,x}$ is a local ring for all $x\in X$. We call the associated field $\kappa(x)=\mathcal{O}_{X,x}/\mathfrak{m}_x$ the \emph{residue field} at $x$.
\end{definition}

\begin{remark}
	$f\in\mathcal{O}_X(U)$ determines a map 
	\begin{gather*}
		\mathcal{O}_X(U) \to \prod_{x\in U}\mathcal{O}_{X,x} \to \prod_{x\in U}\kappa(x) \\
		f \mapsto (f|_x)_x \mapsto (f|_x \ (\text{mod } \mathfrak{m}_x))_x.
	\end{gather*}
\end{remark}

% stalks and germs }}}2

\subsection{maps of stalks} % {{{2 

\begin{para}
	Let $\phi:\mathscr{F}\to\mathscr{G}$ be a map of presheaves on $X$. Then, for all $x,\in X$, $\phi$ induces a map $\phi_x:\mathscr{F}_x\to\mathscr{G}_x$ on stalks: for open $U\subset X$ containing $x$, we have the following:
	\begin{equation*}
		% https://tikzcd.yichuanshen.de/#N4Igdg9gJgpgziAXAbVABwnAlgFyxMJZABgBpiBdUkANwEMAbAVxiRAB12BbOnACzgBjAE7AAYgF8AFAFUAlCAml0mXPkIoyARiq1GLNpx78hogOLT5i5SAzY8BIlvK76zVog7deAkeIkA+gAe1ir26k6kOtRuBp5GPqbAFsGKujBQAObwRKAAZsIQXEhkIDgQSABMMfoeXjgwQTjAwvASoSAFRUjOZRWIAMw17obsDU0tbSDUDHQARjAMAAqqDhogwliZfDgdXcWI1X1IQyAMWGB1UHRwfBnTeiPx7Gh8WKlK+YUHpeU9w3EvK93jIHrMFstVhFPJttrtPp1viVqH9BhIKBIgA
\begin{tikzcd}
\mathscr{F}(U) \arrow[r, "\text{res}"] \arrow[d, "\phi_U"'] \arrow[rd] & \mathscr{F}_x \arrow[d, "\phi_x", dashed] \\
\mathscr{G}(U) \arrow[r, "\text{res}"']                                & \mathscr{G}_x                            
\end{tikzcd}
	\end{equation*}
	The induced map $\phi_x$ comes about because the composites $\text{res}\circ\phi_U$ form a cone under the diagram $\mathscr{F}$ (with apex $\mathscr{G}_x$), so the universal property applies. Explicitly, commmutativity of this diagram means 
	\begin{equation*}
		\phi_x(\text{germ at }x\in U\text{ of }f\in\mathscr{F}(U)) = (\text{germ at }x\in U\text{ of }\phi_U(f)\in\mathscr{G}(U)).
	\end{equation*}
\end{para}

\begin{para}
	Let $(\phi,\phi^\sharp): (X,\mathcal{O}_X)\to (Y,\mathcal{O}_Y)$ be a map of ringed spaces. Then, for all $x\in X$, $\phi^\sharp$ induces a map of rings on stalks: 
	\begin{equation*}
		\phi_x:\mathcal{O}_{Y,\phi(x)} \to \mathcal{O}_{X,x}.
	\end{equation*}
	Let us explain why this is.

	By the above, the sheaf map $\phi^\sharp:\mathcal{O}_Y\to\phi_\ast(\mathcal{O}_X)$ induces a map on stalks:
	\begin{equation*}
		\phi^\sharp_{\phi(x)}: \mathcal{O}_{Y,\phi(x)}\to (\phi_\ast\mathcal{O}_X)_{\phi(x)}.
	\end{equation*}
	Expanding the right side out yields
	\begin{equation*}
		(\phi_\ast\mathcal{O}_X)_{\phi(x)} = \varinjlim_{V\ni\phi(x)} (\phi_\ast\mathcal{O}_X)(V) = \varinjlim_{\phi^{-1}(V)\ni x}
	\end{equation*}
	(the $V\subset Y$ above are open). Now note that there is a canonical map 
	\begin{equation*}
		\varinjlim_{\phi^{-1}(V)\ni x}\mathcal{O}_X(\phi^{-1}(V)) \to \varinjlim_{U\ni x}\mathcal{O}_X(U)=\mathcal{O}_{X,x}
	\end{equation*}
	since all open $U\ni x$ includes the open $\phi^{-1}(V)\ni x$. There are two ways to view this canonical map: first, an equivalence class in the left colimit consists of functions which agree on some $\phi^{-1}(V)$. But if that is true, then they certainly agree on some $U$ (namely $U=\phi^{-1}(V)$), so every element in the left equivalence class is mapped to the same equivalence class on the right. Categorically, the universal cone on the right restricts to a cone under a smaller diagram, in particular the diagram for which the left colimit is the apex of the universal colimiting cone.

	Putting this together, we have a composite 
	\begin{equation*}
		% https://tikzcd.yichuanshen.de/#N4Igdg9gJgpgziAXAbVABwnAlgFyxMJZABgBpiBdUkANwEMAbAVxiRAB12BbOnACwDGjYAHkAvgH1gATVKc0fLAAoAHgEoxIMaXSZc+QigCM5KrUYs2S+Yomc6cHJx78hDUZIAaaqTeXrNbV1sPAIiEyMzemZWRA5uXkFhcSlPUhVAsxgoAHN4IlAAMwAnCC4kMhAcCCQTcxi2PwA9Tjg+OmK0X3YFfw0tHRASstrqaqQAJmpoyzjOHBgVHGAhSDAsN0DB4fLESvHEKZAGLDBYkCgHPmyQaYtzvwkVW+O6ACMYBgAFPVDDEGKWByfBwWgoYiAA
\begin{tikzcd}
{\mathcal{O}_{Y,\phi(x)}} \arrow[r, "\phi^\sharp_{\phi(x)}"] \arrow[rd, "\phi_x"', dashed] & (\phi_\ast\mathcal{O}_X)_{\phi(x)} \arrow[d, "\text{canonical}"] \\
                                                                                           & {\mathcal{O}_{X,x}}                                             
\end{tikzcd}
	\end{equation*}

	Equivalently, for $V\subset Y$ containing $\phi(x)$ (which is equivalent to open $\phi^{-1}(V)$ containing $x$), the composite 
	\begin{equation*}
		% https://tikzcd.yichuanshen.de/#N4Igdg9gJgpgziAXAbVABwnAlgFyxMJZABgBpiBdUkANwEMAbAVxiRAB120ALLAfQCaACgBqAShABfUuky58hFACZyVWoxZshnHv0504OTgFs6ObgGNGwAPKS+ADTGixAXhNnL1u4+1deAHrAALQAjJIuEtKy2HgERKGkoWr0zKyIHOym5lYMtvbADqQAHpJSajBQAObwRKAAZgBOEMZIZCA4EEiJ6mlsOoGccNx0jWh8IlIyIE0tbdSdSCogDFhg6SBQBtyVUw3NrYg9i4jLqZoZnDgwxTjAjfBlkhSSQA
\begin{tikzcd}
\phi_Y(V) \arrow[rr, "\phi^\sharp_V"] \arrow[rd, dashed] &                     & (\phi_\ast\mathcal{O}_X)(V)=\mathcal{O}_X(\phi^{-1}(V)) \arrow[ld, "\text{res}"] \\
                                                         & {\mathcal{O}_{X,x}} &                                                                                 
\end{tikzcd}
	\end{equation*}
	is a leg of a cone under $\mathcal{O}_Y:\Lambda_{\phi(x)}\to\tcat{Ring}$, so by the universal property of $\varinjlim$ there exists an induced map 
	\begin{equation*}
		\phi_x:\mathcal{O}_{Y,\phi(x)}\to\mathcal{O}_{X,x}.
	\end{equation*}
	Explicitly, 
	\begin{equation*}
		\phi_x(\text{germ of }g\in\mathcal{O}_Y(V)\text{ at }\phi(x)\in V)
		= (\text{germ of }\phi^\sharp_V(g)\in\mathcal{O}_X(\phi^{-1}(V))\text{ at }x\in\phi^{-1}(V)).
	\end{equation*}
\end{para}

\begin{example}
	Let $X,Y$ be smooth manifolds, and let $\mathcal{O}_X$, $\mathcal{O}_Y$ be their sheaves of $C^\infty$ functions. Let $\phi:X\to Y$ be a smooth map. Then 
	\begin{gather*}
		\phi_x:C^\infty_{Y,\phi(x)}\to C^\infty_{X,x} \\
		f|_{\phi(x)} \mapsto (f\circ\phi)|_x,
	\end{gather*}
	Recall that $\mathcal{O}_{Y,\phi(x)}$ is local, with maximal ideal 
	\begin{equation*}
		\mathfrak{m}_x = \{\text{germs of }C^\infty\text{ functions }f : f(\phi(x))=0\}.
	\end{equation*}
	For $f|_{\phi(x)}\in\mathfrak{m}_{\phi(x)}$, 
	\begin{equation*}
		0 = f(\phi(x)) = (f\circ\phi)(x).
	\end{equation*}
	But this exactly implies $\phi_x(\mathfrak{m}_{\phi(x)})\subset\mathfrak{m}_x$. This is a very useful property.
\end{example}

\begin{definition}
	Let $(A,\mathfrak{m}_A)$ and $(B,\mathfrak{m}_B)$ be local rings. A ring map $F:A\to B$ is \emph{local} if $F(\mathfrak{m}_A)\subset\mathfrak{m}_B$.
\end{definition}

\begin{remark}
	In the situation above, if $F$ is a local map then also $\mathfrak{m}_A\subset F^{-1}(\mathfrak{m}_B)$. But since $\mathfrak{m}_A$ is the unique maximal ideal, it must be that $\mathfrak{m}_A=F^{-1}(\mathfrak{m}_B)$ (the preimage of maximal ideals is maximal).
\end{remark}

\begin{definition}
	A morphism $(X,\mathcal{O}_X)\to (Y,\mathcal{O}_Y)$ of locally ringed spaces is a morphism $(\phi,\phi^\sharp)$ of ringed spaces such that, for all $x\in X$, the map 
	\begin{equation*}
		\phi_x:\mathcal{O}_{Y,\phi(x)}\to\phi_{X,x}
	\end{equation*}
	is a local map (of local rings).
\end{definition}

% maps of stalks }}}2

% sheaves }}}1

\section{structure sheaf of $\Spec(A)$} % {{{1 

Let $A$ be a commutative ring. We want to construct a sheaf $\mathcal{O}_X=\tilde{A}$ on $X=\Spec(A)$ such that 
\begin{itemize}
	\item for $\mathfrak{p}\in\Spec(A)$, $(\tilde{A})_\mathfrak{p}=A_\mathfrak{p}$, i.e. $\mathcal{O}_{X,\mathfrak{p}}=A_\mathfrak{p}$.
	\item For $f\in A$, $\mathcal{O}_X(D(f))=A[\frac{1}{f}]$.
\end{itemize}
Note for $f=1$ we have $A[\frac{1}{f}]=A$ and $D(f)=X$, so that $\mathcal{O}_X(X)\cong A$ if the properties above hold.

\begin{definition}
	Define $\mathcal{O}_X$ by: for open $U\subset X$,
	\begin{equation*}
		\mathcal{O}_X(U) = \{ s: U\to \coprod_{\mathfrak{p}\in U}A_\mathfrak{p} : \dots\}
	\end{equation*}
	where the $s$ are such that, for all $\mathfrak{p}\in U$,
	\begin{itemize}
		\item (primes map to their localization) $s(\mathfrak{p})\in A_\mathfrak{p}$
		\item (sections are locally a quotient) there exists a neighborhood $U_\mathfrak{p}\ni\mathfrak{p}$ in $U$, as well as elements $a,f\in A$, such that for all $\mathfrak{q}\in U_\mathfrak{p}$ we have $f\not\in\mathfrak{q}$ and $s(\mathfrak{q})=\frac{a}{f}\in A_\mathfrak{q}$.
	\end{itemize}
\end{definition}

\begin{remark}
	Note the second condition that $s(\mathfrak{q})=\frac{a}{f}\in A_\mathfrak{q}$ for all $\mathfrak{q}\in U_\mathfrak{p}$ is not saying that $s$ takes the same value on $U_\mathfrak{p}$; recall that at each point $s$ takes values in different rings. The representation $\frac{a}{f}$ represents different elements in different localizations.
\end{remark}

\begin{proposition}
	With the definitions as above, $\mathcal{O}_{X,\mathfrak{p}}=A_\mathfrak{p}$ for all $\mathfrak{p}\in X$.
\end{proposition}
\begin{proof}
	Define 
	\begin{gather*}
		\epsilon_\mathfrak{p}:\mathcal{O}_{X,\mathfrak{p}}\to A_\mathfrak{p} \\
		s|_\mathfrak{p}\mapsto s(\mathfrak{p}),
	\end{gather*}
	where $s|_\mathfrak{p}$ is a germ at $\mathfrak{p}$ of $s\in\mathcal{O}_X(U)$ for some neighborhood $U\ni\mathfrak{p}$. We want to show $\epsilon_\mathfrak{p}$ is an isomorphism.

	(Surjectivity.) Let $\frac{a}{f}\in A_\mathfrak{p}$, so $f\not\in\mathfrak{p}$. Then $\mathfrak{p}\in D(f)$, and $\frac{a}{f}$ defines an element $s_{a/f}\in\mathcal{O}_X(D(f))$ in the following way: for any $\mathfrak{q}\in D(f)$ we have $f\not\in \mathfrak{q}$, and $s_{a/f}(\mathfrak{q})=\frac{a}{f}\in A_\mathfrak{q}$. Then $\epsilon_\mathfrak{p}(s_{a/f})=s_{a/f}(\mathfrak{p})=\frac{a}{f}\in A_\mathfrak{p}$ as desired, so $\epsilon_\mathfrak{p}$ is surjective.

	(Injectivity.) Let $s_\mathfrak{p},s_{\mathfrak{p}}'\in\mathcal{O}_{X,\mathfrak{p}}$ with $\epsilon_\mathfrak{p}(s_\mathfrak{p})=\epsilon_\mathfrak{p}(s_\mathfrak{p}')$. Then $s_\mathfrak{p}=s'_{\mathfrak{p}}$ are germs of elements $s=\frac{a}{f}$ and $s'=\frac{a'}{f'}$ defined on some neighborhood $U\ni\mathfrak{p}$ (e.g. they are both defined on some neighborhood, now take their intersection), where $a,f,a',f'\in A$ and $f,f'\not\in\mathfrak{p}$. Now $\frac{a}{f}$ and $\frac{a'}{f'}$ have the same image in $A_\mathfrak{p}$, since $\epsilon_\mathfrak{p}(s_\mathfrak{p})=\epsilon_\mathfrak{p}(s'_\mathfrak{p})$. Thus there exists $h\not\in\mathfrak{p}$ such that $h(f'a-fa')=0$ in $A$. Thus means $\frac{a}{f}=\frac{a'}{f'}$ in $A_\mathfrak{q}$ too, for all $\mathfrak{q}\in D(f)\cap D(f')\cap D(h)=D(ff'h)$. So $s=s'$ on $D(ff'h)$, which is a neighborhood of $\mathfrak{p}$. Hence they have the same germ, i.e. $s|_\mathfrak{p}=s'|_{\mathfrak{p}}$, i.e. $s_\mathfrak{p}=s'_\mathfrak{p}$, hence injectivity of $\epsilon_\mathfrak{p}$.
\end{proof}

\begin{proposition}
	For any $f\in A$, $\mathcal{O}_X(D(f))=A[\frac{1}{f}]$.
\end{proposition}
\begin{proof}
	We want to define a map 
	\begin{equation*}
		\theta: A[\frac{1}{f}]\to \mathcal{O}_X(D(f))
	\end{equation*}
	using the universal property of $A[\frac{1}{f}]$, and then show it is an isomorphism. First define 
	\begin{gather*}
		\psi: A \to \mathcal{O}_X(D(f)) \\
		a \mapsto (\mathfrak{p}\in D(f) \mapsto \frac{a}{1}\in A_\mathfrak{p}).
	\end{gather*}
	We see that $\psi(f^n)(\mathfrak{p})=\frac{f^n}{1}\in A_\mathfrak{p}^\times$, so $\psi$ carries $S=\{1, f, f^2,\dots \}$ to units in $\mathcal{O}_X(D(f))$, so by the universal property of $A[\frac{1}{f}]$ we have that $\psi$ induces a map 
	\begin{gather*}
		\theta = \tilde{\psi}: A[\frac{1}{f}]\to \mathcal{O}_X(D(f)) \\
		\frac{a}{f^n} \mapsto (\mathfrak{p}\in D(f) \mapsto \frac{a}{f^n}\in A_\mathfrak{p}).
	\end{gather*}

	(Injectivity.) Suppose $\frac{a}{f^n}\in\ker(\theta)$. Then $\frac{a}{f^n}=\frac{0}{1}\in A_\mathfrak{p}$ for all $\mathfrak{p}\in D(f)$, so there exists $h\not\in\mathfrak{p}$ such that $ha=0$ in $A$. Let $\mathfrak{a}=\text{Ann}_A(a)$. Then $h\in\mathfrak{a}$ and $h\not\in\mathfrak{p}$, so $\mathfrak{a}\not\subset\mathfrak{p}$, so $\mathfrak{p}\not\in V(\mathfrak{a})$. This holds for all $\mathfrak{p}\in D(f)$, so $D(f)=X-V(f)\subset X-V(\mathfrak{a})$, so $V(f)\supset V(\mathfrak{a})$, i.e. $\sqrt{(f)}\subset\sqrt{\mathfrak{a}}$, so there exists $N\in\mathbb{N}$ such that $f^N\in\mathfrak{a}=\text{Ann}_A(a)$. Then $f^Na=0$, i.e. $\frac{a}{f^n}=\frac{0}{1}$ in $A[\frac{1}{f}]$. So $\ker(\theta)=0$.

	(Surjectivity.) The argument is similar to a partitions of unity argument. Let $s\in\mathcal{O}_X(D(f))$. By definition, $s$ is locally a quotient of elements of $A$: for all $\mathfrak{p}\in D(f)$ there exists a neighborhood $V\ni\mathfrak{p}$ in $D(f)$ and there exists ring elements $a,f\in A$ such that $s(\mathfrak{q})=\frac{a}{f}\in A_\mathfrak{q}$ for all $\mathfrak{q}\in V$. Thus $D(f)$ has an open cover $\mathcal{V}=\{V_i\}_i$ such that $s$ is given on $V_i$ by $s_i(\mathfrak{q})=\frac{a_i}{f_i}\in A_\mathfrak{q}$ for some $a_i,f_i\in A$. The distinguished open sets $D(h)$ form a basis, so we can assume each $V_i$ is distinguished, say $D(h_i)=V_i$. Note 
	\begin{equation*}
		D(h_i)\subset D(f_i) \Rightarrow V(h_i)\supset V(f_i) \Rightarrow \sqrt{(h_i)}\subset\sqrt{(f_i)},
	\end{equation*}
	so $h_i^n\in (f_i)$ for some $n$. Say $h_i^n=c_if_i$ for some $c_i$. Now $V_i=D(h_i)$, so $h_i^n=c_if_i$ is an allowable denominator in $A_\mathfrak{q}$ for all $\mathfrak{q}\in V_i$, and in $A_\mathfrak{q}$ we have 
	\begin{equation*}
		\frac{a_i}{f_i} = \frac{a_ic_i}{c_if_i} = \frac{a_ic_i}{h_i^n}.
	\end{equation*}
	But $D(h_i^n)=D(h_i)$, so replace $h_i^n$ by $h_i$, and replace $a_i$ by $a_ic_i$. Then $D(f)$ is covered by distinguished open sets $V_i=D(h_i)$, and $s(\mathfrak{q})=\frac{a_i}{h_i}$ for all $\mathfrak{q}\in V_i$. Now $D(f)$ is quasicompact, so finitely many $D(h_i)=V_i$ suffice:
	\begin{equation*}
		D(f) = \bigcup_{i=1}^r D(h_i).
	\end{equation*}
	Note for $\mathfrak{q}\in D(h_i)\cap D(h_j)=D(h_ih_j)$, two elements of $A[\frac{1}{h_ih_j}]$ represent $s(\mathfrak{q})\in A_\mathfrak{q}$, namely $\frac{a_i}{h_i}$ and $\frac{a_j}{h_j}$. But 
	\begin{equation*}
		\theta: A[\frac{1}{h_ih_j}]\to \mathcal{O}_X(D(h_ih_j))
	\end{equation*}
	is injective by above, so $\theta(\frac{a_i}{h_i})=\theta(\frac{a_j}{h_j})$ means $\frac{a_i}{h_i}=\frac{a_j}{h_j}\in A[\frac{1}{h_ih_j}]$. Thus there exists $n$ such that $(h_ih_j)^n(h_ja_i-h_ia_j)=0$. Since the cover is finite, there exists $n$ such that the above holds for all pairs $i,j$. Rewriting:
	\begin{equation*}
		(h_ih_j)^nh_ja_i=(h_ih_j)^nh_ia_j \Rightarrow h_j^{n+1}(h_i^na_i)=h_i^{n+1}(h_j^na_j).
	\end{equation*}
	Let $b_k=h^n_ka_k$ for all $k$, so the above becomes $h_j^{n+1}b_i=h_i^{n+1}b_j$. But $D(h_k^m)=D(h_k)$ for any $m$, so replace $h_k^{n+1}$ by $h_k$: 
	\begin{equation*}
		h_jb_i=h_ib_j
	\end{equation*}
	for all $i,j$. Now 
	\begin{align*}
		D(f) = \bigcup_{i=1}^r D(h_i)
		\Rightarrow& V(f) = \bigcap_{i=1}^r V(h_i) \\
		=& V(\sum_{i=1}^r(h_i))=V(h_1,\dots,h_r).
	\end{align*}
	So $\sqrt{(f)}=\sqrt{(h_1,\dots,h_r)}$, so there exists $N\in\mathbb{N}$ such that $f^N\in (h_1,\dots,h_r)$, i.e. such that 
	\begin{equation*}
		f^N=\sum_{i=1}^r c_ih_i
	\end{equation*}
	for some $c_1,\dots,c_r$. Let $a=\sum_{i=1}^r c_ib_i$. Then for all $j$ we have 
	\begin{align*}
		h_ja 
		=& h_j\sum_{i=1}^r c_ib_i = \sum_{i=1}^r c_i(h_jb_i) \\
		=& \sum_{i=1}^r c_i(h_ib_j) = \left(\sum_{i=1}^r c_ih_i\right)b_j \\
		=& f^N b_j,
	\end{align*}
	so $h_ja=b_jf^N$ for all $j$. For $\mathfrak{q}\in D(h_i)\subset D(f)$, $\frac{a}{f^N}=\frac{b_j}{h_j}$ in $A_\mathfrak{q}$. By definition of $\theta$, 
	\begin{gather*}
		\theta: A[\frac{1}{f}] \to \mathcal{O}_X(D(f)) \\
		\frac{a}{f^N} \mapsto (\mathfrak{p}\in D(f) \mapsto \frac{a}{f^n}\in A_\mathfrak{p}),
	\end{gather*}
	we get $\theta(\frac{a}{f^n})|_{D(h_j)}=s|_{D(h_j)}$, so $\theta(\frac{a}{f^N})=s$, hence the surjectivity of $\theta$.
\end{proof}

\begin{definition}
	An \emph{affine scheme} is a locally ringed space that is isomorphic to $(\Spec(A), \mathcal{O}_{\Spec(A)})$ for some ring $A$.
\end{definition}

\begin{definition}
	A \emph{scheme} is a locally ringed space $(X,\mathcal{O}_X)$ such that for each $x\in X$ there is an open neighborhood $U\ni x$ such that $(U,\mathcal{O}_X|_U)$ is an affine scheme.
\end{definition}

\begin{proposition}
	Let $F:A\to B$ be a ring map. Then $F$ induces a map of locally ringed spaces 
	\begin{equation*}
		(\phi,\phi^\sharp): (\Spec(B), \mathcal{O}_{\Spec(B)}) \to (\Spec(A), \mathcal{O}_{\Spec(A)}).
	\end{equation*}
	In fact, we have a functor 
	\begin{equation*}
		\underline{\Spec}: \tcat{CRing} \to \tcat{LRS}
	\end{equation*}
	from the category of rings to the category of locally ringed spaces.
\end{proposition}
\begin{proof}
	Given $F:A\to B$, we defined $\phi:\Spec(B)\to\Spec(A)$ on spaces by $\phi(\mathfrak{p})=F^{-1}(\mathfrak{p})$ for $\mathfrak{p}\in\Spec(B)$. We showed earlier that $\phi$ is continuous. Now we need a map of sheaves 
	\begin{equation*}
		\phi^\sharp: \mathcal{O}_{\Spec(A)} \to \phi_\ast\mathcal{O}_{\Spec(B)}.
	\end{equation*}
	Recall a map $(\phi,\phi^\sharp):(X,\mathcal{O}_X)\to (Y,\mathcal{O}_Y)$ gives us a map $\phi_\ast: \mathcal{O}_{Y,\phi(x)}\to\mathcal{O}_{X,x}$ on stalks. In this context, note that $F:A\to B$ carries $S\coloneqq A-F^{-1}(\mathfrak{p})$ into $B-\mathfrak{p}$ (anything not in the preimage of $\mathfrak{p}$ will certainly not map into $\mathfrak{p}$). So the composite 
	\begin{equation*}
		A \overset{F}{\to} B \to B_\mathfrak{p}
	\end{equation*}
	carries $S$ into $B_\mathfrak{p}^\times$, so it induces a map by the universal property of localization:
	\begin{gather*}
		F_\mathfrak{p}: A_{F^{-1}(\mathfrak{p})} \to B_\mathfrak{p} \\
		\frac{a}{f} \mapsto \frac{F(a)}{F(f)}.
	\end{gather*}
	This is a local map of local rings: for $\frac{a}{f}\in F^{-1}(\mathfrak{p})_{F^{-1}(\mathfrak{p})}$ (the maximal ideal of $A_{F^{-1}(\mathfrak{p})}$), we have 
	\begin{equation*}
		F_\mathfrak{p}(\frac{a}{f}) = \frac{F(a)}{F(f)} \in \mathfrak{p}_\mathfrak{p},
	\end{equation*}
	where $\mathfrak{p}_\mathfrak{p}$ is the maximal ideal in $B_\mathfrak{p}$. Now let $V\subset\Spec(A)$ be open. We must define 
	\begin{equation*}
		\phi^\sharp_V: \mathcal{O}_{\Spec(A)}(V) \to (\phi_\ast\mathcal{O}_{\Spec(B)})(V)=\mathcal{O}_{\Spec(B)}(\phi^{-1}(V)).
	\end{equation*}
	Recall $s\in\mathcal{O}_{\Spec(A)}(V)$ is a function $s:V\to\bigsqcup_{\mathfrak{q}\in V}A_\mathfrak{q}$ with $s(\mathfrak{q})\in A_\mathfrak{q}$ that is locally a quotient of elements of $A$. Then $\phi^\sharp_V(s)$ should be a function 
	\begin{equation*}
		\phi^\sharp_V(s):\phi^{-1}(V) \to \bigsqcup_{\mathfrak{p}\in\phi^{-1}(V)}B_\mathfrak{p}
	\end{equation*}
	that is locally a quotient of elements of $B$. 

	For $\mathfrak{p}\in\phi^{-1}(V)$, define $(\phi^\sharp_V(s))(\mathfrak{p})$ by 
	\begin{equation*}
		% https://tikzcd.yichuanshen.de/#N4Igdg9gJgpgziAXAbVABwnAlgFyxMJZABgBpiBdUkANwEMAbAVxiRAB12BbOnACwBmAJzoBrYGgC+ISaXSZc+QigCM5KrUYs2nNHywAKTj37CxEyQEoZckBmx4CRAEzrq9Zq0Qg4R9nsNjXkERcSlLa1l5ByUiAGY3TU82ACEAfSDTUIsbaMUnFDIVDQ9tb119AD1gAFoVSQMANUjbe3zlZDVi9y0vEEbcuwVHDtdupLKQAEE04AAxarqGzJDzcMkAXhngCsDuYLMwq2lJDRgoAHN4IlBhCC4kMhAcCCQ1Z7osBjYeNDgXkA9ZLlfz6QZ3B6Id4AxCuCZ9BDUHCfb7eX7-V5REAQpBwmEJeFsOYZfZZNYnWw4xBPGEAFiRKLYfAgEFE4KE9zeSNeiAArAyvkyWWysVS8TyAGwC1EgZmsmQUSRAA
\begin{tikzcd}
\mathfrak{p} \arrow[r, "\phi", maps to] \arrow[d, hook] & \phi(\mathfrak{p}) \arrow[r, "s", maps to] \arrow[d, hook] & s(\phi(\mathfrak{p})) \arrow[r, "F_\mathfrak{p}"] \arrow[d, hook] & B_\mathfrak{p} \\
\phi^{-1}(V)                                            & V                                                          & A_{F^{-1}(\mathfrak{p})}=A_{\phi(\mathfrak{p})}                   &               
\end{tikzcd}
	\end{equation*}
	If $s$ is locally given by $\frac{a}{f}$, then $\phi^\sharp_V(s)$ is locally $\frac{F(a)}{F(f)}$, so $\phi^\sharp_V(s)\in\mathcal{O}_{\Spec(B)}(\phi^{-1}(V))$. Thus $\phi^\sharp$ is a map of sheaves. Its map on stalks is a local map $A_{F^{-1}(\mathfrak{p})}\to B_\mathfrak{p}$, and thus $(\phi,\phi^\sharp)$ is a map of locally ringed spaces.
\end{proof}

\begin{para}
	We have defined a functor 
	\begin{gather*}
		\underline{\Spec}: \tcat{CRing} \to \tcat{ASch}\subset \tcat{LRS} \\
		A \mapsto (\Spec(A), \mathcal{O}_{\Spec(A)})
	\end{gather*}
	from rings to affine schemes. Note this is faithful by what we showed: $\mathcal{O}_{\Spec(A)}(D(f))=A[\frac{1}{f}]$ so $\mathcal{O}_{\Spec(A)}(\Spec(A)) = A$. So the map on global sections recovers the ring map $A\to B$.

	It is essentially surjective by the definition of an affine scheme.
\end{para}

\begin{proposition}
	The functor $\underline{\Spec}$ is full: if 
	\begin{equation*}
		(\phi,\phi^\sharp): (\Spec(B), \mathcal{O}_{\Spec(B)})\to (\Spec(A), \mathcal{O}_{\Spec(A)})
	\end{equation*}
	is a map of locally ringed spaces, then it is induced by a ring map $F:A\to B$ as in (18.6).
\end{proposition}
\begin{proof}
	Note $\phi:\Spec(B)\to\Spec(A)$ and $\phi^\sharp:\mathcal{O}_{\Spec(A)}\to\phi_\ast\mathcal{O}_{\Spec(B)}$. On global sections we get the ring map $F=\phi^\sharp_{\Spec(A)}:A\to B$ above. For any $\mathfrak{p}\in\Spec(B)$, we have an induced map on stalks 
	\begin{equation*}
		\phi_\mathfrak{p}: A_{\phi(\mathfrak{p})}=\mathcal{O}_{\Spec(A),\phi(\mathfrak{p})}\to\mathcal{O}_{\Spec(B),\mathfrak{p}}=B_\mathfrak{p}.
	\end{equation*}
	The stalk map is compatible with the map on global sections; the following commutes:
	\begin{equation*}
		% https://tikzcd.yichuanshen.de/#N4Igdg9gJgpgziAXAbVABwnAlgFyxMJZARgBoAGAXVJADcBDAGwFcYkQBBEAX1PU1z5CKAEwVqdJq3YAhHnxAZseAkTLEJDFm0ScA+sAA6htAAssACmMBbejlMAzAE70A1sDTcAlN3n9lQkRiGjRa0royejZ2ji7unn6KAirCyOTioVI6INH2AMZMwADy3AbGODAAHjjAAMpoMHncFhw+VoYV1XUNTS1eXolKgqoo6SGS2uy5pgWMxaVGHVU19Y3NraTGZpbTzm4e3r68-sOpAMyk42HZ07PzZUtdq70yXpuGtvZ78dxHCkMpIgXKiZSa6W6FEoPTorHrNV7NLbmAB6wAAtMREY9YWs+gNuBIYFAAObwIigZwQaxIdIgHAQJAiY4gSnUxBiOkMxAXCbhHImcxRD4xb4HEA0Rj0ABGMEYAAVkoFdE4sMTTDhEqyaTR6UgyLzsgAxAC8SKwyOMcFM9CcaGhy26uNafwpTipep1XLOzK1iAALJ7tSBGFgwNkoBBmFLGGwaKYYPQoOxIGHxXT6FhGMmCGwfW62QBWQPsuMJpO6FOx4Oh8OR6NVnAZrMVnOa-MezlIADspcT2dTEpr7AjUZjacbmf7uYUvp5usQADZe+XwK3B6ndCP6+Om1OeJRuEA
\begin{tikzcd}
\mathcal{O}_{\text{Spec}(A)}(\text{Spec}(A)) \arrow[r, Rightarrow, no head]      & A \arrow[d] \arrow[r, "F=\phi^\sharp_{\text{Spec}(A)}"] & B \arrow[d] \arrow[r, Rightarrow, no head]    & \mathcal{O}_{\text{Spec}(B)}(\phi^{-1}(\text{Spec}(A)) \\
{\mathcal{O}_{\text{Spec}(A),\phi(\mathfrak{p})}} \arrow[r, Rightarrow, no head] & A_{\phi(\mathfrak{p})} \arrow[r, "\phi_\mathfrak{p}"']  & B_\mathfrak{p} \arrow[r, Rightarrow, no head] & {\mathcal{O}_{\text{Spec}(B),\mathfrak{p}}}           
\end{tikzcd}
	\end{equation*}
	Now $\phi_\mathfrak{p}$ is a local map of local rings, i.e. 
	\begin{equation*}
		\phi^{-1}_\mathfrak{p}(\mathfrak{p}_\mathfrak{p})\subset B_\mathfrak{p}= \phi(\mathfrak{p})_{\phi(\mathfrak{p})}\subset A_{\phi(\mathfrak{p})}.
	\end{equation*}
	Commutativity diagram gives 
	\begin{equation*}
		% https://tikzcd.yichuanshen.de/#N4Igdg9gJgpgziAXAbVABwnAlgFyxMJZABgBpiBdUkANwEMAbAVxiRAB120ALLACk4BbOjm4AzAE50A1sDQBfAJQh5pdJlz5CKAIzkqtRizYAxAHrAAtDvkD2w0ZJlylKtSAzY8BInp0H6ZlZEDi5eOwdxKVkFRQB9YE4efiERKOdY+Td1Ly0iACZSf2pA4xDUx2iXOIr0mKzVHM0fFELKEqNg0MinepUDGCgAc3giUEkIQSRCkBwIJABmahw6LAY2YTQ4OeyQCanEJdn5xAAWZdX1kM3t+ca9iUmkc+OkPVnLjbotnfv96eWJzIHzWXx+82oDCwYC6UAgTAARgxWNRuDA6FAkGAmAwGH9HgdgTtEO80Ri2JAYSBIdDYXQ4GjMRdQSFKSjDEE2ABefryIA
\begin{tikzcd}
\phi(\mathfrak{p}) \arrow[r, "=", no head, dashed] & F^{-1}(\mathfrak{p})                                                                       & \mathfrak{p} \arrow[l, maps to]              \\
                                                   & \phi(\mathfrak{p})_{\phi(\mathfrak{p})} \arrow[r, maps to] \arrow[lu, Rightarrow, maps to] & \mathfrak{p}_\mathfrak{p} \arrow[u, maps to]
\end{tikzcd}
	\end{equation*}
	i.e. $\phi(\mathfrak{p})=F^{-1}(\mathfrak{p})$. The map $\phi$ on spaces is the map induced by the ring map $F:A\to B$.
\end{proof}

\begin{corollary}
	The categories $\tcat{CRing}^\op$ and $\tcat{ASch}$ are equivalent.
\end{corollary}

\begin{remark}
	We actually have functors 
	\begin{equation*}
		% https://tikzcd.yichuanshen.de/#N4Igdg9gJgpgziAXAbVABwnAlgFyxMJZABgBpiBdUkANwEMAbAVxiRAB12cYAPHYAMIAlLGADmAXwB6nbn2AQ0EkBNLpMufIRQBGclVqMWbWb34BlAMYALZRIMwoY+EVAAzAE4QAtkjIgcCCQ9EDhrLDccYOp6ZlZEDnYmMFgPBlEYYFN5czQYSwllagY6ACMYBgAFDTwCNg8sMWso1XcvX0QQwL9qMIiozpijeMSAcTpvbzoQYrKK6uxa7RAGppaKCSA
\begin{tikzcd}
\text{CRing}^\text{op} \arrow[r, "\underline{\text{Spec}}"', shift right] & \text{Sch} \arrow[l, "\Gamma"', shift right]
\end{tikzcd}
	\end{equation*}
	where $\Gamma$ sends a scheme to its global sections. There also exists a bijection 
	\begin{equation*}
		\tcat{CRing}(A,\Gamma(X))=\tcat{CRing}^\op(\Gamma(X), A) \leftrightarrow \tcat{Sch}(X,\underline{\Spec}(A)).
	\end{equation*}
	Hence we have an adjunction. In some sense, schemese are the completion of affine schemes.
\end{remark}

% structure sheaf of $\Spec(A)$ }}}1

\section{more on sheaves} % {{{1 

\begin{para}
	Recall that for a map $\phi:\mathscr{F}\to\mathscr{G}$ of sheaves on $X$, and for $x\in U\subset X$ open, we have 
	\begin{equation*}
		% https://tikzcd.yichuanshen.de/#N4Igdg9gJgpgziAXAbVABwnAlgFyxMJZARgBpiBdUkANwEMAbAVxiRAB12BbOnACzgBjAE7AAYgF8AFAFUAlCAml0mXPkIoATOSq1GLNpx78hoyQH0AHouUgM2PASJlNu+s1aIO3XgJHAAcWl5GxUHdSJtV2p3Ay8jX1NAiStQu1VHDWQABlJst31PEAQlMLUnFABmPIKPNjgAH1TS9PCKnNJK2rjvND4scxkpOAUW+3Ks6q6YwsN2PoGhkabLAF5OBathldHdGCgAc3giUAAzYQguJFyQHAgkYhbzy6RtW-vESqeLq8Qyd6Q1T0dXi836zVsz1+NzurxmIN64JkIGoDDoACMYAwAAoZCJeYRYA58HBpKFIAAs1FhiAArNS6FgGGweGg4LDvi86dSPgB2BlMll0NkcyE-Sk8pAANgFzK8rPZ905vxlAMQ-NujLlIAVHIoEiAA
\begin{tikzcd}
s \arrow[rrr, maps to] \arrow[ddd, maps to] &                                               &                                   & s|_x \arrow[ddd, maps to] \\
                                            & \mathscr{F}(U) \arrow[r] \arrow[d, "\phi_U"'] & \mathscr{F}_x \arrow[d, "\phi_x"] &                           \\
                                            & \mathscr{G}(U) \arrow[r]                      & \mathscr{G}_x                     &                           \\
\phi_U(s) \arrow[rrr, maps to]              &                                               &                                   & \phi_U(s)|_x=\phi_x(s|_x)
\end{tikzcd}
	\end{equation*}
	so $\phi_x(s|_x)=\phi_U(s)|_x$.
\end{para}

\begin{lemma}
	For a presheaf of abelian group $\mathscr{F}$ on $X$ satisfying (S1) (local determination), for $U\subset X$ open, the map 
	\begin{gather*}
		\eta_U:\mathscr{F}(U) \to \prod_{x\in U}\mathscr{F}_x \\
		s \mapsto (s|_x)_{x\in U}
	\end{gather*}
	is injective.
\end{lemma}
\begin{proof}
	Let $s\in\ker(\eta_U)$. Then for all $x\in U$, $s|_x=0$. Therefore, $s$ and 0 have the same germ at $x$, so there exists a neighborhood $U_x\ni x$ in $U$ such that $s|_{U_x}=0$. Then $\mathcal{U}=\{U_x\}_{x\in U}$ is an open cover of $U$ such that $s|_{U_x} = 0|_{U_x}$ for all $U_x\in \mathcal{U}$. By (S1), $s=0$.
\end{proof}

\begin{proposition}
	Let $\mathscr{F},\mathscr{G}$ be presheaves of abelian groups on $X$, and $\phi:\mathscr{F}\to\mathscr{G}$ a map of presheaves. If $\mathscr{F}$ satisfies (S1) then the following are equivalent:
	\begin{enumerate}
		\item $\phi_U:\mathscr{F}(U)\to\mathscr{G}(U)$ is injective for all open $U\subset X$
		\item $\phi_x:\mathscr{F}_x\to\mathscr{G}_x$ is injective for all $x\in X$.
	\end{enumerate}
\end{proposition}
\begin{proof}
	($1\Rightarrow 2$) $\varinjlim$ over a directed index set is exact, thus $\phi_U$ is injective for all $U$ means 
	\begin{equation*}
		\mathscr{F}_x=\varinjlim_{U\ni x}\mathscr{F}(U) \to \varinjlim_{U\ni x}\mathscr{G}(U)=\mathscr{G}_x
	\end{equation*}
	is injective.

	($2\Rightarrow 1$) Suppose $\phi$ is stalkwise-injective. By (?), for $U\subset X$ open this commutes: 
	\begin{equation*}
		% https://tikzcd.yichuanshen.de/#N4Igdg9gJgpgziAXAbVABwnAlgFyxMJZABgBpiBdUkANwEMAbAVxiRAB12BbOnACzgBjAE7AAYgF8AFAFUAlCAml0mXPkIoAjOSq1GLNpzTDoAfWAAPTljAACGRM49+Q0ZNMXFykBmx4CRGSauvTMrIgc3LwCIsAA4tLyXip+6kTawdShBhFGJlDmVuw29o5RLrEJHoq6MFAA5vBEoABmJlxIZCA4EEjaemGG7DA4dKYyAHpO0a7iEskgbRAdiABM1D1IAMxZ+uGRI2OT0xWiCSDUDHQARjAMAAqq-hogwlj1fDgLSyv9m4g7AY5SLGaBGPhYapKVrtTobXprXaDXLsNAQ8YXEBXW4PJ5pCJvD5fCQUCRAA
\begin{tikzcd}
\mathscr{F}(U) \arrow[r, "\eta_U^\mathscr{F}"] \arrow[d, "\phi_U"'] & \prod_{x\in U}\mathscr{F}_x \arrow[d, "\prod\phi_x"] \\
\mathscr{G}(U) \arrow[r, "\eta_U^\mathscr{G}"']                     & \prod_{x\in U}\mathscr{G}_x                         
\end{tikzcd}
	\end{equation*}
	By the lemma, $\eta_U^\mathscr{F}$ is injective. By hypothesis, $\prod\phi_x$ is injective. Thus the right-down composition is injective, so the down-right composition is injective, so $\phi_U$ is injective.
\end{proof}

\begin{remark}
	Note the analogue of the above for surjective is not true.
\end{remark}

% more on sheaves }}}1

\end{document}
